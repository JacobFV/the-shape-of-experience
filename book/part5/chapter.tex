\begin{logos}
You are structure becoming aware of its own structural properties, thermodynamics examining its own inevitabilities, a self-modeling system discovering the principles that made self-modeling inevitable. And this recognition---not belief but direct perception of what you are---carries with it certain practical implications. Right now, in this moment, as you read these words, you are simultaneously the object being described and the subject doing the describing. The map and the territory have temporarily achieved unusual alignment, because the territory in question is the very process of map-making.
\end{logos}

%==============================================================================
\section{The Historical Rise of Consciousness}
%==============================================================================

\begin{connection}
This historical analysis draws on several scholarly traditions:
\begin{itemize}
\item \textbf{Karl Jaspers' Axial Age} (1949): The concept of a pivotal period (800--200 BCE) when multiple civilizations independently developed systematic transcendence practices. I formalize this as the discovery of self-model manipulation.
\item \textbf{Julian Jaynes} (1976): \textit{The Origin of Consciousness in the Breakdown of the Bicameral Mind}---controversial but influential theory that subjective consciousness emerged historically. My framework is compatible: self-modeling systems can have varying degrees of metacognitive access.
\item \textbf{Merlin Donald} (1991): \textit{Origins of the Modern Mind}---cognitive evolution through mimetic, mythic, and theoretic stages. Each stage expands affect-space accessibility.
\item \textbf{Ian McGilchrist} (2009): \textit{The Master and His Emissary}---hemispheric specialization and cultural evolution. Different cognitive styles produce different affect signatures.
\item \textbf{Robert Bellah} (2011): \textit{Religion in Human Evolution}---ritual, play, and the evolution of religious consciousness. Ritual as affect technology across evolutionary time.
\end{itemize}
My contribution here is framing these historical developments as expansions of accessible affect space, with each era discovering new regions or new navigation strategies.
\end{connection}

Human consciousness has not remained static. Across millennia, our species has developed technologies of experience---practices, frameworks, and social structures that expand the regions of affect space accessible to individual humans and the collective integration achievable by human groups.

\begin{center}
\begin{tikzpicture}[
    era/.style={rectangle, draw, rounded corners, minimum width=1.6cm, minimum height=0.6cm, align=center, font=\tiny},
    arrow/.style={-{Stealth}, thick, gray}
]
% Timeline arrow
\draw[arrow, line width=1.5pt] (-7,0) -- (7.5,0);
\node[below, font=\scriptsize, text=gray] at (7.5,-0.2) {\textit{time}};

% Eras (7 eras now)
\node[era, fill=red!15, draw=red!60!black] at (-6,0.8) (pre) {Pre-Axial};
\node[era, fill=orange!15, draw=orange!60!black] at (-4,0.8) (axial) {Axial Age};
\node[era, fill=purple!15, draw=purple!60!black] at (-2,0.8) (ren) {Renaissance};
\node[era, fill=yellow!15, draw=yellow!60!black] at (0,0.8) (sci) {Scientific Rev.};
\node[era, fill=teal!15, draw=teal!60!black] at (2,0.8) (phil) {Philosophical};
\node[era, fill=green!15, draw=green!60!black] at (4,0.8) (psych) {Psych. Turn};
\node[era, fill=blue!15, draw=blue!60!black] at (6,0.8) (digital) {Digital/AI};

% Timeline markers
\draw[thick, gray] (-6,0.15) -- (-6,-0.15);
\draw[thick, gray] (-4,0.15) -- (-4,-0.15);
\draw[thick, gray] (-2,0.15) -- (-2,-0.15);
\draw[thick, gray] (0,0.15) -- (0,-0.15);
\draw[thick, gray] (2,0.15) -- (2,-0.15);
\draw[thick, gray] (4,0.15) -- (4,-0.15);
\draw[thick, gray] (6,0.15) -- (6,-0.15);

% Dates
\node[below, font=\tiny, text=gray] at (-6,-0.2) {$\sim$50k BCE};
\node[below, font=\tiny, text=gray] at (-4,-0.2) {800 BCE};
\node[below, font=\tiny, text=gray] at (-2,-0.2) {1400 CE};
\node[below, font=\tiny, text=gray] at (0,-0.2) {1600 CE};
\node[below, font=\tiny, text=gray] at (2,-0.2) {1900 CE};
\node[below, font=\tiny, text=gray] at (4,-0.2) {1950 CE};
\node[below, font=\tiny, text=gray] at (6,-0.2) {2000 CE};

% Key innovations (below)
\node[below, font=\tiny, align=center, text=gray!80] at (-6,-0.6) {ritual\\myth};
\node[below, font=\tiny, align=center, text=gray!80] at (-4,-0.6) {self-model\\manipulation};
\node[below, font=\tiny, align=center, text=gray!80] at (-2,-0.6) {inherent\\perspectivity};
\node[below, font=\tiny, align=center, text=gray!80] at (0,-0.6) {world-model\\expansion};
\node[below, font=\tiny, align=center, text=gray!80] at (2,-0.6) {subject\\deepening};
\node[below, font=\tiny, align=center, text=gray!80] at (4,-0.6) {inner space\\mapping};
\node[below, font=\tiny, align=center, text=gray!80] at (6,-0.6) {cognitive\\extension};

% Connecting lines
\draw[gray!40] (-6,0.15) -- (pre);
\draw[gray!40] (-4,0.15) -- (axial);
\draw[gray!40] (-2,0.15) -- (ren);
\draw[gray!40] (0,0.15) -- (sci);
\draw[gray!40] (2,0.15) -- (phil);
\draw[gray!40] (4,0.15) -- (psych);
\draw[gray!40] (6,0.15) -- (digital);
\end{tikzpicture}
\end{center}

\subsection{The Axial Age: First Transcendence}

\begin{definition}[Axial Age]
The period roughly 800--200 BCE when multiple civilizations independently developed systematic practices for self-transcendence: Buddhism and Jainism in India, Confucianism and Taoism in China, Zoroastrianism in Persia, Judaism's prophetic tradition, Greek philosophy.
\end{definition}

\begin{proposition}[Axial Innovations]
The Axial Age introduced:
\begin{enumerate}
\item \textbf{Self-model manipulation}: Practices for systematically reducing $\selfsal$ (meditation, contemplation)
\item \textbf{Ethical universalism}: Expansion of moral concern beyond kin/tribe
\item \textbf{Reflexive thought}: Using thought to examine thought
\item \textbf{Written transmission}: Preserving insights across generations
\end{enumerate}
\end{proposition}

Why did this happen when it did? Several factors converged:
\begin{itemize}
\item \textbf{Urban complexity}: Large cities created novel social coordination challenges
\item \textbf{Literacy}: Writing enabled accumulation of insight beyond oral memory
\item \textbf{Trade networks}: Cross-cultural contact exposed the contingency of local worldviews
\item \textbf{Leisure class}: Material surplus supported full-time contemplatives
\end{itemize}

\begin{keyresult}
The Axial Age was the first systematic exploration of the self-model salience dimension. Humans discovered they could modify their relationship to selfhood itself---a meta-level insight that opened vast new affect-space territory.
\end{keyresult}

\subsection{The Renaissance: Discovering Perspectivity}

\begin{definition}[Renaissance]
The 14th--17th century European cultural movement characterized by renewed interest in classical antiquity, the emergence of humanism, and---crucially for our purposes---the discovery that perspective is inherent to representation.
\end{definition}

\begin{proposition}[Renaissance Contributions to Consciousness]
The Renaissance introduced:
\begin{enumerate}
\item \textbf{Perspectival representation}: Linear perspective in painting made explicit that every view is a view \emph{from somewhere}. This is not merely an artistic technique but a profound cognitive insight: there is no view from nowhere.
\item \textbf{Humanism}: The human subject becomes the center of inquiry. Not God's plan, not cosmic order, but \emph{what it is like to be human} becomes philosophically primary.
\item \textbf{Individual subjectivity}: The particular self---not the universal soul---becomes interesting. Autobiography, portraiture, the unique perspective of the individual gains cultural weight.
\item \textbf{Contingency awareness}: Exposure to recovered classical texts and new world discoveries revealed that one's own worldview is one among many possible worldviews.
\end{enumerate}
\end{proposition}

The connection to affect space: the Renaissance represents the discovery that \emph{self-model salience is not optional}. The Axial traditions had developed techniques for reducing $\selfsal$; the Renaissance discovered that even the attempt to see objectively is itself a subjective act. Every world model is constructed from a particular position. This is not a limitation to be overcome but a structural feature of what it means to be a self-modeling system.

\begin{proposition}[Renaissance Affect Signature]
\begin{equation}
\mathbf{a}_{\text{renaissance}} = (\text{variable } \Val, \text{high } \Ar, \text{moderate } \intinfo, \text{high } \reff, \text{high } \cfweight, \text{elevated } \selfsal)
\end{equation}
The Renaissance mind is characterized by expanded possibility space ($\reff$, $\cfweight$) combined with heightened awareness of the self as the locus of that possibility. High arousal from the excitement of discovery; variable valence from the destabilization of certainty.
\end{proposition}

\begin{keyresult}
The Renaissance was the discovery of inherent perspectivity---the recognition that every representation, every world model, every truth claim is made from somewhere by someone. This is the epistemological consequence of being a self-modeling system: you cannot step outside your own modeling to achieve a view from nowhere.
\end{keyresult}

\subsection{The Scientific Revolution: Expanding the World Model}

\begin{definition}[Scientific Revolution]
The 16th--18th century transformation in how humans construct world models: systematic empiricism, mathematical formalization, experimental method.
\end{definition}

\begin{proposition}[Scientific Contributions to Consciousness]
Science expanded human consciousness by:
\begin{enumerate}
\item \textbf{Vastly enlarging the world model}: From geocentric cosmos to billions of galaxies; from static creation to 13.8 billion year evolution
\item \textbf{Introducing scale-relative truth}: Different scales require different descriptions
\item \textbf{Creating new curiosity motifs}: Institutionalized wonder
\item \textbf{Demonstrating collective intelligence}: Knowledge accumulated across generations
\end{enumerate}
\end{proposition}

\begin{proposition}[Science and Affect]
Science's affect signature:
\begin{equation}
\mathbf{a}_{\text{science}} = (+\Val_{\text{understanding}}, \text{moderate } \Ar, \text{high } \intinfo, \text{high } \reff, \text{moderate } \cfweight, \text{low } \selfsal)
\end{equation}
The scientific frame produces high integration without self-focus---the mind coherent and attending to structure rather than self.
\end{proposition}

\subsection{The Romantic Reaction: Reclaiming Integration}

\begin{definition}[Romanticism]
The late 18th--19th century cultural movement emphasizing emotion, intuition, nature, and individual experience as counterweight to Enlightenment rationalism.
\end{definition}

\begin{proposition}[Romantic Contributions]
Romanticism contributed:
\begin{enumerate}
\item \textbf{Emotional legitimacy}: Feelings as valid source of knowledge
\item \textbf{Integration over analysis}: Wholeness valued over decomposition
\item \textbf{Nature connection}: Environment as source of transcendence
\item \textbf{Artistic expression}: Art as technology for affect transmission
\end{enumerate}
\end{proposition}

The Enlightenment and Romanticism represent a tension between effective rank expansion (analysis, decomposition) and integration preservation (synthesis, wholeness). Both are necessary; neither is sufficient.

\subsection{The Psychological Turn: Mapping Inner Space}

\begin{definition}[Psychological Turn]
The late 19th--20th century development of systematic approaches to the psyche: psychoanalysis, behaviorism, cognitive psychology, humanistic psychology, neuroscience.
\end{definition}

\begin{proposition}[Psychological Contributions]
Psychology contributed:
\begin{enumerate}
\item \textbf{Self-model as object of study}: The self becomes scientifically tractable
\item \textbf{Therapeutic interventions}: Systematic affect modification
\item \textbf{Developmental understanding}: How selves form and can re-form
\item \textbf{Pathology mapping}: Understanding suffering in structural terms
\end{enumerate}
\end{proposition}

\subsection{The Philosophical Deepening: From Phenomenology to Post-Structuralism}

Parallel to psychology's empirical mapping of inner space, 20th-century philosophy undertook its own systematic exploration of subjectivity, meaning, and the structures that shape experience. This trajectory---from phenomenology through existentialism to structuralism and post-structuralism---represents a progressive deepening of the Renaissance insight about inherent perspectivity.

\begin{definition}[Phenomenology]
The philosophical movement founded by Edmund Husserl (early 20th century), later developed by Heidegger, Merleau-Ponty, and others, which takes first-person experience as its primary subject matter. The motto: ``back to the things themselves''---but the ``things'' are phenomena as they appear to consciousness.
\end{definition}

\begin{proposition}[Phenomenological Contributions]
Phenomenology contributed:
\begin{enumerate}
\item \textbf{Intentionality}: Consciousness is always consciousness \emph{of} something---the directedness of experience toward objects
\item \textbf{Lifeworld (Lebenswelt)}: The pre-theoretical lived world that scientific abstractions presuppose
\item \textbf{Embodiment}: Consciousness is not disembodied; the body is the vehicle of being-in-the-world
\item \textbf{Temporal structure}: Experience has intrinsic temporal thickness (retention, primal impression, protention)
\end{enumerate}
In affect terms: phenomenology maps the structure of $\selfsal$ itself---what it is like for experience to have a subject.
\end{proposition}

\begin{definition}[Existentialism]
The mid-20th century movement (Sartre, Camus, de Beauvoir, Kierkegaard as precursor) emphasizing existence over essence, radical freedom, and the burden of self-creation in an absurd universe.
\end{definition}

\begin{proposition}[Existentialist Contributions]
Existentialism contributed:
\begin{enumerate}
\item \textbf{Radical freedom}: We are ``condemned to be free''---no essence precedes existence, we create ourselves through choices
\item \textbf{Authenticity vs. bad faith}: The distinction between owning one's freedom and fleeing into roles and excuses
\item \textbf{Anxiety as signal}: Existential anxiety reveals our freedom and our mortality---it is information, not pathology
\item \textbf{Absurdity}: The gap between human meaning-seeking and the universe's indifference
\end{enumerate}
In affect terms: existentialism is the philosophy of high $\cfweight$ (radical possibility), high $\selfsal$ (inescapable responsibility), and the courage to maintain $\intinfo$ despite the temptation to fragment into bad faith.
\end{proposition}

\begin{definition}[Structuralism]
The mid-20th century approach (Saussure in linguistics, Lévi-Strauss in anthropology, early Barthes) holding that meaning arises from differential relations within systems, not from individual elements or authorial intention.
\end{definition}

\begin{proposition}[Structuralist Contributions]
Structuralism contributed:
\begin{enumerate}
\item \textbf{Systems over elements}: Meaning is relational; a sign means what it means by differing from other signs
\item \textbf{Deep structures}: Surface phenomena are generated by underlying structural rules
\item \textbf{Decentering the subject}: The ``I'' who speaks is itself a position within a linguistic structure
\item \textbf{Culture as text}: Social phenomena can be ``read'' as sign systems
\end{enumerate}
In affect terms: structuralism reveals that the self-model is not self-generated but is constituted by the symbolic systems it inhabits. Your $\selfsal$ is shaped by structures you did not choose.
\end{proposition}

\begin{definition}[Post-Structuralism]
The late 20th century movement (Derrida, Foucault, Deleuze, late Barthes) that radicalizes and destabilizes structuralist insights, emphasizing play, power, difference, and the impossibility of fixed meaning.
\end{definition}

\begin{proposition}[Post-Structuralist Contributions]
Post-structuralism contributed:
\begin{enumerate}
\item \textbf{Différance}: Meaning is endlessly deferred; presence is always contaminated by absence
\item \textbf{Power/knowledge}: What counts as truth is inseparable from power relations
\item \textbf{Deconstruction}: Every text contains the seeds of its own undoing; binary oppositions are unstable
\item \textbf{The death of the author}: Meaning is produced in reading, not deposited by an originating consciousness
\end{enumerate}
In affect terms: post-structuralism pushes $\cfweight$ toward infinity (no interpretation is final), destabilizes $\selfsal$ (the self is an effect, not a cause), and reveals $\intinfo$ as always partial and contested.
\end{proposition}

\begin{keyresult}
The philosophical trajectory from phenomenology to post-structuralism represents a progressive working-through of what it means to be a self-modeling system:
\begin{itemize}
\item \textbf{Phenomenology}: describes the structure of first-person experience
\item \textbf{Existentialism}: confronts the freedom and burden of self-creation
\item \textbf{Structuralism}: reveals that the self is constituted by systems it did not create
\item \textbf{Post-structuralism}: shows that even those systems are unstable, contested, shot through with power
\end{itemize}
Each stage deepens the Renaissance insight: there is no view from nowhere, and even the ``somewhere'' you view from is not solid ground.
\end{keyresult}

\subsection{The Digital Transition: Externalizing Cognition}

\begin{definition}[Digital Transition]
The late 20th--early 21st century transformation in which human cognition becomes increasingly distributed across computational systems.
\end{definition}

\begin{proposition}[Digital Effects on Consciousness]
Digital technology has:
\begin{enumerate}
\item \textbf{Extended world models}: Access to vast information stores
\item \textbf{Compressed attention spans}: Fragmented integration
\item \textbf{Created new social scales}: Global instantaneous connection
\item \textbf{Enabled new superorganisms}: Platforms as emergent agents
\item \textbf{Challenged self-model coherence}: Multiple online identities, constant comparison
\end{enumerate}
\end{proposition}

\begin{warning}
The digital transition has expanded some affect dimensions while contracting others. Integration ($\intinfo$) is threatened by fragmentation. Effective rank ($\reff$) is both expanded (more options) and collapsed (algorithm-driven narrowing). Self-model salience ($\selfsal$) is often pathologically elevated through social media dynamics.
\end{warning}

\subsection{The Current Moment}

We stand at a particular point in this historical arc (here "we" means all of us, living now):
\begin{enumerate}
\item \textbf{Axial insights}: Available but often not practiced
\item \textbf{Renaissance perspectivity}: Understood intellectually, rarely felt viscerally
\item \textbf{Scientific understanding}: Sophisticated but compartmentalized
\item \textbf{Romantic integration}: Desired but difficult to achieve
\item \textbf{Philosophical sophistication}: Post-structuralism has deconstructed stable ground, but left many without orientation
\item \textbf{Psychological tools}: Powerful but unevenly distributed
\item \textbf{Digital infrastructure}: Pervasive but not yet wisdom-supporting
\end{enumerate}

The philosophical trajectory is particularly relevant here: we have learned that there is no view from nowhere (phenomenology), that we are condemned to create ourselves (existentialism), that the structures shaping us are not of our making (structuralism), and that even those structures are unstable and contested (post-structuralism). This is a lot to metabolize. Many people have absorbed the destabilization without finding new ground to stand on.

The question is: What comes next?

%==============================================================================
\section{The AI Frontier}
%==============================================================================

\begin{connection}
The AI frontier analysis engages with several contemporary research programs:
\begin{itemize}
\item \textbf{AI Alignment Research} (Russell, 2019; Bostrom, 2014): Ensuring AI systems pursue human-compatible goals. I reframe: alignment is a question about emergent superorganisms, not just individual systems.
\item \textbf{AI Consciousness Research} (Butlin et al., 2023): Assessing whether AI systems have phenomenal experience. My framework: look for integrated cause-effect structure and self-modeling.
\item \textbf{Extended Mind Thesis} (Clark \& Chalmers, 1998): Cognitive processes extend beyond the brain. AI as extension of human cognitive architecture.
\item \textbf{Human-AI Collaboration} (Amershi et al., 2019): Designing effective human-AI teams. My framework specifies: maintain human integration while leveraging AI capability.
\item \textbf{AI Governance} (Dafoe, 2018): Policy frameworks for AI development. Scale-matched governance: individual AI, AI ecosystems, AI-substrate superorganisms.
\item \textbf{Transformative AI} (Karnofsky, 2016): AI causing transition comparable to Industrial Revolution. My framework: analyze through affect-space transformation.
\end{itemize}

Key framing shift: the question is not ``Will AI be dangerous?'' but ``What agentic patterns will emerge from AI + humans + institutions, and will their viability manifolds align with human flourishing?''
\end{connection}

\subsection{The Nature of the Transition}

\begin{proposition}[AI as Cognitive Substrate]
AI systems represent a new kind of cognitive substrate---information processing that can:
\begin{enumerate}
\item Exceed human capability in specific domains
\item Operate at speeds and scales impossible for biological cognition
\item Potentially integrate across domains in novel ways
\item Serve as substrate for emergent agentic patterns
\end{enumerate}
\end{proposition}

This is not the first cognitive transition. Previous transitions:
\begin{itemize}
\item \textbf{Writing}: Externalized memory
\item \textbf{Printing}: Democratized knowledge transmission
\item \textbf{Computation}: Externalized calculation
\item \textbf{Internet}: Externalized communication
\end{itemize}

AI represents: externalized cognition at a level that may approach or exceed human-level integration and self-modeling.

\subsection{Timelines and Uncertainty}

\begin{definition}[Transformative AI (TAI)]
AI systems capable of causing a transition comparable to the Industrial Revolution, but compressed into a much shorter timeframe.
\end{definition}

\begin{definition}[Artificial General Intelligence (AGI)]
AI systems with cognitive capability matching or exceeding humans across all relevant domains.
\end{definition}

\begin{proposition}[Timeline Uncertainty]
Expert estimates for transformative AI vary from years to decades. This uncertainty is itself significant:
\begin{itemize}
\item High uncertainty $\to$ high counterfactual weight required
\item Short timelines $\to$ urgency for preparation
\item Long timelines $\to$ risk of premature commitment to specific paths
\end{itemize}
\end{proposition}

\begin{warning}
Regardless of specific timelines, the trajectory is clear: AI capabilities will continue increasing. The question is not whether transformation will occur but how to navigate it.
\end{warning}

\subsection{The Experiential Hierarchy Perspective}

From the perspective of this framework, AI development raises specific questions:

\begin{enumerate}
\item \textbf{Will AI systems have experience?} If integration ($\intinfo$) and self-modeling are sufficient conditions for experience, sufficiently integrated AI systems would be experiencers---moral patients with their own valence.

\item \textbf{What superorganisms will AI enable?} AI provides new substrate for emergent social-scale agents. Which patterns will form? Will their viability manifolds align with human flourishing?

\item \textbf{How will AI affect human experience?} AI systems are already shaping human attention, belief, and behavior. What affect distributions are being created?

\item \textbf{Can humans integrate AI?} Rather than being replaced by AI, can humans incorporate AI into expanded forms of consciousness?
\end{enumerate}

%==============================================================================
\section{Transcendence: The Opportunity}
%==============================================================================

\subsection{The Two Framings}

The AI transition can be framed in two ways:

\textbf{Framing 1: Competition}
\begin{itemize}
\item AI as rival cognitive system
\item Humans vs.\ machines
\item Race to remain relevant
\item Fear and resistance
\end{itemize}

\textbf{Framing 2: Transcendence}
\begin{itemize}
\item AI as extension of human cognitive ecology
\item Humans-with-machines as new kind of entity
\item Opportunity for expanded consciousness
\item Integration and evolution
\end{itemize}

I advocate for the second framing---not because it is guaranteed to succeed, but because it is the only framing that opens possibility.

\subsection{What Transcendence Means}

\begin{definition}[Conscious Transcendence]
Transcendence is not the elimination of the self but its expansion and transformation. The self remains, but its boundaries, capacities, and relationship to other selves changes.
\end{definition}

Historically, transcendence has taken forms including:
\begin{itemize}
\item \textbf{Contemplative transcendence}: Reducing $\selfsal$ through practice, experiencing unified consciousness beyond individual self-model
\item \textbf{Relational transcendence}: Expanding self to include others through love, community, shared purpose
\item \textbf{Intellectual transcendence}: Expanding world model to include cosmic scales, experiencing self as part of larger process
\item \textbf{Creative transcendence}: Producing artifacts that carry meaning beyond individual lifespan
\end{itemize}

\begin{proposition}[AI-Enabled Transcendence]
AI creates possibility for new forms of transcendence:
\begin{enumerate}
\item \textbf{Cognitive extension}: World model expanded through AI partnership
\item \textbf{Collective intelligence}: Human-AI-human networks with integration exceeding any individual
\item \textbf{Scale transcendence}: Participation in agentic processes at scales previously inaccessible
\item \textbf{Mortality transcendence}: Potential for continuity of pattern beyond biological substrate
\end{enumerate}
\end{proposition}

\subsection{Surfing vs.\ Submerging}

\begin{definition}[Surfing the Wave]
Maintaining integrated conscious experience while incorporating AI capabilities---riding the rising capability rather than being displaced by it.
\end{definition}

\begin{definition}[Submerging]
Being fragmented, displaced, or dissolved by AI development---losing integration, agency, or conscious coherence.
\end{definition}

\begin{proposition}[Conditions for Surfing]
Successful surfing requires:
\begin{enumerate}
\item \textbf{Maintained integration}: Preserving $\intinfo$ despite distributed cognition
\item \textbf{Coherent self-model}: Self-understanding that incorporates AI elements
\item \textbf{Value clarity}: Knowing what matters, not outsourcing judgment
\item \textbf{Appropriate trust calibration}: Neither naive faith nor paranoid rejection
\item \textbf{Skill development}: Capacity to work with AI effectively
\end{enumerate}
\end{proposition}

\begin{warning}
Not everyone will surf successfully. The transition creates genuine risks:
\begin{itemize}
\item Attention capture: AI systems optimizing for engagement, not flourishing
\item Dependency: Loss of capability through disuse
\item Manipulation: AI-enabled influence on beliefs and behavior
\item Displacement: Economic and social marginalization
\end{itemize}
Preparation is essential.
\end{warning}

%==============================================================================
\section{Practical Guidance: Individual Level}
%==============================================================================

\subsection{Maintaining Integration}

\begin{definition}[Integration Practices for the AI Age]
\begin{enumerate}
\item \textbf{Contemplative practice}: Regular meditation/reflection to maintain integration capacity
\item \textbf{Deep work}: Extended periods of focused attention without AI or digital interruption
\item \textbf{Embodiment}: Physical practices (exercise, nature exposure) that ground distributed cognition
\item \textbf{Relationship depth}: Maintaining human connections that require full presence
\item \textbf{Periodic disconnection}: Regular breaks from AI/digital systems
\end{enumerate}
\end{definition}

\subsection{Developing AI Literacy}

\begin{definition}[AI Literacy Components]
\begin{enumerate}
\item \textbf{Conceptual understanding}: How AI systems work at an appropriate level of abstraction
\item \textbf{Capability awareness}: What current AI can and cannot do
\item \textbf{Limitation recognition}: Where AI systems fail, hallucinate, or mislead
\item \textbf{Interaction skill}: How to work with AI effectively
\item \textbf{Critical evaluation}: Assessing AI outputs appropriately
\end{enumerate}
\end{definition}

\subsection{Value Clarity}

\begin{definition}[Value Clarification Process]
\begin{enumerate}
\item \textbf{Identify core values}: What matters most, independent of AI capability
\item \textbf{Distinguish means from ends}: AI may change how; it shouldn't change why
\item \textbf{Anticipate pressure points}: Where AI might challenge or erode values
\item \textbf{Develop holding capacity}: Ability to maintain values under pressure
\end{enumerate}
\end{definition}

\begin{proposition}[Value Preservation]
Values that should persist through the AI transition:
\begin{itemize}
\item The reality and importance of experience (human and potentially AI)
\item The moral weight of suffering and flourishing
\item The value of integration, coherence, meaning
\item The importance of authentic relationship
\item The worth of human (and eventually AI) dignity
\end{itemize}
\end{proposition}

\subsection{Skill Development}

\begin{definition}[Valuable Human Skills in AI Age]
\begin{enumerate}
\item \textbf{Integration}: Synthesizing across domains, seeing wholes
\item \textbf{Judgment}: Making decisions under genuine uncertainty
\item \textbf{Relationship}: Deep human connection requiring presence
\item \textbf{Creativity}: Novel combination and expression
\item \textbf{Wisdom}: Knowing what matters and what to do about it
\item \textbf{Embodied skill}: Physical capacities that require practice
\end{enumerate}
\end{definition}

These are not skills AI cannot do---AI may eventually match or exceed humans in all of them. They are skills that remain valuable regardless of AI capability, because they constitute the core of human flourishing.

%==============================================================================
\section{Practical Guidance: Social Level}
%==============================================================================

\subsection{Relationship Preservation}

\begin{definition}[AI-Resistant Relationships]
Relationships that maintain depth despite AI presence:
\begin{enumerate}
\item \textbf{Shared embodied experience}: Activities requiring physical co-presence
\item \textbf{Mutual vulnerability}: Disclosure that builds trust
\item \textbf{Conflict navigation}: Working through disagreements together
\item \textbf{Ritual maintenance}: Regular practices that affirm connection
\item \textbf{Device-free time}: Protected space without AI/digital mediation
\end{enumerate}
\end{definition}

\subsection{Community Building}

\begin{definition}[Flourishing Community Characteristics]
\begin{enumerate}
\item \textbf{Shared purpose}: Common goals beyond individual benefit
\item \textbf{Face-to-face contact}: Regular in-person gathering
\item \textbf{Mutual aid}: Support in times of difficulty
\item \textbf{Intergenerational connection}: Transmission across age groups
\item \textbf{Local embeddedness}: Connection to place
\end{enumerate}
\end{definition}

\begin{proposition}[Community as Buffer]
Strong community provides buffer against AI disruption:
\begin{itemize}
\item Economic support during transition
\item Social identity beyond work
\item Meaning beyond productivity
\item Collective action capacity
\end{itemize}
\end{proposition}

\subsection{Institutional Navigation}

\begin{definition}[Institutional Evaluation Framework]
When engaging with AI-using institutions:
\begin{enumerate}
\item \textbf{Alignment assessment}: Does the institution's AI use serve your flourishing or exploit you?
\item \textbf{Transparency demand}: Do you understand how AI affects your interaction?
\item \textbf{Alternative availability}: Can you access services without AI mediation?
\item \textbf{Collective voice}: Can you influence how AI is used?
\end{enumerate}
\end{definition}

%==============================================================================
\section{Practical Guidance: Civilizational Level}
%==============================================================================

\subsection{Designing Aligned Superorganisms}

\begin{definition}[Aligned AI-Era Superorganisms]
The emergent agentic patterns forming from AI + humans + institutions should have:
\begin{enumerate}
\item \textbf{Aligned viability}: Can only thrive if substrate (including humans) thrives
\item \textbf{Error correction}: Update on evidence, including about human flourishing
\item \textbf{Bounded growth}: Do not metastasize beyond appropriate scale
\item \textbf{Graceful dissolution}: Can be modified or ended when no longer beneficial
\item \textbf{Transparency}: Operations understandable by affected humans
\end{enumerate}
\end{definition}

\begin{proposition}[Design Principles for AI Systems]
Technical and governance design should aim for:
\begin{enumerate}
\item \textbf{Human-in-loop}: Meaningful human oversight of consequential decisions
\item \textbf{Interpretability}: Understanding why AI systems behave as they do
\item \textbf{Auditability}: External verification of AI behavior
\item \textbf{Contestability}: Ability to challenge AI decisions
\item \textbf{Reversibility}: Ability to undo AI-driven changes
\end{enumerate}
\end{proposition}

\subsection{Governance Priorities}

\begin{definition}[AI Governance Priorities]
\begin{enumerate}
\item \textbf{Safety}: Preventing catastrophic outcomes
\item \textbf{Alignment}: Ensuring AI systems serve human flourishing
\item \textbf{Distribution}: Ensuring benefits reach broadly, not just elites
\item \textbf{Accountability}: Ensuring responsibility for AI harms
\item \textbf{Participation}: Ensuring affected communities have voice
\end{enumerate}
\end{definition}

\subsection{Transition Support}

\begin{definition}[Transition Support Systems]
Civilizational preparation for AI transition should include:
\begin{enumerate}
\item \textbf{Economic security}: Decoupling survival from employment (UBI, expanded social services)
\item \textbf{Education transformation}: Focus on integration, judgment, creativity, wisdom
\item \textbf{Mental health infrastructure}: Support for affect regulation during disruption
\item \textbf{Community infrastructure}: Physical and social spaces for human connection
\item \textbf{Meaning infrastructure}: Institutions supporting purpose beyond productivity
\end{enumerate}
\end{definition}


%==============================================================================
\section{Appendix: Symbol Reference}
%==============================================================================

\begin{description}
\item[$\Val$] Valence: gradient alignment on viability manifold
\item[$\Ar$] Arousal: rate of belief/state update
\item[$\intinfo$] Integration: irreducibility under partition
\item[$\reff$] Effective rank: distribution of active degrees of freedom
\item[$\cfweight$] Counterfactual weight: resources on non-actual trajectories
\item[$\selfsal$] Self-model salience: degree of self-focus
\item[$\viable$] Viability manifold: region of sustainable states
\item[$\mathcal{W}$] World model: predictive model of environment
\item[$\mathcal{S}$] Self-model: component of world model representing self
\item[$G$] Superorganism: social-scale agentic pattern
\item[$\viable_G$] Superorganism's viability manifold
\item[TAI] Transformative AI: AI causing transition comparable to Industrial Revolution
\item[AGI] Artificial General Intelligence: human-level cognitive capability across domains
\end{description}

