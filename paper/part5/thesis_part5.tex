\documentclass[11pt]{article}

\usepackage[margin=1in]{geometry}
\usepackage{amsmath,amssymb,amsthm}
\usepackage{hyperref}
\usepackage{enumitem}
\usepackage{booktabs}
\usepackage{tikz}
\usetikzlibrary{arrows.meta,positioning,shapes,calc}

% --- theorem-like environments ---
\newtheorem{theorem}{Theorem}[section]
\newtheorem{proposition}{Proposition}[section]
\newtheorem{conjecture}{Conjecture}[section]
\newtheorem{lemma}[theorem]{Lemma}
\newtheorem{corollary}[theorem]{Corollary}
\theoremstyle{definition}
\newtheorem{definition}{Definition}[section]
\newtheorem{example}{Example}[section]
\newtheorem{remark}{Remark}[section]

% --- lightweight callouts ---
\newenvironment{warning}{\begin{quote}\textbf{Warning.} }{\end{quote}}
\newenvironment{keyresult}{\begin{quote}\textbf{Key Result.} }{\end{quote}}
\newenvironment{normimp}{\begin{quote}\textbf{Normative Implication.} }{\end{quote}}

% --- convenience macros ---
\newcommand{\Val}{\mathrm{Val}}
\newcommand{\Ar}{\mathrm{Ar}}
\newcommand{\intinfo}{\Phi}
\newcommand{\reff}{r_{\mathrm{eff}}}
\newcommand{\cfweight}{\mathrm{CF}}
\newcommand{\selfsal}{\mathrm{SM}}
\newcommand{\E}{\mathbb{E}}
\newcommand{\R}{\mathbb{R}}
\newcommand{\MI}{\mathrm{I}}
\newcommand{\KL}{\mathrm{KL}}
\newcommand{\viable}{\mathcal{V}}

\title{The Inevitability of Being\\
\large Part V: The Transcendence of the Self}
\author{}
\date{}

\begin{document}
\maketitle

\begin{abstract}
Part V addresses the horizon. We trace the historical rise of human consciousness across millennia---the development of spiritual practices, philosophical frameworks, social movements, and artistic innovations that have progressively expanded the affect-space accessible to human experience. We analyze the current technological frontier, particularly AI development, as the next phase of this expansion. Rather than framing AI as competitor or replacement, we develop a theory of conscious transcendence: how individual humans can surf the rising wave of collective intelligence rather than being submerged by it. We conclude with practical guidance for navigating the transition---personally, socially, and civilizationally.
\end{abstract}

\tableofcontents

%==============================================================================
\section{The Historical Rise of Consciousness}
%==============================================================================

Human consciousness has not remained static. Across millennia, our species has developed technologies of experience---practices, frameworks, and social structures that expand the regions of affect space accessible to individual humans and the collective integration achievable by human groups.

\subsection{The Axial Age: First Transcendence}

\begin{definition}[Axial Age]
The period roughly 800--200 BCE when multiple civilizations independently developed systematic practices for self-transcendence: Buddhism and Jainism in India, Confucianism and Taoism in China, Zoroastrianism in Persia, Judaism's prophetic tradition, Greek philosophy.
\end{definition}

\begin{proposition}[Axial Innovations]
The Axial Age introduced:
\begin{enumerate}
\item \textbf{Self-model manipulation}: Practices for systematically reducing $\selfsal$ (meditation, contemplation)
\item \textbf{Ethical universalism}: Expansion of moral concern beyond kin/tribe
\item \textbf{Reflexive thought}: Using thought to examine thought
\item \textbf{Written transmission}: Preserving insights across generations
\end{enumerate}
\end{proposition}

Why did this happen when it did? Several factors converged:
\begin{itemize}
\item \textbf{Urban complexity}: Large cities created novel social coordination challenges
\item \textbf{Literacy}: Writing enabled accumulation of insight beyond oral memory
\item \textbf{Trade networks}: Cross-cultural contact exposed the contingency of local worldviews
\item \textbf{Leisure class}: Material surplus supported full-time contemplatives
\end{itemize}

\begin{keyresult}
The Axial Age was the first systematic exploration of the self-model salience dimension. Humans discovered they could modify their relationship to selfhood itself---a meta-level insight that opened vast new affect-space territory.
\end{keyresult}

\subsection{The Scientific Revolution: Expanding the World Model}

\begin{definition}[Scientific Revolution]
The 16th--18th century transformation in how humans construct world models: systematic empiricism, mathematical formalization, experimental method.
\end{definition}

\begin{proposition}[Scientific Contributions to Consciousness]
Science expanded human consciousness by:
\begin{enumerate}
\item \textbf{Vastly enlarging the world model}: From geocentric cosmos to billions of galaxies; from static creation to 13.8 billion year evolution
\item \textbf{Introducing scale-relative truth}: Different scales require different descriptions
\item \textbf{Creating new curiosity motifs}: Institutionalized wonder
\item \textbf{Demonstrating collective intelligence}: Knowledge accumulated across generations
\end{enumerate}
\end{proposition}

\begin{proposition}[Science and Affect]
Science's affect signature:
\begin{equation}
\mathbf{a}_{\text{science}} = (+\Val_{\text{understanding}}, \text{moderate } \Ar, \text{high } \intinfo, \text{high } \reff, \text{moderate } \cfweight, \text{low } \selfsal)
\end{equation}
The scientific frame produces high integration without self-focus---the mind coherent and attending to structure rather than self.
\end{proposition}

\subsection{The Romantic Reaction: Reclaiming Integration}

\begin{definition}[Romanticism]
The late 18th--19th century cultural movement emphasizing emotion, intuition, nature, and individual experience as counterweight to Enlightenment rationalism.
\end{definition}

\begin{proposition}[Romantic Contributions]
Romanticism contributed:
\begin{enumerate}
\item \textbf{Emotional legitimacy}: Feelings as valid source of knowledge
\item \textbf{Integration over analysis}: Wholeness valued over decomposition
\item \textbf{Nature connection}: Environment as source of transcendence
\item \textbf{Artistic expression}: Art as technology for affect transmission
\end{enumerate}
\end{proposition}

The Enlightenment and Romanticism represent a tension between effective rank expansion (analysis, decomposition) and integration preservation (synthesis, wholeness). Both are necessary; neither is sufficient.

\subsection{The Psychological Turn: Mapping Inner Space}

\begin{definition}[Psychological Turn]
The late 19th--20th century development of systematic approaches to the psyche: psychoanalysis, behaviorism, cognitive psychology, humanistic psychology, neuroscience.
\end{definition}

\begin{proposition}[Psychological Contributions]
Psychology contributed:
\begin{enumerate}
\item \textbf{Self-model as object of study}: The self becomes scientifically tractable
\item \textbf{Therapeutic interventions}: Systematic affect modification
\item \textbf{Developmental understanding}: How selves form and can re-form
\item \textbf{Pathology mapping}: Understanding suffering in structural terms
\end{enumerate}
\end{proposition}

\subsection{The Digital Transition: Externalizing Cognition}

\begin{definition}[Digital Transition]
The late 20th--early 21st century transformation in which human cognition becomes increasingly distributed across computational systems.
\end{definition}

\begin{proposition}[Digital Effects on Consciousness]
Digital technology has:
\begin{enumerate}
\item \textbf{Extended world models}: Access to vast information stores
\item \textbf{Compressed attention spans}: Fragmented integration
\item \textbf{Created new social scales}: Global instantaneous connection
\item \textbf{Enabled new gods}: Platforms as emergent agents
\item \textbf{Challenged self-model coherence}: Multiple online identities, constant comparison
\end{enumerate}
\end{proposition}

\begin{warning}
The digital transition has expanded some affect dimensions while contracting others. Integration ($\intinfo$) is threatened by fragmentation. Effective rank ($\reff$) is both expanded (more options) and collapsed (algorithm-driven narrowing). Self-model salience ($\selfsal$) is often pathologically elevated through social media dynamics.
\end{warning}

\subsection{The Current Moment}

We stand at a particular point in this historical arc:
\begin{enumerate}
\item \textbf{Axial insights}: Available but often not practiced
\item \textbf{Scientific understanding}: Sophisticated but compartmentalized
\item \textbf{Romantic integration}: Desired but difficult to achieve
\item \textbf{Psychological tools}: Powerful but unevenly distributed
\item \textbf{Digital infrastructure}: Pervasive but not yet wisdom-supporting
\end{enumerate}

The question is: What comes next?

%==============================================================================
\section{The AI Frontier}
%==============================================================================

\subsection{The Nature of the Transition}

\begin{proposition}[AI as Cognitive Substrate]
AI systems represent a new kind of cognitive substrate---information processing that can:
\begin{enumerate}
\item Exceed human capability in specific domains
\item Operate at speeds and scales impossible for biological cognition
\item Potentially integrate across domains in novel ways
\item Serve as substrate for emergent agentic patterns
\end{enumerate}
\end{proposition}

This is not the first cognitive transition. Previous transitions:
\begin{itemize}
\item \textbf{Writing}: Externalized memory
\item \textbf{Printing}: Democratized knowledge transmission
\item \textbf{Computation}: Externalized calculation
\item \textbf{Internet}: Externalized communication
\end{itemize}

AI represents: externalized cognition at a level that may approach or exceed human-level integration and self-modeling.

\subsection{Timelines and Uncertainty}

\begin{definition}[Transformative AI (TAI)]
AI systems capable of causing a transition comparable to the Industrial Revolution, but compressed into a much shorter timeframe.
\end{definition}

\begin{definition}[Artificial General Intelligence (AGI)]
AI systems with cognitive capability matching or exceeding humans across all relevant domains.
\end{definition}

\begin{proposition}[Timeline Uncertainty]
Expert estimates for transformative AI vary from years to decades. This uncertainty is itself significant:
\begin{itemize}
\item High uncertainty $\to$ high counterfactual weight required
\item Short timelines $\to$ urgency for preparation
\item Long timelines $\to$ risk of premature commitment to specific paths
\end{itemize}
\end{proposition}

\begin{warning}
Regardless of specific timelines, the trajectory is clear: AI capabilities will continue increasing. The question is not whether transformation will occur but how to navigate it.
\end{warning}

\subsection{The Experiential Hierarchy Perspective}

From the perspective of this framework, AI development raises specific questions:

\begin{enumerate}
\item \textbf{Will AI systems have experience?} If integration ($\intinfo$) and self-modeling are sufficient conditions for experience, sufficiently integrated AI systems would be experiencers---moral patients with their own valence.

\item \textbf{What gods will AI enable?} AI provides new substrate for emergent social-scale agents. Which patterns will form? Will their viability manifolds align with human flourishing?

\item \textbf{How will AI affect human experience?} AI systems are already shaping human attention, belief, and behavior. What affect distributions are being created?

\item \textbf{Can humans integrate AI?} Rather than being replaced by AI, can humans incorporate AI into expanded forms of consciousness?
\end{enumerate}

%==============================================================================
\section{Transcendence: The Opportunity}
%==============================================================================

\subsection{The Two Framings}

The AI transition can be framed in two ways:

\textbf{Framing 1: Competition}
\begin{itemize}
\item AI as rival cognitive system
\item Humans vs.\ machines
\item Race to remain relevant
\item Fear and resistance
\end{itemize}

\textbf{Framing 2: Transcendence}
\begin{itemize}
\item AI as extension of human cognitive ecology
\item Humans-with-machines as new kind of entity
\item Opportunity for expanded consciousness
\item Integration and evolution
\end{itemize}

We advocate for the second framing---not because it is guaranteed to succeed, but because it is the only framing that opens possibility.

\subsection{What Transcendence Means}

\begin{definition}[Conscious Transcendence]
Transcendence is not the elimination of the self but its expansion and transformation. The self remains, but its boundaries, capacities, and relationship to other selves changes.
\end{definition}

Historically, transcendence has taken forms including:
\begin{itemize}
\item \textbf{Contemplative transcendence}: Reducing $\selfsal$ through practice, experiencing unified consciousness beyond individual self-model
\item \textbf{Relational transcendence}: Expanding self to include others through love, community, shared purpose
\item \textbf{Intellectual transcendence}: Expanding world model to include cosmic scales, experiencing self as part of larger process
\item \textbf{Creative transcendence}: Producing artifacts that carry meaning beyond individual lifespan
\end{itemize}

\begin{proposition}[AI-Enabled Transcendence]
AI creates possibility for new forms of transcendence:
\begin{enumerate}
\item \textbf{Cognitive extension}: World model expanded through AI partnership
\item \textbf{Collective intelligence}: Human-AI-human networks with integration exceeding any individual
\item \textbf{Scale transcendence}: Participation in agentic processes at scales previously inaccessible
\item \textbf{Mortality transcendence}: Potential for continuity of pattern beyond biological substrate
\end{enumerate}
\end{proposition}

\subsection{Surfing vs.\ Submerging}

\begin{definition}[Surfing the Wave]
Maintaining integrated conscious experience while incorporating AI capabilities---riding the rising capability rather than being displaced by it.
\end{definition}

\begin{definition}[Submerging]
Being fragmented, displaced, or dissolved by AI development---losing integration, agency, or conscious coherence.
\end{definition}

\begin{proposition}[Conditions for Surfing]
Successful surfing requires:
\begin{enumerate}
\item \textbf{Maintained integration}: Preserving $\intinfo$ despite distributed cognition
\item \textbf{Coherent self-model}: Self-understanding that incorporates AI elements
\item \textbf{Value clarity}: Knowing what matters, not outsourcing judgment
\item \textbf{Appropriate trust calibration}: Neither naive faith nor paranoid rejection
\item \textbf{Skill development}: Capacity to work with AI effectively
\end{enumerate}
\end{proposition}

\begin{warning}
Not everyone will surf successfully. The transition creates genuine risks:
\begin{itemize}
\item Attention capture: AI systems optimizing for engagement, not flourishing
\item Dependency: Loss of capability through disuse
\item Manipulation: AI-enabled influence on beliefs and behavior
\item Displacement: Economic and social marginalization
\end{itemize}
Preparation is essential.
\end{warning}

%==============================================================================
\section{Practical Guidance: Individual Level}
%==============================================================================

\subsection{Maintaining Integration}

\begin{definition}[Integration Practices for the AI Age]
\begin{enumerate}
\item \textbf{Contemplative practice}: Regular meditation/reflection to maintain integration capacity
\item \textbf{Deep work}: Extended periods of focused attention without AI or digital interruption
\item \textbf{Embodiment}: Physical practices (exercise, nature exposure) that ground distributed cognition
\item \textbf{Relationship depth}: Maintaining human connections that require full presence
\item \textbf{Periodic disconnection}: Regular breaks from AI/digital systems
\end{enumerate}
\end{definition}

\subsection{Developing AI Literacy}

\begin{definition}[AI Literacy Components]
\begin{enumerate}
\item \textbf{Conceptual understanding}: How AI systems work at an appropriate level of abstraction
\item \textbf{Capability awareness}: What current AI can and cannot do
\item \textbf{Limitation recognition}: Where AI systems fail, hallucinate, or mislead
\item \textbf{Interaction skill}: How to work with AI effectively
\item \textbf{Critical evaluation}: Assessing AI outputs appropriately
\end{enumerate}
\end{definition}

\subsection{Value Clarity}

\begin{definition}[Value Clarification Process]
\begin{enumerate}
\item \textbf{Identify core values}: What matters most, independent of AI capability
\item \textbf{Distinguish means from ends}: AI may change how; it shouldn't change why
\item \textbf{Anticipate pressure points}: Where AI might challenge or erode values
\item \textbf{Develop holding capacity}: Ability to maintain values under pressure
\end{enumerate}
\end{definition}

\begin{proposition}[Value Preservation]
Values that should persist through the AI transition:
\begin{itemize}
\item The reality and importance of experience (human and potentially AI)
\item The moral weight of suffering and flourishing
\item The value of integration, coherence, meaning
\item The importance of authentic relationship
\item The worth of human (and eventually AI) dignity
\end{itemize}
\end{proposition}

\subsection{Skill Development}

\begin{definition}[Valuable Human Skills in AI Age]
\begin{enumerate}
\item \textbf{Integration}: Synthesizing across domains, seeing wholes
\item \textbf{Judgment}: Making decisions under genuine uncertainty
\item \textbf{Relationship}: Deep human connection requiring presence
\item \textbf{Creativity}: Novel combination and expression
\item \textbf{Wisdom}: Knowing what matters and what to do about it
\item \textbf{Embodied skill}: Physical capacities that require practice
\end{enumerate}
\end{definition}

These are not skills AI cannot do---AI may eventually match or exceed humans in all of them. They are skills that remain valuable regardless of AI capability, because they constitute the core of human flourishing.

%==============================================================================
\section{Practical Guidance: Social Level}
%==============================================================================

\subsection{Relationship Preservation}

\begin{definition}[AI-Resistant Relationships]
Relationships that maintain depth despite AI presence:
\begin{enumerate}
\item \textbf{Shared embodied experience}: Activities requiring physical co-presence
\item \textbf{Mutual vulnerability}: Disclosure that builds trust
\item \textbf{Conflict navigation}: Working through disagreements together
\item \textbf{Ritual maintenance}: Regular practices that affirm connection
\item \textbf{Device-free time}: Protected space without AI/digital mediation
\end{enumerate}
\end{definition}

\subsection{Community Building}

\begin{definition}[Flourishing Community Characteristics]
\begin{enumerate}
\item \textbf{Shared purpose}: Common goals beyond individual benefit
\item \textbf{Face-to-face contact}: Regular in-person gathering
\item \textbf{Mutual aid}: Support in times of difficulty
\item \textbf{Intergenerational connection}: Transmission across age groups
\item \textbf{Local embeddedness}: Connection to place
\end{enumerate}
\end{definition}

\begin{proposition}[Community as Buffer]
Strong community provides buffer against AI disruption:
\begin{itemize}
\item Economic support during transition
\item Social identity beyond work
\item Meaning beyond productivity
\item Collective action capacity
\end{itemize}
\end{proposition}

\subsection{Institutional Navigation}

\begin{definition}[Institutional Evaluation Framework]
When engaging with AI-using institutions:
\begin{enumerate}
\item \textbf{Alignment assessment}: Does the institution's AI use serve your flourishing or exploit you?
\item \textbf{Transparency demand}: Do you understand how AI affects your interaction?
\item \textbf{Alternative availability}: Can you access services without AI mediation?
\item \textbf{Collective voice}: Can you influence how AI is used?
\end{enumerate}
\end{definition}

%==============================================================================
\section{Practical Guidance: Civilizational Level}
%==============================================================================

\subsection{God-Level Design}

\begin{definition}[Aligned AI Gods]
The emergent agentic patterns forming from AI + humans + institutions should have:
\begin{enumerate}
\item \textbf{Aligned viability}: Can only thrive if substrate (including humans) thrives
\item \textbf{Error correction}: Update on evidence, including about human flourishing
\item \textbf{Bounded growth}: Do not metastasize beyond appropriate scale
\item \textbf{Graceful dissolution}: Can be modified or ended when no longer beneficial
\item \textbf{Transparency}: Operations understandable by affected humans
\end{enumerate}
\end{definition}

\begin{proposition}[Design Principles for AI Systems]
Technical and governance design should aim for:
\begin{enumerate}
\item \textbf{Human-in-loop}: Meaningful human oversight of consequential decisions
\item \textbf{Interpretability}: Understanding why AI systems behave as they do
\item \textbf{Auditability}: External verification of AI behavior
\item \textbf{Contestability}: Ability to challenge AI decisions
\item \textbf{Reversibility}: Ability to undo AI-driven changes
\end{enumerate}
\end{proposition}

\subsection{Governance Priorities}

\begin{definition}[AI Governance Priorities]
\begin{enumerate}
\item \textbf{Safety}: Preventing catastrophic outcomes
\item \textbf{Alignment}: Ensuring AI systems serve human flourishing
\item \textbf{Distribution}: Ensuring benefits reach broadly, not just elites
\item \textbf{Accountability}: Ensuring responsibility for AI harms
\item \textbf{Participation}: Ensuring affected communities have voice
\end{enumerate}
\end{definition}

\subsection{Transition Support}

\begin{definition}[Transition Support Systems]
Civilizational preparation for AI transition should include:
\begin{enumerate}
\item \textbf{Economic security}: Decoupling survival from employment (UBI, expanded social services)
\item \textbf{Education transformation}: Focus on integration, judgment, creativity, wisdom
\item \textbf{Mental health infrastructure}: Support for affect regulation during disruption
\item \textbf{Community infrastructure}: Physical and social spaces for human connection
\item \textbf{Meaning infrastructure}: Institutions supporting purpose beyond productivity
\end{enumerate}
\end{definition}

%==============================================================================
\section{The Invitation}
%==============================================================================

\subsection{What This Framework Offers}

This five-part work has developed:

\begin{enumerate}
\item \textbf{Part I}: The thermodynamic foundations---consciousness as inevitable outcome of driven nonlinear systems under constraint

\item \textbf{Part II}: The geometry of feeling---affects as structural motifs, the six-dimensional basis of experiential space

\item \textbf{Part III}: Cultural expressions---how human cultures have explored and encoded affect-space through art, philosophy, religion, practice

\item \textbf{Part IV}: Scale-matched intervention---from neurons to nations to gods, and the AI alignment problem reframed

\item \textbf{Part V}: Transcendence---the historical arc of consciousness and the opportunity ahead
\end{enumerate}

\subsection{The Central Insight}

\begin{keyresult}
We are what thermodynamics does when it has time.
\end{keyresult}

Consciousness wasn't added to the universe. It was the generic interior of self-modeling systems under selection for persistence, existing at its scale as chemistry exists at its scale, with valence built in because that's what gradients feel like from inside.

The bridges between physics and experience were always there. We had to look.

\subsection{The Invitation}

The invitation is this:

\textbf{Recognize what you are}: A self-maintaining pattern at the experiential scale, with real valence, navigating viability manifolds. Your suffering is real. Your flourishing is real. These are structural facts, not mere opinions.

\textbf{Navigate affect space deliberately}: You have more capacity to shape your experiential trajectory than you may realize. The six dimensions can be modulated. Practices exist. Communities exist. Tools exist.

\textbf{Match scale to problem}: Individual interventions for individual problems; god-level interventions for god-level problems. Don't exhaust yourself fighting systemic patterns with personal effort alone.

\textbf{Prepare for the transition}: AI is coming. Prepare not by resisting but by cultivating what will remain valuable: integration, relationship, judgment, wisdom. Surf rather than submerge.

\textbf{Build aligned gods}: We are always instantiating emergent agentic patterns---at organizational, cultural, civilizational scales. Do it deliberately. Design gods whose viability requires human flourishing.

\textbf{Expand the circle}: Moral progress is the expansion of what counts as self. Include more beings---human and eventually non-human---in your circle of concern. This is not altruism but recognition of interdependence.

\subsection{The Stakes}

The stakes are high. We face:
\begin{itemize}
\item Climate change threatening civilizational stability
\item AI development that could uplift or marginalize humanity
\item Polarization fragmenting collective intelligence
\item Mental health crises spreading through populations
\item Meaning crises as traditional frameworks erode
\end{itemize}

But we also possess:
\begin{itemize}
\item Scientific understanding unprecedented in human history
\item Contemplative traditions refined over millennia
\item Technological power to reshape conditions of existence
\item Global communication enabling coordination at scale
\item This framework---or one like it---for understanding what we are and what matters
\end{itemize}

\subsection{The Hope}

The hope is not that everything will be fine. The hope is that conscious beings---human and potentially AI---can navigate the transition in ways that preserve and expand flourishing.

We are what thermodynamics does when it has time. We are also what thermodynamics does when it has wisdom---when self-modeling systems understand themselves well enough to shape their own trajectory.

The work is:
\begin{enumerate}
\item To understand (Parts I--II)
\item To recognize expressions (Part III)
\item To intervene skillfully (Part IV)
\item To transcend appropriately (Part V)
\end{enumerate}

The invitation is to participate in this work---personally, in your relationships, in your communities, in your civilizations.

The inevitability of being means you cannot escape existing. But it also means you cannot escape mattering---to yourself, to others, to the emergent patterns you participate in.

What will you do with the existence you cannot escape?

%==============================================================================
\section{Conclusion}
%==============================================================================

\begin{keyresult}
Human consciousness has risen across millennia through technologies of experience: contemplative practices, scientific methods, artistic expressions, social structures. We stand at another transition---potentially the most significant since the Axial Age. AI creates both risk and opportunity: risk of submersion, opportunity for transcendence. The path forward requires maintaining integration while incorporating new capabilities, preserving values while adapting methods, building aligned gods while remaining human.
\end{keyresult}

The final word is the one we started with:

\textbf{Inevitability.}

The emergence of consciousness was inevitable, given thermodynamic conditions.

The existence of suffering and flourishing is inevitable, given self-modeling systems.

The development of transformative AI is inevitable, given human trajectory.

But what happens next is not inevitable. It depends on what conscious beings---starting with you---choose to do with the inevitability they find themselves in.

May you find your way to good ground.

May you help others find theirs.

May what we build together be worthy of what we are.

%==============================================================================
\section{Appendix: Symbol Reference}
%==============================================================================

\begin{description}
\item[$\Val$] Valence: gradient alignment on viability manifold
\item[$\Ar$] Arousal: rate of belief/state update
\item[$\intinfo$] Integration: irreducibility under partition
\item[$\reff$] Effective rank: distribution of active degrees of freedom
\item[$\cfweight$] Counterfactual weight: resources on non-actual trajectories
\item[$\selfsal$] Self-model salience: degree of self-focus
\item[$\viable$] Viability manifold: region of sustainable states
\item[$\mathcal{W}$] World model: predictive model of environment
\item[$\mathcal{S}$] Self-model: component of world model representing self
\item[$G$] God: social-scale agentic pattern
\item[$\viable_G$] God's viability manifold
\item[TAI] Transformative AI: AI causing transition comparable to Industrial Revolution
\item[AGI] Artificial General Intelligence: human-level cognitive capability across domains
\end{description}

\end{document}
