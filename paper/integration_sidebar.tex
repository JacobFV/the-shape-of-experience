% Sidebar: Estimating Integration via Semantic Decomposition
% Include in Part II or Part III where integration (Φ) is discussed

\begin{tcolorbox}[
  title={\textbf{Sidebar: Operationalizing Integration for Text}},
  colback=gray!5,
  colframe=gray!75,
  fonttitle=\bfseries
]

\textbf{The Problem.} IIT defines integration ($\Phi$) as information lost when a system is partitioned. For neural networks with dense vector representations, computing true $\Phi$ is intractable. For text-based cognitive states (e.g., LLM agent thoughts), we need an alternative.

\textbf{The Method.} We estimate integration via \emph{semantic decomposition}:

\begin{enumerate}
    \item \textbf{Decompose}: Use an LLM to break a thought into atomic components
    \begin{quote}
    \small
    ``I must find resources or die'' $\rightarrow$ \{subject: self, urgency: high, action: find, consequence: death\}
    \end{quote}

    \item \textbf{Concatenate}: Join parts into a string: ``self high find death''

    \item \textbf{Compare}: Compute embedding distance between original and concatenation
\end{enumerate}

\textbf{Interpretation.} High distance = high integration (meaning lost in decomposition). The measure captures \emph{syntactic-semantic coupling}---how much meaning depends on relationships between parts rather than the parts themselves.

\textbf{Empirical Findings.} Testing on 38 thoughts across complexity levels:
\begin{itemize}
    \item Range: $\Phi \in [0.08, 0.68]$---meaningful variation
    \item Highest $\Phi$: Self-referential paradoxes (``What I cannot say is what I need to express'')
    \item Lowest $\Phi$: Relational statements where structure is preserved (``Silence between notes makes music'')
\end{itemize}

\textbf{Limitation.} The measure did not cleanly order our a priori complexity categories (low/medium/high), suggesting it captures a related but distinct construct from theoretical $\Phi$. Best used for \emph{relative comparisons} within agent trajectories rather than absolute thresholds.

\textbf{For Agent Research.} Track $\Phi$ over time as agents approach viability boundaries. Hypothesis: $\Phi$ should increase under threat (more integrated, less decomposable thought patterns). Initial V7 agent data shows $\Phi$ ranges from 0.15-0.44 with higher values during critical survival moments.

\end{tcolorbox}
