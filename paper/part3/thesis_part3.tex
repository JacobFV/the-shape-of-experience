\documentclass[11pt,a4paper]{article}
\usepackage[utf8]{inputenc}
\usepackage[T1]{fontenc}
\usepackage{amsmath,amssymb,amsthm}
\usepackage{mathtools}
\usepackage{physics}
\usepackage{bm}
\usepackage{geometry}
\usepackage{hyperref}
\usepackage{cleveref}
\usepackage{tikz}
\usepackage{pgfplots}
\pgfplotsset{compat=1.18}
\usetikzlibrary{arrows.meta,positioning,shapes,calc,decorations.pathmorphing}
\usepackage{enumitem}
\usepackage{booktabs}
\usepackage{tcolorbox}
\usepackage{fontawesome5}
\usepackage{multirow}
\usepackage{caption}
\usepackage{subcaption}
\usepackage{xcolor}
\usepackage{graphicx}
\graphicspath{{../images/}}

\geometry{margin=1in}

% --- Sidebar and info box environments ---
\newtcolorbox{historical}{
  colback=gray!5!white,
  colframe=gray!60!black,
  title={\faBook\hspace{0.5em}Historical Context},
  fonttitle=\bfseries\small
}

\newtcolorbox{connection}{
  colback=green!5!white,
  colframe=green!60!black,
  title={\faBook\hspace{0.5em}Connection to Existing Theory},
  fonttitle=\bfseries\small
}

\newtcolorbox{empirical}{
  colback=orange!5!white,
  colframe=orange!70!black,
  title={\faFlask\hspace{0.5em}Empirical Grounding},
  fonttitle=\bfseries\small
}

\newtcolorbox{todo_empirical}{
  colback=yellow!10!white,
  colframe=yellow!60!black,
  title={\faClipboardList\hspace{0.5em}\textsc{Future Empirical Work}},
  fonttitle=\bfseries\small
}

\newtcolorbox{sidebar}[1][]{
  colback=blue!3!white,
  colframe=blue!40!black,
  fonttitle=\bfseries\small,
  before upper={\faInfoCircle\hspace{0.5em}},
  #1
}

\newtcolorbox{software}{
  colback=cyan!5!white,
  colframe=cyan!60!black,
  title={\faCode\hspace{0.5em}\textsc{Proposed Software Implementation}},
  fonttitle=\bfseries\small
}

% Theorem environments
\newtheorem{theorem}{Theorem}[section]
\newtheorem{lemma}[theorem]{Lemma}
\newtheorem{proposition}[theorem]{Proposition}
\newtheorem{corollary}[theorem]{Corollary}
\newtheorem{definition}[theorem]{Definition}
\newtheorem{axiom}[theorem]{Axiom}
\newtheorem{remark}[theorem]{Remark}
\newtheorem{example}[theorem]{Example}
\newtheorem{conjecture}[theorem]{Conjecture}
\newtheorem{hypothesis}[theorem]{Hypothesis}

% Custom commands
\newcommand{\E}{\mathbb{E}}
\newcommand{\R}{\mathbb{R}}
\newcommand{\N}{\mathbb{N}}
\newcommand{\Z}{\mathbb{Z}}
\newcommand{\prob}{\mathbb{P}}
\newcommand{\KL}{\mathrm{KL}}
\newcommand{\MI}{\mathrm{I}}
\newcommand{\entropy}{\mathrm{H}}
\newcommand{\argmax}{\operatorname{argmax}}
\newcommand{\argmin}{\operatorname{argmin}}
\newcommand{\tr}{\operatorname{tr}}
\newcommand{\rank}{\operatorname{rank}}
\newcommand{\diag}{\operatorname{diag}}
\newcommand{\sign}{\operatorname{sign}}
\newcommand{\supp}{\operatorname{supp}}
\newcommand{\interior}{\operatorname{int}}
\newcommand{\clos}{\operatorname{cl}}
\newcommand{\conv}{\operatorname{conv}}
\newcommand{\diam}{\operatorname{diam}}
\newcommand{\vol}{\operatorname{vol}}
\newcommand{\manifold}{\mathcal{M}}
\newcommand{\viable}{\mathcal{V}}
\newcommand{\belief}{\mathbf{b}}
\newcommand{\state}{\mathbf{s}}
\newcommand{\action}{\mathbf{a}}
\newcommand{\obs}{\mathbf{o}}
\newcommand{\latent}{\mathbf{z}}
\newcommand{\policy}{\pi}
\newcommand{\value}{V}
\newcommand{\qfunc}{Q}
\newcommand{\reward}{r}
\newcommand{\transition}{T}
\newcommand{\emission}{O}
\newcommand{\freeenergy}{\mathcal{F}}
\newcommand{\intinfo}{\Phi}
\newcommand{\selfmodel}{\mathcal{S}}
\newcommand{\worldmodel}{\mathcal{W}}
\newcommand{\effrank}{r_{\text{eff}}}
\newcommand{\valence}{\mathcal{V}\hspace{-0.8pt}\mathit{al}}
\newcommand{\arousal}{\mathcal{A}\hspace{-0.5pt}\mathit{r}}
\newcommand{\cestructure}{\mathcal{C\!E}}
\newcommand{\phenom}{\mathcal{P}}
\newcommand{\distinction}{\delta}
\newcommand{\relation}{\rho}

% --- styled callout boxes ---
\newtcolorbox{keyresult}{
  colback=blue!5!white,
  colframe=blue!75!black,
  title=Key Result
}
\newtcolorbox{phenomenal}{
  colback=purple!5!white,
  colframe=purple!75!black,
  title=Phenomenal Correspondence
}
\newtcolorbox{warning}{
  colback=red!5!white,
  colframe=red!75!black,
  title=Warning
}

\title{\textbf{The Inevitability of Being}\\[1em]
\Large Part III: Signatures of Affect Under the Existential Burden}
\author{}
\date{}

\begin{document}

\maketitle

\begin{abstract}
Part III examines how human cultural forms---art, sexuality, ideology, science, religion, and technology---serve as responses to the inescapability of self-modeling consciousness. We analyze these phenomena as affect engineering technologies: systematic interventions in experiential structure developed across millennia to manage the existential burden of being a self. We provide detailed structural signatures of aesthetic forms, philosophical traditions, and technological systems, demonstrating how each shapes the probability distributions over human experience. The framework reveals cultural practices as neither arbitrary conventions nor mere survival mechanisms, but as sophisticated technologies for navigating the affect space that consciousness makes inevitable.
\end{abstract}

\tableofcontents
\newpage

%==============================================================================
\section{Notation and Foundational Concepts}
%==============================================================================

This section provides self-contained definitions of the core concepts used throughout Part III. Readers familiar with Parts I--II may skip to Section 2.

\subsection{The Six Affect Dimensions}

\begin{definition}[Valence ($\valence$)]
\textbf{Valence} is the felt quality of approach versus avoidance---the ``goodness'' or ``badness'' of an experiential state. Formally:
\begin{equation}
\valence_t = -\frac{1}{H} \sum_{k=1}^{H} \gamma^k \nabla_{\mathbf{x}} d(\mathbf{x}, \partial\viable) \bigg|_{\hat{\mathbf{x}}_{t+k}} \cdot \frac{d\hat{\mathbf{x}}_{t+k}}{dt}
\end{equation}
Positive valence indicates movement into viable interior; negative valence indicates approach toward viability boundary.
\end{definition}

\begin{definition}[Arousal ($\arousal$)]
\textbf{Arousal} is the rate of belief/state update:
\begin{equation}
\arousal_t = \KL(\belief_{t+1} \| \belief_t)
\end{equation}
High arousal: rapid model updating, activation, intensity. Low arousal: stability, calm, settled state.
\end{definition}

\begin{definition}[Integration ($\intinfo$)]
\textbf{Integration} measures irreducibility of cause-effect structure:
\begin{equation}
\intinfo(\state) = \min_{\text{partitions } P} D\left[ p(\state_{t+1} | \state_t) \| \prod_{p \in P} p(\state^p_{t+1} | \state^p_t) \right]
\end{equation}
High integration: unified experience. Low integration: fragmentation.
\end{definition}

\begin{definition}[Effective Rank ($\effrank$)]
\textbf{Effective rank} measures distribution of active degrees of freedom:
\begin{equation}
\effrank = \frac{(\tr C)^2}{\tr(C^2)} = \frac{\left(\sum_i \lambda_i\right)^2}{\sum_i \lambda_i^2}
\end{equation}
High rank: many dimensions active, openness. Low rank: collapsed into narrow subspace, tunnel vision.
\end{definition}

\begin{definition}[Counterfactual Weight ($\mathcal{CF}$)]
\textbf{Counterfactual weight} is resources devoted to non-actual possibilities:
\begin{equation}
\mathcal{CF}_t = \frac{\text{Compute}_t(\text{imagined rollouts})}{\text{Compute}_t(\text{total})}
\end{equation}
High $\mathcal{CF}$: mind elsewhere (planning, worrying, fantasizing). Low $\mathcal{CF}$: present-focused.
\end{definition}

\begin{definition}[Self-Model Salience ($\mathcal{SM}$)]
\textbf{Self-model salience} is degree of self-focus:
\begin{equation}
\mathcal{SM}_t = \frac{\MI(\latent^{\text{self}}_t; \action_t)}{\entropy(\action_t)}
\end{equation}
High $\mathcal{SM}$: self-conscious, self as prominent object. Low $\mathcal{SM}$: self-forgetting, absorption, flow.
\end{definition}

\subsection{The Affect State}

\begin{definition}[Affect State]
The affect state at time $t$ is characterized by whichever structural dimensions are relevant to the phenomenon under analysis. The full toolkit includes:
\begin{equation}
\mathbf{a}_t = (\valence_t, \arousal_t, \intinfo_t, \effrank_t, \mathcal{CF}_t, \mathcal{SM}_t, \ldots)
\end{equation}
but not all dimensions matter for all phenomena. Cultural forms, practices, and technologies can be characterized by their affect signatures---the structural features they reliably modulate.
\end{definition}

\begin{center}
\begin{tikzpicture}[
    dim/.style={rectangle, draw, rounded corners, minimum width=2.8cm, minimum height=0.6cm, align=center, font=\small},
    label/.style={font=\scriptsize, text=gray}
]
% Central node
\node[circle, draw, fill=gray!10, minimum size=1.2cm, font=\small\bfseries] (center) {$\mathbf{a}_t$};

% Six dimensions arranged in hexagon
\node[dim, fill=red!15, draw=red!60!black, above=1.2cm of center] (val) {Valence $\valence$};
\node[dim, fill=orange!15, draw=orange!60!black, above right=0.6cm and 1.8cm of center] (ar) {Arousal $\arousal$};
\node[dim, fill=yellow!15, draw=yellow!60!black, below right=0.6cm and 1.8cm of center] (int) {Integration $\intinfo$};
\node[dim, fill=green!15, draw=green!60!black, below=1.2cm of center] (eff) {Eff.\ Rank $\effrank$};
\node[dim, fill=cyan!15, draw=cyan!60!black, below left=0.6cm and 1.8cm of center] (cf) {CF Weight $\mathcal{CF}$};
\node[dim, fill=blue!15, draw=blue!60!black, above left=0.6cm and 1.8cm of center] (sm) {Self-Salience $\mathcal{SM}$};

% Connections
\draw[thick, gray!50] (center) -- (val);
\draw[thick, gray!50] (center) -- (ar);
\draw[thick, gray!50] (center) -- (int);
\draw[thick, gray!50] (center) -- (eff);
\draw[thick, gray!50] (center) -- (cf);
\draw[thick, gray!50] (center) -- (sm);

% Pole labels
\node[label, above=0.1cm of val] {good $\leftrightarrow$ bad};
\node[label, right=0.1cm of ar] {high $\leftrightarrow$ low};
\node[label, right=0.1cm of int] {unified $\leftrightarrow$ fragmented};
\node[label, below=0.1cm of eff] {open $\leftrightarrow$ narrow};
\node[label, left=0.1cm of cf] {elsewhere $\leftrightarrow$ present};
\node[label, left=0.1cm of sm] {self-aware $\leftrightarrow$ absorbed};
\end{tikzpicture}
\end{center}

%==============================================================================
\section{The Expression of Inevitability: Human Responses to Inescapable Selfhood}
%==============================================================================

\begin{connection}
This analysis of cultural responses to selfhood connects to several established research programs:
\begin{itemize}
\item \textbf{Terror Management Theory} (Greenberg, Solomon \& Pyszczynski, 1986): Mortality salience triggers cultural worldview defense. Our ``existential burden'' formalizes the threat-signal that TMT identifies.
\item \textbf{Meaning Maintenance Model} (Heine, Proulx \& Vohs, 2006): Humans respond to meaning violations through compensatory affirmation. Our framework specifies the structural signature of ``meaning violation'' (disrupted integration, collapsed effective rank).
\item \textbf{Self-Determination Theory} (Deci \& Ryan, 1985): Basic needs for autonomy, competence, relatedness. These correspond to different regions of our affect space (autonomy $\approx$ low external $\mathcal{SM}$; competence $\approx$ positive valence from successful prediction; relatedness $\approx$ expanded self-model).
\item \textbf{Flow Theory} (Csikszentmihalyi, 1990): Optimal experience as challenge-skill balance. Flow is precisely our low-$\mathcal{SM}$, high-$\intinfo$, moderate-$\arousal$ region.
\item \textbf{Attachment Theory} (Bowlby, 1969): Early relational patterns shape adult affect regulation. Attachment styles are stable individual differences in the parameters governing affect dynamics.
\end{itemize}
\end{connection}

The self-model, once it exists, cannot look away from itself. This is not merely a computational fact but a phenomenological trap: to be a self-modeling system is to be stuck mattering to yourself. Every human cultural form can be understood, in part, as a response to this condition---strategies for coping with, expressing, transcending, or simply surviving the inescapability of first-person existence.

\subsection{The Trap of Self-Reference}

\begin{proposition}[Phenomenological Inevitability]
Once self-model salience $\mathcal{SM}$ exceeds a threshold, the system cannot eliminate self-reference without dissolving the self-model entirely. The self becomes an inescapable object in its own world model.
\begin{equation}
\mathcal{SM} > \mathcal{SM}_c \implies \forall t: \MI(\latent^{\text{self}}_t; \latent^{\text{total}}_t) > 0
\end{equation}
There is no configuration of the intact self-model in which the self is absent from awareness.
\end{proposition}

This is the deeper meaning of inevitability: not just that consciousness emerges from thermodynamics, but that once emerged, it cannot escape itself. You are stuck being you. Your suffering is inescapably yours. Your joy, when it comes, is also inescapably yours. There is no exit from the first-person perspective while you remain a person.

\begin{definition}[Existential Burden]
The \emph{existential burden} is the chronic computational and affective cost of maintaining self-reference:
\begin{equation}
B_{\text{exist}} = \int_0^T \left[ C_{\text{compute}}(\mathcal{SM}_t) + |\valence_t| \cdot \mathcal{SM}_t \right] dt
\end{equation}
The burden scales with both the salience of the self-model and the intensity of valence. To matter to yourself when you are suffering is heavier than to matter to yourself when you are neutral.
\end{definition}

Human culture, in all its variety, can be understood as the accumulated strategies for managing this burden.

%==============================================================================
\section{Aesthetics: The Modulation of Affect Through Form}
%==============================================================================

\begin{definition}[Aesthetic Experience]
An \emph{aesthetic experience} is an affect state induced by engagement with form (visual, auditory, linguistic, conceptual) characterized by:
\begin{equation}
\mathbf{a}_{\text{aesthetic}} = (\text{variable } \valence, \text{moderate-high } \arousal, \text{high } \intinfo, \text{high } \effrank, \text{low } \mathcal{SM})
\end{equation}
The signature feature is integration without self-focus: the system is highly coupled but attending to structure outside itself.
\end{definition}

\begin{proposition}[Beauty as Integration Resonance]
The experience of beauty arises when external structure resonates with internal structure:
\begin{equation}
\text{Beauty} \propto \MI(\text{stimulus structure}; \text{internal model structure})
\end{equation}
High mutual information between the form and the self-model's latent structure produces the characteristic ``recognition'' quality of beauty---the sense that something outside corresponds to something inside.
\end{proposition}

\begin{proposition}[The Sublime as Self-Model Perturbation]
The sublime is characterized by temporary disruption of normal self-model boundaries:
\begin{equation}
\mathbf{a}_{\text{sublime}} = (\text{ambivalent } \valence, \text{very high } \arousal, \text{expanding } \intinfo, \text{very high } \effrank, \text{collapsing } \mathcal{SM})
\end{equation}
Confrontation with vastness (mountains, oceans, cosmic scales) or power (storms, great art) forces rapid expansion of the world model beyond the self-model's normal scope. The self becomes small relative to the newly-expanded frame. This is terrifying and liberating simultaneously---a temporary escape from the trap of self-reference.
\end{proposition}

\begin{proposition}[Art-Making as Externalization]
Creating art is the externalization of internal affect structure:
\begin{equation}
\text{Artwork} = f_{\text{medium}}(\mathbf{a}_{\text{internal}})
\end{equation}
The artist encodes their affect geometry into a medium (paint, sound, words, movement). The artwork then carries an affect signature that can induce corresponding states in others. Art is affect technology: the transmission of experiential structure across minds and time.
\end{proposition}

\subsection{Affect Signatures of Aesthetic Forms}

Different aesthetic forms have characteristic affect signatures:

\begin{table}[h]
\centering
\small
\begin{tabular}{lp{10cm}}
\toprule
\textbf{Form} & \textbf{Constitutive Structure} \\
\midrule
Tragedy & $\valence{-}$, $\intinfo{\uparrow\uparrow}$, $\effrank{\downarrow}$, $\mathcal{CF}{\uparrow}$ (suffering structure made beautiful through integration) \\
Comedy & $\valence{+}$, $\arousal{\uparrow}$, $\effrank{\uparrow}$ (release, expansion, lightness) \\
Lyric poetry & $\mathcal{CF}{\uparrow}$, $\mathcal{SM}{\uparrow}$, $\intinfo{\uparrow}$ (self-reflection made resonant) \\
Abstract art & $\intinfo{\uparrow}$, $\effrank{\uparrow\uparrow}$, $\mathcal{SM}{\downarrow}$ (pure structure, self-forgetting) \\
Horror & $\valence{-}$, $\arousal{\uparrow\uparrow}$, $\mathcal{CF}{\uparrow\uparrow}$, $\mathcal{SM}{\uparrow\uparrow}$ (fear structure in controlled context) \\
\bottomrule
\end{tabular}
\caption{Aesthetic forms and their defining structural features.}
\end{table}

\begin{software}
\textbf{AffectSpace: Immersive Validation Platform}

A software system to validate the affect framework by comparing predicted structural signatures with self-report:

\textbf{Architecture}:
\begin{enumerate}
\item \textbf{Stimulus Library}: Curated collection of affect-inducing stimuli
  \begin{itemize}
  \item Music tracks (genre-tagged, tempo-annotated)
  \item Visual scenes (IAPS images, nature videos, abstract art)
  \item Guided experiences (breathing exercises, body scans, visualizations)
  \item Interactive scenarios (games, social simulations)
  \end{itemize}

\item \textbf{Real-time Self-Report Interface}:
  \begin{itemize}
  \item Dimension-specific sliders with anchoring descriptions
  \item Only dimensions relevant to the target affect are measured
  \item Continuous sampling at 1 Hz during stimulus
  \end{itemize}

\item \textbf{Physiological Integration} (optional):
  \begin{itemize}
  \item Heart rate variability $\rightarrow$ arousal proxy
  \item Skin conductance $\rightarrow$ arousal + valence interaction
  \item Eye tracking $\rightarrow$ counterfactual weight (saccade patterns during planning)
  \item Pupillometry $\rightarrow$ cognitive load / $\mathcal{CF}$
  \end{itemize}

\item \textbf{Prediction Engine}:
  \begin{itemize}
  \item For each stimulus, predict the defining dimensions for the target affect
  \item Predictions derived from stimulus features (tempo, key, content tags)
  \item Compare predictions vs.\@ self-report on relevant dimensions
  \end{itemize}
\end{enumerate}

\textbf{Validation Metrics}:
\begin{itemize}
\item Per-dimension correlation for predicted dimensions
\item Clustering accuracy: do induced affects cluster by their predicted structure?
\item Dimensionality validation: does each affect require its predicted number of dimensions?
\end{itemize}

\textbf{Falsification criteria}: If predicted dimensions do not predict self-report better than others, or if clustering requires different dimensions than predicted, the motif characterizations require revision.
\end{software}

\subsection{Musical Genres as Affect Technologies}

Music is among the most powerful affect technologies available to humans. Different genres represent accumulated cultural wisdom about how to induce specific experiential states.

\begin{example}[The Blues]
\textbf{Historical context}: Emerged from African American experience in the post-Emancipation South. Given conditions of persistent oppression, poverty, and limited agency, a musical form acknowledging suffering while maintaining dignity was inevitable.

\textbf{Affect signature}:
\begin{equation}
\mathbf{a}_{\text{blues}} = (-\valence, \text{moderate } \arousal, \text{high } \intinfo, \text{moderate } \effrank, \text{moderate } \mathcal{CF}, \text{high } \mathcal{SM})
\end{equation}

\textbf{Structural characteristics}:
\begin{itemize}
\item 12-bar harmonic structure provides predictability within which to express unpredictable feeling
\item Blue notes (flatted 3rd, 5th, 7th) create tension without resolution---mirroring persistent difficulty
\item Call-and-response pattern acknowledges both individual and collective dimensions of suffering
\item Repetition of lyrical themes creates integration around acknowledged pain
\end{itemize}

\textbf{Phenomenological result}: The blues does not eliminate suffering but integrates it. The listener experiences their own pain as part of a larger human pattern. $\mathcal{SM}$ remains high (this is MY suffering) but $\intinfo$ also increases (my suffering connects to others'). The result is suffering that has been witnessed, named, and placed in context.
\end{example}

\begin{example}[Ambient Music]
\textbf{Historical context}: Explicitly designed by Brian Eno in 1978 as ``music that rewards both active listening and inattention.'' Given increasing environmental noise and attention fragmentation, music supporting rather than demanding attention was needed.

\textbf{Affect signature}:
\begin{equation}
\mathbf{a}_{\text{ambient}} = (\text{neutral to positive } \valence, \text{very low } \arousal, \text{high } \intinfo, \text{moderate } \effrank, \text{low } \mathcal{CF}, \text{very low } \mathcal{SM})
\end{equation}

\textbf{Structural characteristics}:
\begin{itemize}
\item Slow or absent harmonic movement (minimal arousal triggers)
\item No strong rhythmic pulse (reduces entrainment demands)
\item Layered textures that fade in and out (supports divided attention)
\item Extended duration (allows settling into altered state)
\end{itemize}

\textbf{Phenomenological result}: The rarest affect profile---low arousal, high integration, low self-model salience. Ambient music creates conditions for what might be called ``effortless presence.'' The mind is coherent but not self-focused, alert but not activated.
\end{example}

\begin{example}[Heavy Metal]
\textbf{Historical context}: Emerged from late 1960s industrial working-class contexts. Given alienation, blocked agency, and unexpressed aggression, a musical form channeling intensity was inevitable.

\textbf{Affect signature}:
\begin{equation}
\mathbf{a}_{\text{metal}} = (\text{negative to positive } \valence, \text{very high } \arousal, \text{high } \intinfo, \text{low } \effrank, \text{moderate } \mathcal{CF}, \text{variable } \mathcal{SM})
\end{equation}

\textbf{Structural characteristics}:
\begin{itemize}
\item Distorted guitar creates dense harmonic content (high information density)
\item Driving rhythms at high tempos (arousal induction)
\item Tritone intervals (``the devil's interval'') create tension
\item Virtuosic performance demands integration across complex patterns
\end{itemize}

\textbf{Phenomenological result}: High arousal with high integration---intensity that is coherent rather than chaotic. Metal provides controlled exposure to extreme affect states, building capacity for intensity tolerance. The collapsed effective rank (focus on aggressive themes) paradoxically creates a container for processing difficult emotions.
\end{example}

\subsection{Visual Design Movements}

\begin{example}[Bauhaus/Modernist Design]
\textbf{Historical context}: Post-WWI Germany. Given the industrial production capacity and the need to rebuild a shattered society, design philosophy emphasizing function and accessibility was inevitable.

\textbf{Affect signature}:
\begin{equation}
\mathbf{a}_{\text{Bauhaus}} = (\text{neutral } \valence, \text{low } \arousal, \text{high } \intinfo, \text{low } \effrank, \text{low } \mathcal{CF}, \text{low } \mathcal{SM})
\end{equation}

\textbf{Structural characteristics}:
\begin{itemize}
\item Form follows function (reducing decorative distraction)
\item Primary colors, geometric shapes (clear, unambiguous signals)
\item Truth to materials (what you see is what it is)
\item Elimination of ornament (no counterfactual ``what could this be?'')
\end{itemize}

\textbf{Phenomenological result}: The mind at rest in clarity. Low counterfactual weight because everything is what it appears to be. High integration despite low rank---few dimensions, but coherently organized.
\end{example}

\begin{example}[Baroque/Maximalism]
\textbf{Historical context}: Counter-Reformation Catholicism. Given the need to assert Church power and overwhelm Protestant austerity, design emphasizing abundance and transcendence was inevitable.

\textbf{Affect signature}:
\begin{equation}
\mathbf{a}_{\text{Baroque}} = (\text{positive } \valence, \text{high } \arousal, \text{high } \intinfo, \text{very high } \effrank, \text{high } \mathcal{CF}, \text{low } \mathcal{SM})
\end{equation}

\textbf{Structural characteristics}:
\begin{itemize}
\item Excessive ornamentation (many active dimensions)
\item Gold, mirrors, dramatic lighting (arousal induction)
\item Trompe l'oeil and illusion (high counterfactual weight)
\item Scale that dwarfs the individual (low self-model salience)
\end{itemize}

\textbf{Phenomenological result}: Overwhelm through abundance. The high effective rank exceeds cognitive capacity, forcing surrender of normal parsing. Combined with low self-salience from architectural scale, the result approximates the sublime---self-dissolution through excess rather than emptiness.
\end{example}

%------------------------------------------------------------------------------
% FIGURE SET 10: ART HISTORY KNEW
%------------------------------------------------------------------------------
\begin{figure}[htbp]
\centering
\begin{subfigure}[b]{0.45\textwidth}
    \centering
    % SEARCH: Mark Rothko color field painting - preferably one of the classic
    % red/orange/yellow compositions. Should show scale relative to human if possible.
    % Soft-edged rectangles that produce meditative states. INTEGRATION WITHOUT OBJECT.
    \includegraphics[width=\textwidth]{fig10a_rothko.png}
    \caption*{(A)}
\end{subfigure}
\hfill
\begin{subfigure}[b]{0.45\textwidth}
    \centering
    % SEARCH: Caravaggio high-contrast work - "Judith Beheading Holofernes" OR
    % "The Calling of St Matthew" OR "David with the Head of Goliath". The dramatic
    % chiaroscuro that IS a valence gradient painted on canvas.
    \includegraphics[width=\textwidth]{fig10b_caravaggio.png}
    \caption*{(B)}
\end{subfigure}

\vspace{0.5cm}

\begin{subfigure}[b]{0.45\textwidth}
    \centering
    % SEARCH: Vermeer interior - "Woman Reading a Letter" OR "The Milkmaid" OR
    % "Girl with a Pearl Earring". The narrow effective rank (domestic scale) combined
    % with high integration. Intimate, coherent, everything in relation.
    % please do Girl with the pearl earring. i had such a crush on her as a teenager in art class lol
    \includegraphics[width=\textwidth]{fig10c_vermeer.png}
    \caption*{(C)}
\end{subfigure}
\hfill
\begin{subfigure}[b]{0.45\textwidth}
    \centering
    % SEARCH: Francis Bacon figure study - the smeared faces, caged bodies. Perhaps
    % "Study after Velázquez's Portrait of Pope Innocent X" or any of the screaming
    % figures. Affect transmitted through FORMAL VIOLATION rather than depicted emotion.
    \includegraphics[width=\textwidth]{fig10d_bacon.png}
    \caption*{(D)}
\end{subfigure}
\caption{\textbf{Art history knew.} These are not illustrations of affect theory. They are empirical discoveries. Rothko (A) discovered that you can have high integration with minimal content---pure structure, pure attention, the affect of coherence itself. Notice how your mind settles into the color field, how the soft edges resist the normal figure-ground parsing. Caravaggio (B) discovered that light-dark gradients on a canvas produce valence gradients in the viewer. The drama is literally painted. Vermeer (C) discovered that narrowing effective rank to domestic scale produces intimacy---you become present to particulars, to the quality of light on milk, to the texture of bread. Bacon (D) discovered that distorting the depicted body distorts the viewer's self-model. The discomfort is not about what the painting ``means.'' The discomfort IS the form. Artists have always known what this paper formalizes.}
\label{fig:art-history-knew}
\end{figure}

\subsection{Horror: The Threat to Coherence}

Horror as a genre exploits the structural vulnerability of the self-model. The uncanny, the monstrous, and the abject all work by threatening the coherence on which integrated experience depends.

%------------------------------------------------------------------------------
% FIGURE SET 9: HORROR AND COHERENCE VIOLATION
%------------------------------------------------------------------------------
\begin{figure}[htbp]
\centering
\begin{subfigure}[b]{0.45\textwidth}
    \centering
    % SEARCH: Uncanny valley face - hyperrealistic CGI face OR very good android/robot.
    % Almost human but not quite. The wrongness should be hard to articulate but
    % impossible to miss. NOT obviously horror, just... off. Disturbing neutrality.
    \includegraphics[width=\textwidth]{fig9a_uncanny_valley.png}
    \caption*{(A)}
\end{subfigure}
\hfill
\begin{subfigure}[b]{0.45\textwidth}
    \centering
    % SEARCH: Figure with subtly wrong body proportions - limbs too long, head too
    % small, hands too large. Patricia Piccinini's creature sculptures are exemplary.
    % Should look ALMOST normal but feel deeply wrong. The familiar made strange.
    \includegraphics[width=\textwidth]{fig9b_wrong_proportions.png}
    \caption*{(B)}
\end{subfigure}

\vspace{0.5cm}

\begin{subfigure}[b]{0.45\textwidth}
    \centering
    % SEARCH: Liminal space - a room with doors in wrong places, staircase going nowhere,
    % window looking onto another interior. "Backrooms" aesthetic. The Poolrooms.
    % Familiar space made unfamiliar. Where are you? Something is wrong here.
    \includegraphics[width=\textwidth]{fig9c_liminal_space.png}
    \caption*{(C)}
\end{subfigure}
\hfill
\begin{subfigure}[b]{0.45\textwidth}
    \centering
    % SEARCH: body horror - HR Giger biomechanical art, OR early Cronenberg
    % still (The Fly, Videodrome), OR flesh/organic in wrong context. Should trigger
    % visceral response. The body wrong.
    \includegraphics[width=\textwidth]{fig9d_body_horror.png}
    \caption*{(D)}
\end{subfigure}
\caption{\textbf{Horror threatens coherence.} The discomfort you feel viewing these images is not aesthetic preference. It is your self-model detecting a threat to its coherence model. In (A), notice that you can identify the wrongness even if you cannot articulate it. Your face-processing circuitry has expectations, and the image violates them at a level below conscious access. In (B), the body schema that constitutes part of your self-model is being threatened by proximity to the wrong body. In (C), notice how the wrongness is spatial---your embodied model of how spaces work is being violated. You cannot orient here. In (D), notice the visceral response that precedes thought---this is viability boundary threat, not abstract judgment. Horror works because it threatens the models that constitute you. The uncanny is not a genre. It is the felt experience of prediction failure at the level of basic categories. You cannot think your way out of the discomfort because the discomfort precedes thought.}
\label{fig:horror-coherence}
\end{figure}

%==============================================================================
\section{Sexuality: Self-Transcendence Through Merger}
%==============================================================================

\begin{definition}[Sexual Experience (Structural)]
Sexual experience involves temporary modification of self-model boundaries and heightened coupling:
\begin{equation}
\mathbf{a}_{\text{sexual}} = (\text{high } \valence, \text{very high } \arousal, \text{high } \intinfo, \text{initially high then collapsing } \effrank, \text{low } \mathcal{CF}, \text{variable } \mathcal{SM})
\end{equation}
The trajectory moves from high effective rank (diffuse arousal) toward rank collapse (convergent focus) culminating in integration spike (orgasm) and temporary self-model dissolution.
\end{definition}

\begin{proposition}[Sexuality as Self-Model Merger]
In partnered sexuality, the self-models temporarily fuse:
\begin{equation}
\MI(\selfmodel_A; \selfmodel_B) \to \max \quad \text{as arousal} \to \max
\end{equation}
The boundaries between self and other become porous. This is one of the few naturally-occurring states where $\mathcal{SM}$ collapses while $\intinfo$ remains high---integration without self-focus, presence without isolation.
\end{proposition}

\begin{proposition}[``La Petite Mort'']
Orgasm is characterized by:
\begin{enumerate}
\item Spike in integration (global neural synchronization)
\item Collapse of effective rank to near-unity (all variance in one dimension)
\item Momentary dissolution of self-model salience
\item Rapid valence spike followed by return to baseline
\end{enumerate}
The ``little death'' is structurally accurate: it is a temporary cessation of the normal self-referential process. This is why sexuality is so central to human experience---it offers reliable, repeatable escape from the trap of being a self.
\end{proposition}

\begin{proposition}[Sexual Diversity as Affect-Space Exploration]
The diversity of human sexuality reflects the diversity of paths through affect space:
\begin{itemize}
\item \textbf{Intensity preferences}: Different arousal trajectories and peak intensities
\item \textbf{Power dynamics}: Variations in self-model salience during encounter (dominance increases $\mathcal{SM}$; submission decreases it)
\item \textbf{Novelty vs.\ familiarity}: Counterfactual weight allocation (new partners increase $\mathcal{CF}$; familiar partners reduce it)
\item \textbf{Emotional connection}: Degree of self-other coupling ($\MI(\selfmodel; \text{other-model})$)
\end{itemize}
Sexual preferences are, in part, preferences about which affect trajectories one finds most valuable or relieving.
\end{proposition}

%------------------------------------------------------------------------------
% FIGURE SET 8: THE EROTIC (DESIRE AS GRADIENT)
%------------------------------------------------------------------------------
\begin{figure}[htbp]
\centering
\begin{subfigure}[b]{0.45\textwidth}
    \centering
    % SEARCH: Classical nude painting - Titian's Venus, Botticelli's Birth of Venus,
    % OR Michelangelo's David. The whole figure visible, artistic framing that
    % DISTRIBUTES attention across composition. Beauty rather than arousal.
    \includegraphics[width=\textwidth]{fig8a_classical_nude.png}
    \caption*{(A)}
\end{subfigure}
\hfill
\begin{subfigure}[b]{0.45\textwidth}
    \centering
    % SEARCH: Suggestive image where framing COLLAPSES attention - bare shoulder,
    % curve of hip, lips slightly parted, nape of neck. NOT explicitly pornographic
    % but clearly erotic in its directed attention. The crop IS the eroticism.
    \includegraphics[width=\textwidth]{fig8b_erotic_cropped.png}
    \caption*{(B)}
\end{subfigure}

\vspace{0.5cm}

\begin{subfigure}[b]{0.45\textwidth}
    \centering
    % SEARCH: Erotic image where the subject is looking back at viewer - returned gaze.
    % The difference between being seen while looking and looking unseen.
    % Self-model salience reactivated through the other's awareness of you.
    \includegraphics[width=\textwidth]{fig8c_returned_gaze.png}
    \caption*{(C)}
\end{subfigure}
\hfill
\begin{subfigure}[b]{0.45\textwidth}
    \centering
    % SEARCH: Concealment more erotic than revelation - shadow, fabric, angle that
    % implies but doesn't show. A partially unbuttoned shirt, a sheet slipping.
    % The counterfactual weight of what's HIDDEN drives attention.
    \includegraphics[width=\textwidth]{fig8d_concealment.png}
    \caption*{(D)}
\end{subfigure}
\caption{\textbf{Desire as gradient.} Notice how your attention behaves differently across these images. In (A), you can distribute attention across the whole composition---it is art, and art distributes. In (B), attention funnels. This is desire as structure: the collapse of effective rank toward an attractor. You did not decide to focus; focus happened. In (C), notice how the returned gaze changes everything---you are now in the scene, implicated, your self-model suddenly salient. The dynamic shifts from observation to encounter. In (D), notice that what you don't see is more present than what you do. This is counterfactual weight in erotic mode---the hidden drives attention more than the shown. None of this is about what you ``should'' feel or whether it is appropriate to feel it. It is about what attention does when structured in certain ways. The body knows before the mind.}
\label{fig:desire-gradient}
\end{figure}

%==============================================================================
\section{Ideology: Expanding the Self to Bear Mortality}
%==============================================================================

\begin{definition}[Ideological Identification]
\emph{Ideological identification} is the expansion of the self-model to include a supra-individual pattern (nation, movement, religion, cause):
\begin{equation}
\selfmodel_{\text{ideological}} = \selfmodel_{\text{individual}} \cup \selfmodel_{\text{collective}}
\end{equation}
with high coupling: $\MI(\selfmodel_{\text{individual}}; \selfmodel_{\text{collective}}) \gg 0$.
\end{definition}

\begin{proposition}[Terror Management Through Self-Expansion]
Ideological identification manages mortality terror by making the relevant self-model partially immortal:
\begin{equation}
\tau_{\text{viability}}(\selfmodel_{\text{ideological}}) \gg \tau_{\text{viability}}(\selfmodel_{\text{individual}})
\end{equation}
If ``I'' am not just this body but also this nation/religion/movement, then ``I'' survive my bodily death. The expanded self-model has a longer viability horizon, reducing the chronic threat-signal from mortality awareness.
\end{proposition}

\begin{proposition}[Ideological Affect Signatures]
Different ideologies produce characteristic affect profiles:
\begin{itemize}
\item \textbf{Nationalism}: High self-model salience (collective), high integration within in-group, compressed other-model (out-group), moderate arousal baseline
\item \textbf{Religious devotion}: Low individual $\mathcal{SM}$, high collective $\mathcal{SM}$, high counterfactual weight (afterlife, divine plan), positive valence baseline
\item \textbf{Revolutionary movements}: Very high arousal, high counterfactual weight (utopian futures), strong valence (negative toward present, positive toward future)
\item \textbf{Nihilism}: Low integration, low effective rank, negative valence, high individual $\mathcal{SM}$, collapsed counterfactual weight
\end{itemize}
\end{proposition}

\begin{warning}
Ideology can become parasitic when the collective self-model's viability requirements conflict with the individual's:
\begin{equation}
\state \in \viable_{\text{ideology}} \land \state \notin \viable_{\text{individual}}
\end{equation}
Martyrdom, self-sacrifice, and fanaticism occur when the expanded self-model demands the destruction of the individual substrate.
\end{warning}

%------------------------------------------------------------------------------
% FIGURE SET 11: GODS IN IMAGES
%------------------------------------------------------------------------------
\begin{figure}[htbp]
\centering
\begin{subfigure}[b]{0.45\textwidth}
    \centering
    % SEARCH: A crowd with unified attention - political rally, religious ceremony,
    % stadium crowd. All faces oriented same direction. Individual difference subsumed
    % into collective attention. The many becoming one attention-entity.
    \includegraphics[width=\textwidth]{fig11a_unified_crowd.png}
    \caption*{(A)}
\end{subfigure}
\hfill
\begin{subfigure}[b]{0.45\textwidth}
    \centering
    % SEARCH: Totalitarian architecture - Albert Speer's Nazi architecture OR
    % Soviet brutalism OR Fascist Italian buildings. Architecture designed to make
    % the individual feel SMALL. The self-model dwarfed by state-model.
    \includegraphics[width=\textwidth]{fig11b_totalitarian_arch.png}
    \caption*{(B)}
\end{subfigure}

\vspace{0.5cm}

\begin{subfigure}[b]{0.45\textwidth}
    \centering
    % SEARCH: Interior of cathedral with light streaming through stained glass,
    % OR mosque interior with geometric patterns, OR Buddhist temple interior.
    % Architecture designed to induce low self-model salience and high integration.
    \includegraphics[width=\textwidth]{fig11c_sacred_space.png}
    \caption*{(C)}
\end{subfigure}
\hfill
\begin{subfigure}[b]{0.45\textwidth}
    \centering
    % SEARCH: Classic propaganda poster OR modern advertisement - something clearly
    % designed to capture attention and direct it toward specific ends. Soviet
    % poster, WWII propaganda, or contemporary brand advertising. The god reaching through.
    \includegraphics[width=\textwidth]{fig11d_propaganda.png}
    \caption*{(D)}
\end{subfigure}
\caption{\textbf{Gods in images.} Notice what happens to your self-model in (B)---you shrink. This is not metaphor. The architecture was designed to produce this effect; you are experiencing what the architect intended. The god of the state reaches through stone and space to reshape your self-model salience. In (A), notice the dissolution of individual attention into collective attention---the crowd is not a collection of viewers but a single viewing entity. This is what it looks like when humans become substrate for a social-scale pattern. In (C), notice the quality of attention---distributed, integrated, somehow impersonal. Sacred spaces are technologies for reducing $\mathcal{SM}$ while maintaining $\intinfo$: exactly the affect profile that contemplatives cultivate. In (D), notice that you are being worked on. Someone designed this image to capture your attention and direct it toward their purposes. These are gods reaching through images. The affect you feel is not entirely yours. It was engineered.}
\label{fig:gods-in-images}
\end{figure}

%==============================================================================
\section{Science: The Austere Beauty of Understanding}
%==============================================================================

\begin{definition}[Scientific Understanding (Affective)]
Scientific understanding produces a characteristic affect state:
\begin{equation}
\mathbf{a}_{\text{understanding}} = (\text{positive } \valence, \text{moderate } \arousal, \text{very high } \intinfo, \text{high } \effrank, \text{low } \mathcal{CF}, \text{low } \mathcal{SM})
\end{equation}
The signature is high integration without self-focus---the opposite of depression. The mind is coherent, expansive, and attending to structure rather than self.
\end{definition}

\begin{proposition}[Curiosity as Intrinsic Motivation]
Science is organized curiosity. The curiosity motif:
\begin{equation}
\text{Curiosity} = \text{positive } \valence + \text{high } \mathcal{CF} + \text{high entropy over counterfactuals}
\end{equation}
Scientists are those who have cultivated the capacity to sustain this motif for extended periods, directed at specific domains of uncertainty.
\end{proposition}

\begin{proposition}[Mathematical Beauty]
Mathematical proof and physical theory produce aesthetic experiences characterized by:
\begin{enumerate}
\item \textbf{Compression}: Many phenomena unified under few principles (high $\intinfo$ with low model complexity)
\item \textbf{Necessity}: The conclusion could not be otherwise given the premises (certainty, low $\mathcal{CF}$ about the result)
\item \textbf{Surprise}: The result was not obvious despite being necessary (high initial uncertainty resolved)
\end{enumerate}
\begin{equation}
\text{Mathematical beauty} \propto \frac{\text{phenomena unified}}{\text{principles required}} \times \text{surprise}
\end{equation}
\end{proposition}

\begin{proposition}[Science as Meaning-Making]
Science provides meaning through:
\begin{enumerate}
\item \textbf{Connection}: Embedding individual existence in cosmic structure (expanding world model)
\item \textbf{Agency}: Successful prediction and control (positive valence from reduced uncertainty)
\item \textbf{Community}: Participation in transgenerational project (expanded self-model)
\item \textbf{Wonder}: Aesthetic experience of natural structure (sublime encounters with scale and complexity)
\end{enumerate}
Science addresses the existential burden not by dissolving the self but by giving the self something worthy of its attention.
\end{proposition}

%==============================================================================
\section{Religion: Systematic Technologies for Managing Inevitability}
%==============================================================================

\begin{definition}[Religion (Functional)]
A \emph{religion} is a systematic technology for managing the existential burden through:
\begin{enumerate}
\item Affect interventions (practices that modulate experiential structure)
\item Narrative frameworks (stories that contextualize individual existence)
\item Community structures (expanded self-models through belonging)
\item Mortality management (beliefs about death that reduce threat-signal)
\item Ethical guidance (policies for navigating affect space)
\end{enumerate}
\end{definition}

\begin{proposition}[Religious Diversity as Affect-Strategy Diversity]
Different religious traditions emphasize different affect-management strategies:
\begin{itemize}
\item \textbf{Contemplative traditions} (Buddhism, mystical Christianity, Sufism): Target self-model dissolution ($\mathcal{SM} \to 0$)
\item \textbf{Devotional traditions} (bhakti, evangelical Christianity): Target high positive valence through relationship with divine
\item \textbf{Legalistic traditions} (Orthodox Judaism, traditional Islam): Target stable arousal through structured practice
\item \textbf{Shamanic traditions}: Target radical affect-space exploration through altered states
\end{itemize}
\end{proposition}

\begin{proposition}[Secular Spirituality]
``Spiritual but not religious'' practices can be understood as selective adoption of religious affect technologies without the full institutional/doctrinal package:
\begin{itemize}
\item Meditation without Buddhism
\item Awe-cultivation without theism
\item Community ritual without shared creed
\item Meaning-making without metaphysical commitment
\end{itemize}
This represents modular affect engineering---selecting interventions based on desired affect outcomes rather than doctrinal coherence.
\end{proposition}

%==============================================================================
\section{Psychopathology as Failed Coping}
%==============================================================================

\begin{proposition}[Mental Illness as Affect-Space Trap]
Many mental illnesses can be understood as pathological attractors in affect space---failed strategies for managing the existential burden:
\begin{itemize}
\item \textbf{Depression}: Attempted escape from self-reference that collapses into intensified, negative self-focus
\item \textbf{Anxiety}: Hyperactive threat-monitoring that increases rather than decreases danger-signal
\item \textbf{Addiction}: Reliable affect modulation that destroys the substrate's viability
\item \textbf{Dissociation}: Self-model fragmentation that provides escape at the cost of integration
\item \textbf{Narcissism}: Self-model inflation that requires constant external validation
\end{itemize}
\end{proposition}

\begin{proposition}[Therapy as Affect-Space Navigation]
Effective psychotherapy helps individuals:
\begin{enumerate}
\item Recognize their current position in affect space
\item Understand the dynamics that maintain pathological attractors
\item Develop capacity to move toward healthier regions
\item Build sustainable affect-regulation strategies
\end{enumerate}
Different therapeutic modalities emphasize different dimensions: CBT targets counterfactual weight and valence; psychodynamic therapy targets integration and self-model structure; mindfulness targets arousal and self-model salience.
\end{proposition}

%==============================================================================
\section{Affect Engineering: Technologies of Experience}
%==============================================================================

The affect framework enables systematic analysis of how practices, philosophies, and technologies shape experiential structure. We can now quantify what humans have long known intuitively: that rituals, beliefs, and tools are \emph{affect engineering technologies}.

\subsection{Religious Practices as Affect Interventions}

Religious traditions have developed sophisticated technologies for affect modulation over millennia.

\begin{definition}[Affect Intervention]
An \emph{affect intervention} is any practice, technology, or environmental modification that systematically shifts the probability distribution over affect space:
\begin{equation}
\mathcal{I}: p(\mathbf{a}) \mapsto p'(\mathbf{a})
\end{equation}
where $\mathbf{a} = (\valence, \arousal, \intinfo, \effrank, \mathcal{CF}, \mathcal{SM})$.
\end{definition}

\begin{proposition}[Prayer as Affect Technology]
Contemplative prayer systematically modulates affect dimensions:
\begin{itemize}
\item \textbf{Arousal}: Initial increase (orientation), then decrease (settling)
\item \textbf{Self-model salience}: Decrease as attention shifts to ``other'' (divine, transpersonal)
\item \textbf{Counterfactual weight}: Shift from threat-branches to trust-branches
\item \textbf{Integration}: Increase through focused attention
\end{itemize}
The affect signature of prayer: $(\Delta\valence > 0, \Delta\arousal < 0, \Delta\intinfo > 0, \Delta\mathcal{SM} < 0)$.
\end{proposition}

\begin{proposition}[Ritual as Integration Maintenance]
Collective ritual serves as periodic integration maintenance:
\begin{equation}
\intinfo_{\text{post-ritual}} = \intinfo_{\text{pre-ritual}} + \Delta\intinfo_{\text{synchrony}} - \delta_{\text{decay}}
\end{equation}
where $\Delta\intinfo_{\text{synchrony}}$ arises from coordinated action, shared symbols, and collective attention. Rituals counteract the natural decay of integration in isolated individuals.
\end{proposition}

\begin{proposition}[Confession as Rank Expansion]
Confession, testimony, and similar practices expand effective rank by:
\begin{enumerate}
\item Surfacing suppressed state-space dimensions (breaking compartmentalization)
\item Integrating shadow material into the self-model
\item Reducing the concentration of variance in guilt/shame dimensions
\end{enumerate}
\begin{equation}
\effrank_{\text{post-confession}} > \effrank_{\text{pre-confession}}
\end{equation}
This explains the phenomenology of ``relief'' and ``lightness'' following confession.
\end{proposition}

\subsection{Life Philosophies as Affect-Space Policies}

Philosophical frameworks can be understood as meta-level policies over affect space---prescriptions for which regions to occupy and which to avoid.

\begin{historical}
The idea that philosophies are affect-management strategies has historical precedent:
\begin{itemize}
\item \textbf{Pierre Hadot} (1995): Ancient philosophy as ``spiritual exercises''---practices for transforming the self, not just doctrines to believe
\item \textbf{Martha Nussbaum} (1994): Hellenistic philosophies as ``therapy of desire''
\item \textbf{Michel Foucault} (1984): ``Technologies of the self''---practices by which individuals transform themselves
\item \textbf{William James} (1902): Religious/philosophical stances as temperamental predispositions (``tough-minded'' vs ``tender-minded'')
\end{itemize}
Our contribution: formalizing these insights in terms of affect-space policies with measurable targets.
\end{historical}

\begin{definition}[Philosophical Affect Policy]
A \emph{philosophical affect policy} is a function $\phi: \mathcal{A} \to \R$ specifying the desirability of affect states, plus a strategy for achieving high-$\phi$ states.
\end{definition}

\begin{example}[Stoicism]
\textbf{Historical context}: Hellenistic period, cosmopolitan empires. Given exposure to diverse cultures and the instability of fortune, a philosophy emphasizing internal control was inevitable.

\textbf{Affect policy}:
\begin{equation}
\phi_{\text{Stoic}}(\mathbf{a}) = -\arousal - \mathcal{CF} + \text{const}
\end{equation}
Stoicism targets low arousal (equanimity) and low counterfactual weight (focus on what is within control).

\textbf{Core techniques}:
\begin{itemize}
\item Dichotomy of control: Reduce $\mathcal{CF}$ on uncontrollable outcomes
\item Negative visualization: Controlled exposure to loss scenarios to reduce their arousal impact
\item View from above: Zoom out to cosmic perspective, reducing $\mathcal{SM}$
\end{itemize}

\textbf{Phenomenological result}: Equanimity---stable low arousal with moderate integration, regardless of external circumstances.
\end{example}

\begin{example}[Buddhism (Theravada)]
\textbf{Historical context}: Iron Age India, extreme asceticism proving ineffective. Given the persistence of suffering despite extreme practice, a middle path was inevitable.

\textbf{Affect policy}:
\begin{equation}
\phi_{\text{Buddhist}}(\mathbf{a}) = -\mathcal{SM} + \intinfo - |\valence| + \text{const}
\end{equation}
Target: very low self-model salience (anatt\=a), high integration (sam\=adhi), and reduced attachment to valence (equanimity toward pleasure and pain).

\textbf{Core techniques}:
\begin{itemize}
\item Sati (mindfulness): Observe arising/passing without identification
\item Sam\=adhi (concentration): Build integration capacity through sustained attention
\item Vipassan\=a (insight): See the constructed nature of self-model
\item Mett\=a (loving-kindness): Expand self-model to include all beings
\end{itemize}

\textbf{Phenomenological result}: The jhanas (meditative absorptions) represent systematically mapped affect states---from high positive valence with low $\mathcal{SM}$ (first jhana) to pure equanimity beyond valence (fourth jhana and beyond).
\end{example}

\begin{example}[Existentialism]
\textbf{Historical context}: Post-Nietzsche, post-WWI Europe. Given the death of God and collapse of traditional meaning structures, confrontation with groundlessness was inevitable.

\textbf{Affect policy}:
\begin{equation}
\phi_{\text{Existentialist}}(\mathbf{a}) = \mathcal{CF} + \effrank - \text{bad faith penalty}
\end{equation}
Existentialism embraces high counterfactual weight (awareness of radical freedom) and high effective rank (authentic engagement with possibilities). The strategy: confront anxiety rather than flee into ``bad faith.''

\textbf{Core concepts}:
\begin{itemize}
\item Existence precedes essence: No fixed nature, radical freedom
\item Radical freedom: High $\mathcal{CF}$---you could always choose otherwise
\item Angst: The affect signature of confronting freedom
\item Authenticity: Acting from genuine choice, not conformity
\item Absurdity: The gap between human meaning-seeking and cosmic indifference
\end{itemize}

\textbf{Phenomenological result}: A distinctive acceptance of difficulty---not eliminating negative valence but refusing to flee into self-deception. High $\mathcal{CF}$ and high $\effrank$ with full awareness of their cost.
\end{example}

\begin{table}[h]
\centering
\small
\begin{tabular}{lp{9cm}}
\toprule
\textbf{Philosophy} & \textbf{Target Structure (Constitutive Policy)} \\
\midrule
Stoicism & $\arousal{\downarrow}$, $\mathcal{CF}{\downarrow}$ (equanimity through control of attention) \\
Buddhism & $\mathcal{SM}{\downarrow\downarrow}$, $\arousal{\downarrow}$, $\intinfo{\uparrow}$ (self-dissolution through integration) \\
Existentialism & $\mathcal{CF}{\uparrow}$, $\effrank{\uparrow}$ (embrace radical freedom and its anxiety) \\
Hedonism & $\valence{\uparrow}$, $\arousal{\uparrow}$ (maximize positive intensity) \\
Epicureanism & $\valence{+}$ (moderate), $\arousal{\downarrow}$ (sustainable pleasure) \\
\bottomrule
\end{tabular}
\caption{Philosophical traditions as affect-space policies.}
\end{table}

\subsection{Information Technology as Affect Infrastructure}

Modern information technology constitutes affect infrastructure at civilizational scale, shaping the experiential structure of billions.

\begin{definition}[Affect Infrastructure]
\emph{Affect infrastructure} is any technological system that shapes affect distributions across populations:
\begin{equation}
\mathcal{T}: \{p_i(\mathbf{a})\}_{i \in \text{population}} \mapsto \{p'_i(\mathbf{a})\}_{i \in \text{population}}
\end{equation}
\end{definition}

\begin{proposition}[Social Media Affect Signature]
Social media platforms systematically produce:
\begin{itemize}
\item \textbf{Arousal spikes}: Notification-driven, intermittent reinforcement creates high-variance arousal
\item \textbf{Low integration}: Rapid context-switching fragments attention, reducing $\intinfo$
\item \textbf{High self-model salience}: Performance of identity, social comparison
\item \textbf{Counterfactual hijacking}: FOMO (fear of missing out) colonizes $\mathcal{CF}$ with social-comparison branches
\end{itemize}
\begin{equation}
\mathbf{a}_{\text{social media}} \approx (\text{variable }\valence, \text{high }\arousal, \text{low }\intinfo, \text{low }\effrank, \text{high }\mathcal{CF}, \text{high }\mathcal{SM})
\end{equation}
This is structurally similar to the anxiety motif.
\end{proposition}

\begin{proposition}[Algorithmic Feed Dynamics]
Engagement-optimizing algorithms create affect selection pressure:
\begin{equation}
\text{Content}_{\text{selected}} = \argmax_c \E[\text{engagement} | c] \approx \argmax_c |\Delta\valence(c)| + \Delta\arousal(c)
\end{equation}
Content that maximizes engagement is content that maximizes valence magnitude (outrage or delight) and arousal. This selects for affectively extreme content, shifting population affect distributions toward the tails.
\end{proposition}

\begin{definition}[Technology-Mediated Affect Drift]
The systematic shift in population affect distributions due to technology:
\begin{equation}
\frac{d\bar{\mathbf{a}}}{dt} = \sum_{\mathcal{T} \in \text{technologies}} w_\mathcal{T} \cdot \nabla_\mathbf{a} \mathcal{T}(\mathbf{a})
\end{equation}
where $w_\mathcal{T}$ is the population-weighted usage of technology $\mathcal{T}$.
\end{definition}

%------------------------------------------------------------------------------
% FIGURE SET 12: DIGITAL FRAGMENTATION
%------------------------------------------------------------------------------
\begin{figure}[htbp]
\centering
\begin{subfigure}[b]{0.45\textwidth}
    \centering
    % SEARCH: Screenshot of desktop with DOZENS of open windows, tabs, notifications.
    % Visual chaos of modern interface. Fragmented attention made visible.
    % Should induce slight overwhelm just from looking at it.
    \includegraphics[width=\textwidth]{fig12a_chaotic_screen.png}
    \caption*{(A)}
\end{subfigure}
\hfill
\begin{subfigure}[b]{0.45\textwidth}
    \centering
    % SEARCH: Phone screen showing infinite scroll - Instagram or TikTok feed.
    % The design that captures and holds. Should feel both familiar and slightly
    % disturbing when framed as object of analysis rather than used.
    \includegraphics[width=\textwidth]{fig12b_infinite_scroll.png}
    \caption*{(B)}
\end{subfigure}

\vspace{0.5cm}

\begin{subfigure}[b]{0.45\textwidth}
    \centering
    % SEARCH: Close-up of app icons with notification badges - the red circles
    % with numbers. Small but demanding. Should trigger slight arousal spike
    % even in a static image. The design that hijacks attention.
    \includegraphics[width=\textwidth]{fig12c_notification_badges.png}
    \caption*{(C)}
\end{subfigure}
\hfill
\begin{subfigure}[b]{0.45\textwidth}
    \centering
    % SEARCH: Person reading a physical book in natural light, OR library interior.
    % Extended, integrated attention. The format that produces high integration,
    % low arousal. CONTRAST to digital fragmentation.
    \includegraphics[width=\textwidth]{fig12d_reading_book.png}
    \caption*{(D)}
\end{subfigure}
\caption{\textbf{The affect infrastructure you inhabit.} You know what (A) feels like. The fragmentation is not metaphorical. Your integration ($\intinfo$) is literally being reduced by the structural demand to switch contexts. In (C), notice your response to the notification badges---perhaps a tiny spike of arousal, a pull toward the unread, even though these are not your notifications. This is affect engineering at population scale. Billions of humans are having their arousal chronically elevated by red circles designed in a room somewhere by people optimizing for engagement. In (B), notice the pull of the infinite scroll---the format that never provides closure, that always promises one more. In (D), notice the different quality of attention. The book enables something that the feed prevents. This is not nostalgia. It is structure. The formats produce the affects. You are the medium through which they work.}
\label{fig:digital-fragmentation}
\end{figure}

\subsection{Quantitative Frameworks}

\begin{definition}[Affect Impact Assessment]
For any intervention $\mathcal{I}$, the \emph{affect impact} is:
\begin{equation}
\text{Impact}(\mathcal{I}) = \E_{p'}[\mathbf{a}] - \E_p[\mathbf{a}]
\end{equation}
with component-wise analysis:
\begin{equation}
\text{Impact}(\mathcal{I}) = (\Delta\bar{\valence}, \Delta\bar{\arousal}, \Delta\bar{\intinfo}, \Delta\bar{\effrank}, \Delta\overline{\mathcal{CF}}, \Delta\overline{\mathcal{SM}})
\end{equation}
\end{definition}

\begin{definition}[Flourishing Score]
A weighted aggregate of affect dimensions aligned with human flourishing:
\begin{equation}
\mathcal{F}(\mathbf{a}) = \alpha_1 \valence + \alpha_2 \intinfo + \alpha_3 \effrank - \alpha_4 (\mathcal{SM} - \mathcal{SM}_{\text{optimal}})^2 - \alpha_5 |\arousal - \arousal_{\text{optimal}}|
\end{equation}
The weights $\{\alpha_i\}$ encode normative commitments about what constitutes flourishing.
\end{definition}

\begin{proposition}[Comparative Analysis]
Using standardized affect measurement, we can compare:
\begin{itemize}
\item Meditation retreat vs.\ social media usage (expected: opposite affect signatures)
\item Different workplace designs (open office vs.\ private: integration differences)
\item Educational approaches (lecture vs.\ discussion: counterfactual weight differences)
\item Urban vs.\ rural environments (arousal and integration differences)
\end{itemize}
\end{proposition}

%==============================================================================
\section{Summary of Part III}
%==============================================================================

\begin{enumerate}
\item \textbf{The existential burden}: Self-modeling systems cannot escape self-reference. Human culture is accumulated strategies for managing this burden.

\item \textbf{Aesthetics as affect technology}: Art forms have characteristic affect signatures and serve as technologies for transmitting experiential structure across minds and time.

\item \textbf{Sexuality as transcendence}: Sexual experience offers reliable, repeatable escape from the trap of self-reference through self-model merger and dissolution.

\item \textbf{Ideology as immortality project}: Identification with supra-individual patterns manages mortality terror by expanding the self-model's viability horizon.

\item \textbf{Science as meaning}: Scientific understanding produces high integration without self-focus---giving the self something worthy of its attention.

\item \textbf{Religion as systematic technology}: Religious traditions represent millennia of accumulated affect-engineering wisdom.

\item \textbf{Psychopathology as failed coping}: Mental illnesses are pathological attractors in affect space---attempted solutions that trap rather than liberate.

\item \textbf{Technology as infrastructure}: Modern information technology shapes affect distributions at population scale, often toward anxiety-like profiles.
\end{enumerate}

Part IV will develop:
\begin{itemize}
\item The grounding of normativity in viability structure
\item Scale-matched interventions from neurons to nations
\item Gods as agentic systems with their own viability manifolds
\item The AI alignment problem reframed at god-level
\end{itemize}

%==============================================================================
\section{Appendix: Symbol Reference}
%==============================================================================

\begin{description}
\item[$\valence$] Valence: gradient alignment on viability manifold
\item[$\arousal$] Arousal: rate of belief/state update
\item[$\intinfo$] Integration: irreducibility under partition
\item[$\effrank$] Effective rank: distribution of active degrees of freedom
\item[$\mathcal{CF}$] Counterfactual weight: resources on non-actual trajectories
\item[$\mathcal{SM}$] Self-model salience: degree of self-focus
\item[$\mathbf{a}$] Affect state vector: $(\valence, \arousal, \intinfo, \effrank, \mathcal{CF}, \mathcal{SM})$
\item[$\viable$] Viability manifold: region of sustainable states
\item[$\worldmodel$] World model: predictive model of environment
\item[$\selfmodel$] Self-model: component of world model representing self
\item[$B_{\text{exist}}$] Existential burden: cost of maintaining self-reference
\item[$\mathcal{I}$] Affect intervention: practice or technology that shifts affect distribution
\item[$\mathcal{F}$] Flourishing score: weighted aggregate of affect dimensions
\end{description}

\end{document}
