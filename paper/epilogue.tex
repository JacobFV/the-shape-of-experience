% Epilogue placeholder
% To be filled in by the author

% Suggested structure:
% - What has been accomplished?
% - What remains open?
% - Where do we go from here?
% - Final reflections

\vspace{2em}

\textit{[Epilogue to be written]}

\vspace{2em}

% The epilogue might cover:
%
% 1. The journey completed: From thermodynamic gradients to surfing
%    superintelligence
%
% 2. The central claims revisited:
%    - Affect is geometric, not epiphenomenal
%    - The hard problem dissolves under ontological democracy
%    - Human culture is affect engineering technology
%    - Gods have viability manifolds
%    - AI systems will develop affect-like structure under the
%      right forcing functions
%
% 3. What this changes:
%    - For philosophy of mind
%    - For AI development
%    - For therapeutic intervention
%    - For understanding human institutions
%
% 4. What remains to be done:
%    - Empirical validation of the framework
%    - Extension to multi-agent dynamics
%    - Development of affect-aware AI architectures
%    - Practical applications of scale-matched intervention
%
% 5. A final reflection on inevitability:
%    - Not determinism, but typicality
%    - Not design, but structure
%    - Not accident, but what systems of this kind become
%
% 6. Closing thought: The shape of experience is the shape of viability
%    navigated under constraint. You are that shape, reading about itself.
