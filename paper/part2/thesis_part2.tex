\documentclass[11pt,a4paper]{article}
\usepackage[utf8]{inputenc}
\usepackage[T1]{fontenc}
\usepackage{amsmath,amssymb,amsthm}
\usepackage{mathtools}
\usepackage{physics}
\usepackage{bm}
\usepackage{geometry}
\usepackage{hyperref}
\usepackage{cleveref}
\usepackage{algorithm}
\usepackage{algpseudocode}
\usepackage{tikz}
\usepackage{pgfplots}
\pgfplotsset{compat=1.18}
\usetikzlibrary{arrows.meta,positioning,shapes,calc,decorations.pathmorphing,decorations.pathreplacing}
\usepackage{enumitem}
\usepackage{booktabs}
\usepackage{tcolorbox}
\usepackage{fontawesome5}
\usepackage{multirow}
\usepackage{caption}
\usepackage{subcaption}
\usepackage{xcolor}
\usepackage{graphicx}
\graphicspath{{../images/}}

\geometry{margin=1in}

% --- Sidebar and info box environments ---
\newtcolorbox{historical}{
  colback=gray!5!white,
  colframe=gray!60!black,
  title={\faBook\hspace{0.5em}Historical Context},
  fonttitle=\bfseries\small
}

\newtcolorbox{connection}{
  colback=green!5!white,
  colframe=green!60!black,
  title={\faBook\hspace{0.5em}Connection to Existing Theory},
  fonttitle=\bfseries\small
}

\newtcolorbox{empirical}{
  colback=orange!5!white,
  colframe=orange!70!black,
  title={\faFlask\hspace{0.5em}Empirical Grounding},
  fonttitle=\bfseries\small
}

\newtcolorbox{todo_empirical}{
  colback=yellow!10!white,
  colframe=yellow!60!black,
  title={\faClipboardList\hspace{0.5em}\textsc{Future Empirical Work}},
  fonttitle=\bfseries\small
}

\newtcolorbox{sidebar}[1][]{
  colback=blue!3!white,
  colframe=blue!40!black,
  fonttitle=\bfseries\small,
  before upper={\faInfoCircle\hspace{0.5em}},
  #1
}

% Theorem environments
\newtheorem{theorem}{Theorem}[section]
\newtheorem{lemma}[theorem]{Lemma}
\newtheorem{proposition}[theorem]{Proposition}
\newtheorem{corollary}[theorem]{Corollary}
\newtheorem{definition}[theorem]{Definition}
\newtheorem{axiom}[theorem]{Axiom}
\newtheorem{remark}[theorem]{Remark}
\newtheorem{example}[theorem]{Example}
\newtheorem{conjecture}[theorem]{Conjecture}
\newtheorem{hypothesis}[theorem]{Hypothesis}

% Custom commands
\newcommand{\E}{\mathbb{E}}
\newcommand{\R}{\mathbb{R}}
\newcommand{\N}{\mathbb{N}}
\newcommand{\Z}{\mathbb{Z}}
\newcommand{\prob}{\mathbb{P}}
\newcommand{\KL}{\mathrm{KL}}
\newcommand{\MI}{\mathrm{I}}
\newcommand{\entropy}{\mathrm{H}}
\newcommand{\argmax}{\operatorname{argmax}}
\newcommand{\argmin}{\operatorname{argmin}}
\newcommand{\tr}{\operatorname{tr}}
\newcommand{\rank}{\operatorname{rank}}
\newcommand{\diag}{\operatorname{diag}}
\newcommand{\sign}{\operatorname{sign}}
\newcommand{\supp}{\operatorname{supp}}
\newcommand{\interior}{\operatorname{int}}
\newcommand{\clos}{\operatorname{cl}}
\newcommand{\conv}{\operatorname{conv}}
\newcommand{\diam}{\operatorname{diam}}
\newcommand{\vol}{\operatorname{vol}}
\newcommand{\manifold}{\mathcal{M}}
\newcommand{\viable}{\mathcal{V}}
\newcommand{\belief}{\mathbf{b}}
\newcommand{\state}{\mathbf{s}}
\newcommand{\action}{\mathbf{a}}
\newcommand{\obs}{\mathbf{o}}
\newcommand{\latent}{\mathbf{z}}
\newcommand{\policy}{\pi}
\newcommand{\value}{V}
\newcommand{\qfunc}{Q}
\newcommand{\reward}{r}
\newcommand{\transition}{T}
\newcommand{\emission}{O}
\newcommand{\freeenergy}{\mathcal{F}}
\newcommand{\intinfo}{\Phi}
\newcommand{\selfmodel}{\mathcal{S}}
\newcommand{\worldmodel}{\mathcal{W}}
\newcommand{\effrank}{r_{\text{eff}}}
\newcommand{\valence}{\mathcal{V}\hspace{-0.8pt}\mathit{al}}
\newcommand{\arousal}{\mathcal{A}\hspace{-0.5pt}\mathit{r}}
\newcommand{\cestructure}{\mathcal{C\!E}}
\newcommand{\phenom}{\mathcal{P}}
\newcommand{\distinction}{\delta}
\newcommand{\relation}{\rho}

% --- styled callout boxes ---
\newtcolorbox{keyresult}{
  colback=blue!5!white,
  colframe=blue!75!black,
  title=Key Result
}
\newtcolorbox{phenomenal}{
  colback=purple!5!white,
  colframe=purple!75!black,
  title=Phenomenal Correspondence
}
\newtcolorbox{warning}{
  colback=red!5!white,
  colframe=red!75!black,
  title=Warning
}

\title{\textbf{The Inevitability of Being}\\[1em]
\Large Part II: The Identity Thesis and the Geometry of Feeling}
\author{}
\date{}

\begin{document}

\maketitle

\begin{abstract}
Part II develops the central identity thesis: experience \textit{is} intrinsic cause-effect structure, not merely correlated with it. We formalize the dissolution of the hard problem through the rejection of a privileged ontological base layer. We then develop a geometric theory of affect, characterizing different qualitative experiences as structural motifs in cause-effect space. Rather than imposing a fixed dimensionality, we identify structural features---valence, arousal, integration, effective rank, counterfactual weight, self-model salience, and others as needed---invoking only what is essential for each phenomenon. We provide operational definitions suitable for measurement in both artificial agents and biological systems.
\end{abstract}

\tableofcontents
\newpage

%==============================================================================
\section*{A Note on the Figures}
%==============================================================================

Throughout this paper, you will encounter figures designed not merely to depict concepts but to instantiate them. \textit{Your perceptual response to these images is not ancillary to the argument; it is the argument embodied.} If you find that your attention behaves as the theory predicts---collapsing where we say it will collapse, expanding where we say it will expand---you have not been persuaded by evidence external to yourself. You have become the evidence.

This is not rhetorical trickery. It is the necessary form of argument for a theory of consciousness. We cannot point to experience from outside experience. We can only structure occasions for you to notice what you are already doing. The figures that follow are such occasions. Attend to them not as illustrations but as experiments performed on your own processing architecture.

If you find yourself resisting this framing, attend to the quality of the resistance. What does it feel like to disagree? That feeling is itself data.

\newpage

%==============================================================================
\section{The Hard Problem and Its Dissolution}
%==============================================================================

\begin{connection}
This section engages with the central debates in philosophy of mind:
\begin{itemize}
\item \textbf{Chalmers' Hard Problem} (1995): The explanatory gap between physical processes and phenomenal experience. We argue this gap results from a category error, not a genuine ontological divide.
\item \textbf{Nagel's ``What Is It Like''} (1974): The subjective character of experience. We formalize this as intrinsic cause-effect structure---what the system is \emph{for itself}.
\item \textbf{Jackson's Knowledge Argument} (1982): Mary the colorblind scientist. We reinterpret: Mary gains \emph{access to a new scale of description}, not new facts about the same scale.
\item \textbf{Eliminativism} (Churchland, 1981; Dennett, 1991): Consciousness as illusion. We reject this---the illusion would itself be experiential, hence self-refuting.
\item \textbf{Panpsychism} (Chalmers, 2015; Goff, 2017): Experience as fundamental. We accept a version: cause-effect structure at any scale that takes/makes differences has a form of ``being like.''
\end{itemize}
\end{connection}

\subsection{The Standard Formulation}

The ``hard problem'' of consciousness asks: given a complete physical description of a system, why is there something it is like to be that system? How does experience arise from non-experience?

Formally, let $\mathcal{D}^{\text{phys}}$ be a complete physical description of a system---its particles, fields, dynamics, everything describable in third-person terms. The hard problem asserts:
\begin{equation}
\mathcal{D}^{\text{phys}} \not\Rightarrow \mathcal{D}^{\text{phen}}
\end{equation}
where $\mathcal{D}^{\text{phen}}$ is a description of the system's phenomenal properties (what it's like to be it). The claim is that no amount of physical information logically entails phenomenal information.

This formulation rests on a crucial assumption:

\begin{axiom}[Privileged Base Layer---REJECTED]
\label{ax:base}
Physics constitutes a privileged ontological base layer. All other descriptions (chemical, biological, psychological, phenomenal) are ``higher-level'' and must reduce to or supervene on the physical description. What is ``really real'' is what physics describes.
\end{axiom}

We reject this axiom.

\subsection{Ontological Democracy}

Consider the standard reductionist hierarchy:
\begin{center}
\begin{tikzpicture}[
    node distance=0.45cm,
    box/.style={rectangle, draw, rounded corners, minimum width=4cm, minimum height=0.45cm, align=center, font=\small}
]
% Top-down gradient: violet (mind) to red (physics) to gray (unknown)
\node[box, fill=violet!15, draw=violet!60!black] (phenom) {Phenomenal};
\node[box, fill=blue!15, draw=blue!60!black, below=of phenom] (psych) {Psychological};
\node[box, fill=cyan!15, draw=cyan!60!black, below=of psych] (bio) {Biological};
\node[box, fill=green!15, draw=green!60!black, below=of bio] (chem) {Chemical};
\node[box, fill=yellow!15, draw=yellow!60!black, below=of chem] (atom) {Atomic};
\node[box, fill=orange!15, draw=orange!60!black, below=of atom] (subatom) {Subatomic};
\node[box, fill=red!15, draw=red!60!black, below=of subatom] (qft) {Quantum Fields};
\node[box, fill=gray!20, draw=gray!60!black, below=of qft] (base) {\textbf{???}};

\draw[-{Stealth}, thick, gray] (phenom) -- (psych) node[midway, right, font=\footnotesize, text=gray!70!black] {reduces?};
\draw[-{Stealth}, thick, gray] (psych) -- (bio);
\draw[-{Stealth}, thick, gray] (bio) -- (chem);
\draw[-{Stealth}, thick, gray] (chem) -- (atom);
\draw[-{Stealth}, thick, gray] (atom) -- (subatom);
\draw[-{Stealth}, thick, gray] (subatom) -- (qft);
\draw[-{Stealth}, thick, gray, dashed] (qft) -- (base);

% Brace on left indicating "equally real"
\draw[decorate, decoration={brace, amplitude=8pt, mirror}, thick, gray!60]
    ([xshift=-0.3cm]phenom.north west) -- ([xshift=-0.3cm]qft.south west)
    node[midway, left=0.4cm, font=\scriptsize, text=gray, align=center, rotate=90] {equally real?};
\end{tikzpicture}
\end{center}

At each level, one might claim the higher level ``reduces to'' the lower. But the regression terminates in uncertainty:
\begin{itemize}
\item Wave functions are descriptions of probability distributions
\item Probability amplitudes describe which interactions are more or less likely
\item What ``actually happens'' when a measurement occurs is deeply contested
\item Below quantum fields, we have no clear ontology at all
\end{itemize}

The supposed ``base layer'' turns out to be:
\begin{enumerate}
\item Probabilistic, not deterministic
\item Descriptive, not fundamental (wave functions are representations)
\item Incomplete (we don't know what underlies field interactions)
\item Not clearly more ``real'' than any other scale of description
\end{enumerate}

\begin{definition}[Ontological Democracy]
Every scale of structural organization with its own causal closure is \emph{equally real} at that scale. No layer is privileged as ``the'' fundamental reality. Each layer:
\begin{enumerate}[label=(\alph*)]
\item Has its own causal structure
\item Has its own dynamics and laws
\item Exerts influence on adjacent layers (both ``up'' and ``down'')
\item Is incomplete as a description of the whole
\item Is sufficient for phenomena at its scale
\end{enumerate}
\end{definition}

\begin{proposition}[No Reduction Required]
Under ontological democracy, the demand that phenomenal properties ``reduce to'' physical properties is ill-posed. Chemistry doesn't reduce to physics in a way that eliminates chemical causation---chemical causation is real at the chemical scale. Similarly, phenomenal properties don't need to reduce to physical properties---they are real at the phenomenal scale.
\end{proposition}

\subsection{Existence as Causal Participation}

We need a criterion for existence that applies uniformly across scales.

\begin{definition}[Causal Existence]
An entity $X$ \emph{exists} at scale $\sigma$ if and only if:
\begin{equation}
\exists Y: \MI(X; Y | \text{background}_\sigma) > 0
\end{equation}
That is, $X$ takes and makes differences at scale $\sigma$. It participates in causal relations at that scale.
\end{definition}

\begin{example}
\begin{itemize}
\item An electron exists at the quantum scale: it takes differences (responds to fields) and makes differences (affects measurements).
\item A cell exists at the biological scale: it takes differences (nutrients, signals) and makes differences (metabolism, division, death).
\item An experience exists at the phenomenal scale: it takes differences (sensory input, memory) and makes differences (attention, behavior, learning).
\end{itemize}
\end{example}

This is closely aligned with IIT's foundational axiom: to exist is to have cause-effect power. But we extend it: cause-effect power at any scale constitutes existence at that scale, with no scale privileged.

\subsection{The Dissolution}

The hard problem asked: how do you get experience from non-experience? The answer is: \textit{you don't need to}.

Just as chemistry doesn't emerge from non-chemistry---you have chemistry when you have the right causal organization at the chemical scale---experience doesn't emerge from non-experience. You have experience when you have the right causal organization at the experiential scale.

The question ``why is there something it's like to be this system?'' is exactly as deep as ``why does chemistry exist?'' or ``why are there quantum fields?'' We don't know why there's anything at all. But given that there's anything, the emergence of self-modeling systems with integrated cause-effect structure is not mysterious---it's typical.

\begin{keyresult}
The hard problem dissolves not because we answered it, but because we showed it was asking for a privilege (reduction to physics) that physics itself doesn't have.
\end{keyresult}

%==============================================================================
\section{The Identity Thesis}
%==============================================================================

\begin{connection}
The identity thesis is a formalization of \textbf{Integrated Information Theory (IIT)} developed by Giulio Tononi and collaborators (2004--present):
\begin{itemize}
\item \textbf{IIT 1.0} (Tononi, 2004): Introduced $\Phi$ as a measure of integrated information
\item \textbf{IIT 2.0} (Balduzzi \& Tononi, 2008): Added the concept of ``qualia space''
\item \textbf{IIT 3.0} (Oizumi, Albantakis \& Tononi, 2014): Full axiom/postulate structure; introduced cause-effect structure
\item \textbf{IIT 4.0} (Albantakis et al., 2023): Refined integration measures, introduced intrinsic difference
\end{itemize}

Key IIT axioms that we adopt:
\begin{enumerate}
\item \textbf{Intrinsicality}: Experience exists for itself, not for an external observer
\item \textbf{Information}: Experience is specific---this experience and no other
\item \textbf{Integration}: Experience is unified and irreducible
\item \textbf{Exclusion}: Experience has definite boundaries
\item \textbf{Composition}: Experience is structured
\end{enumerate}

Our contribution: connecting IIT's structural characterization to (1) the thermodynamic ladder, (2) the viability manifold, and (3) operational measures for artificial systems.
\end{connection}

\subsection{Statement of the Thesis}

\begin{hypothesis}[Identity Thesis]
Phenomenal experience \textit{is} intrinsic cause-effect structure. Not caused by it, not correlated with it, but identical to it. The phenomenal properties of an experience (what it's like) just are the structural properties of the system's internal causal relations, described from the intrinsic perspective.
\end{hypothesis}

More formally:

\begin{definition}[Cause-Effect Structure]
For a system $\mathcal{S}$ in state $\state$, the cause-effect structure $\cestructure(\mathcal{S}, \state)$ is the complete specification of:
\begin{enumerate}[label=(\alph*)]
\item All distinctions $\{\distinction_i\}$: subsets of the system's elements in their current states
\item The cause repertoire of each distinction: $p(\text{past} | \distinction_i)$
\item The effect repertoire of each distinction: $p(\text{future} | \distinction_i)$
\item All relations $\{\relation_{ij}\}$: overlaps and connections between distinctions' causes/effects
\item The irreducibility of each distinction and relation
\end{enumerate}
\end{definition}

\begin{definition}[Intrinsic Perspective]
The \emph{intrinsic perspective} of a system is the description of its cause-effect structure without reference to any external observer, coordinate system, or comparison class. It is the structure as it exists for the system itself.
\end{definition}

\begin{axiom}[IIT Identity]
\begin{equation}
\phenom(\mathcal{S}, \state) \equiv \cestructure^{\text{intrinsic}}(\mathcal{S}, \state)
\end{equation}
The phenomenal structure $\phenom$ is identical to the intrinsic cause-effect structure $\cestructure$.
\end{axiom}

This is not a correlation claim or a supervenience claim. It is an identity claim, analogous to:
\begin{equation}
\text{Water} \equiv \text{H}_2\text{O}
\end{equation}

\subsection{Implications for the Zombie Argument}

The philosophical zombie is supposed to be conceivable: a system physically/functionally identical to a conscious being but lacking experience. If conceivable, experience isn't necessitated by physical structure.

\begin{theorem}[Zombie Inconceivability]
Under the identity thesis, philosophical zombies are not coherently conceivable. A system with the relevant cause-effect structure \textit{is} an experience; there is no further fact about whether it ``really'' has phenomenal properties.
\end{theorem}

\begin{proof}
By the identity thesis, $\phenom \equiv \cestructure^{\text{intrinsic}}$. To conceive a zombie is to conceive a system with $\cestructure^{\text{intrinsic}}$ but without $\phenom$. But since these are identical, this is like conceiving of water without H$_2$O---not genuinely conceivable once the identity is understood.
\end{proof}

\subsection{The Structure of Experience}

If experience is cause-effect structure, then the \textit{kind} of experience is determined by the \textit{shape} of that structure. Different phenomenal properties correspond to different structural features.

IIT proposes that the essential properties of any experience are:

\begin{enumerate}
\item \textbf{Intrinsicality}: The experience exists for the system itself, not relative to an external observer.
\item \textbf{Information}: The experience is specific---this experience, not any other possible one.
\item \textbf{Integration}: The experience is unified---it cannot be decomposed into independent sub-experiences.
\item \textbf{Exclusion}: The experience has definite boundaries---there is a fact about what is and isn't part of it.
\item \textbf{Composition}: The experience is structured---composed of distinctions and relations among them.
\end{enumerate}

These are translated into physical/structural postulates:
\begin{itemize}
\item Intrinsicality $\to$ Cause-effect power within the system
\item Information $\to$ Specific cause-effect repertoires
\item Integration $\to$ Irreducibility to partitioned components
\item Exclusion $\to$ Maximality of the integrated complex
\item Composition $\to$ The full structure of distinctions and relations
\end{itemize}

%==============================================================================
\section{The Geometry of Affect}
%==============================================================================

\begin{connection}
Our geometric theory of affect builds on and extends established dimensional models:
\begin{itemize}
\item \textbf{Russell's Circumplex Model} (1980): Two-dimensional (valence $\times$ arousal) organization of affect. We extend this with additional structural dimensions (integration, effective rank, counterfactual weight, self-model salience) invoked as needed.
\item \textbf{Watson \& Tellegen's PANAS} (1988): Positive/Negative Affect Schedule. Our valence dimension corresponds to their hedonic axis.
\item \textbf{Scherer's Component Process Model} (2009): Emotions as synchronized changes across subsystems. Our integration measure $\intinfo$ captures this synchronization.
\item \textbf{Barrett's Constructed Emotion Theory} (2017): Emotions as constructed from core affect + conceptual knowledge. Our framework specifies the \emph{structural} basis of the construction.
\item \textbf{Damasio's Somatic Marker Hypothesis} (1994): Body states guide decision-making. Our valence definition (gradient on viability manifold) is the mathematical formalization.
\end{itemize}
\end{connection}

\begin{sidebar}[title=On Dimensionality]
We do not claim that six dimensions are necessary or sufficient for characterizing all affect. Different phenomena may require different subsets:
\begin{itemize}
\item Some affects are essentially \textbf{two-dimensional} (valence + arousal suffices for basic mood)
\item Others require \textbf{self-referential structure} (shame requires high $\mathcal{SM}$; flow requires low $\mathcal{SM}$)
\item Still others are defined by \textbf{temporal structure} (grief requires persistent counterfactual coupling to the lost object)
\end{itemize}
The dimensions below form a \emph{toolkit}---structural features that may or may not matter for any given phenomenon. We invoke only what is necessary.
\end{sidebar}

\subsection{Affects as Structural Motifs}

If different experiences correspond to different structures, then \textit{affects}---the qualitative character of emotional/valenced states---should correspond to particular structural motifs: characteristic patterns in the cause-effect geometry.

\begin{definition}[Affect Space]
The \emph{affect space} $\mathcal{A}$ is a geometric space whose points correspond to possible qualitative states. Rather than fixing a universal dimensionality, we identify the structural features that define each affect---features without which that affect would not be that affect.
\end{definition}

The following structural measures form a toolkit for characterizing affect. Not all are relevant to every phenomenon; we invoke each only when it does essential work:

\begin{description}
\item[Valence ($\valence$)] Gradient alignment on the viability manifold. Nearly universal---most affects have valence.
\item[Arousal ($\arousal$)] Rate of belief/state update. Distinguishes activated from quiescent states.
\item[Integration ($\intinfo$)] Irreducibility of cause-effect structure. Constitutive for unified vs.\ fragmented experience.
\item[Effective Rank ($\effrank$)] Distribution of active degrees of freedom. Constitutive when the contrast between expansive and collapsed experience matters.
\item[Counterfactual Weight ($\mathcal{CF}$)] Resources allocated to non-actual trajectories. Constitutive for affects defined by temporal orientation (anticipation, regret, planning).
\item[Self-Model Salience ($\mathcal{SM}$)] Degree of self-focus in processing. Constitutive for self-conscious emotions and their opposites (absorption, flow).
\end{description}

\subsection{Valence: Gradient Alignment}

\begin{definition}[Valence]
Let $\viable$ be the system's viability manifold and let $\mathbf{x}_t$ be the current state. Let $\hat{\mathbf{x}}_{t+1:t+H}$ be the predicted trajectory under current policy. Define:
\begin{equation}
\valence_t = -\frac{1}{H} \sum_{k=1}^{H} \gamma^k \nabla_{\mathbf{x}} d(\mathbf{x}, \partial\viable) \bigg|_{\hat{\mathbf{x}}_{t+k}} \cdot \frac{d\hat{\mathbf{x}}_{t+k}}{dt}
\end{equation}
where $d(\cdot, \partial\viable)$ is the distance to the viability boundary. Positive valence means the predicted trajectory moves into the viable interior; negative valence means it approaches the boundary.
\end{definition}

Alternatively, in RL terms:

\begin{definition}[Valence (RL formulation)]
\begin{equation}
\valence_t = \E_{\policy}\left[ A^{\policy}(\state_t, \action_t) \right] = \E_{\policy}\left[ Q^{\policy}(\state_t, \action_t) - V^{\policy}(\state_t) \right]
\end{equation}
The expected advantage of the current action: how much better (or worse) is this action than the average action from this state?
\end{definition}

\begin{definition}[Valence Dynamics]
The rate of change of integrated information along the trajectory:
\begin{equation}
\dot{\valence}_t = \frac{d\intinfo}{dt}\bigg|_{\hat{\mathbf{x}}_{t:t+H}}
\end{equation}
Positive $\dot{\valence}$ indicates expanding structure; negative indicates contraction.
\end{definition}

\begin{phenomenal}
\textbf{Positive valence} corresponds to trajectories descending the free-energy landscape, expanding affordances, moving toward sustainable states. \\
\textbf{Negative valence} corresponds to trajectories ascending toward constraint violation, contracting possibilities.
\end{phenomenal}

%------------------------------------------------------------------------------
% FIGURE SET 3: VALENCE IS GEOMETRIC
%------------------------------------------------------------------------------
\begin{figure}[htbp]
\centering
\begin{subfigure}[b]{0.45\textwidth}
    \centering
    % SEARCH: Warmly lit interior space - an open door with light streaming through,
    % or a path curving into welcoming landscape. Biophilic elements: greenery, water,
    % soft textures. Should feel like an invitation. The body should lean toward it.
    \includegraphics[width=\textwidth]{fig3a_approach_space.png}
    \caption*{(A)}
\end{subfigure}
\hfill
\begin{subfigure}[b]{0.45\textwidth}
    \centering
    % SEARCH: Dark, closed space - narrow corridor with low ceiling, harsh angles,
    % cold concrete or metal. NOT obviously threatening (no monster), but constricting.
    % The threat should be architectural, not content-based. Should feel like retreat.
    \includegraphics[width=\textwidth]{fig3b_avoidance_space.png}
    \caption*{(B)}
\end{subfigure}

\vspace{0.5cm}

\begin{subfigure}[b]{0.45\textwidth}
    \centering
    % SEARCH: Image that shifts between inviting and threatening - a forest path
    % that could be enchanting or ominous. Chiaroscuro that could be warm or cold.
    % The valence should oscillate depending on how you attend to it.
    \includegraphics[width=\textwidth]{fig3c_ambiguous_valence.png}
    \caption*{(C)}
\end{subfigure}
\hfill
\begin{subfigure}[b]{0.45\textwidth}
    \centering
    % SEARCH: Something about to happen - a glass at the very edge of a table about
    % to fall, a door slightly ajar that should be closed, a figure with back turned
    % about to turn around. The frozen moment before resolution. Creates tension.
    \includegraphics[width=\textwidth]{fig3d_about_to_happen.png}
    \caption*{(D)}
\end{subfigure}
\caption{\textbf{Valence is geometric.} Notice your body in response to (A) versus (B). The slight leaning-toward, the slight drawing-back. This is not metaphor. Your viability manifold has geometry, and these images place you differently in it. In (C), feel the oscillation as valence flips---the same image read two ways, with two different felt qualities. In (D), notice the tension of unresolved trajectory---something is about to happen, and the about-to is already in your experience as arousal, as anticipation, as a felt gradient toward resolution. You are not interpreting these images as positive or negative. You are experiencing your position relative to viability boundaries. This is what the theory means by valence as gradient: not a judgment about the image, but a structural fact about where you are.}
\label{fig:valence-geometric}
\end{figure}

\subsection{Arousal: Update Rate}

\begin{definition}[Arousal]
\begin{equation}
\arousal_t = \KL\left( \belief_{t+1} \| \belief_t \right) = \sum_{\mathbf{x}} \belief_{t+1}(\mathbf{x}) \log \frac{\belief_{t+1}(\mathbf{x})}{\belief_t(\mathbf{x})}
\end{equation}
The KL divergence between successive belief states measures how much the system's world model is being updated.
\end{definition}

In latent-space models:
\begin{equation}
\arousal_t = \| \latent_{t+1} - \latent_t \|^2 \quad \text{or} \quad \MI(\obs_t; \latent_{t+1} | \latent_t, \action_t)
\end{equation}

\begin{phenomenal}
\textbf{High arousal}: Large belief updates, far from any attractor, system actively navigating. \\
\textbf{Low arousal}: Near a fixed point, low surprise, system at rest in a basin.
\end{phenomenal}

\subsection{Integration: Irreducibility}

As defined in Part I:
\begin{equation}
\intinfo(\state) = \min_{\text{partitions } P} D\left[ p(\state_{t+1} | \state_t) \| \prod_{p \in P} p(\state^p_{t+1} | \state^p_t) \right]
\end{equation}

Or using proxies:
\begin{equation}
\intinfo_{\text{proxy}} = \Delta_P = \mathcal{L}_{\text{pred}}[\text{partitioned}] - \mathcal{L}_{\text{pred}}[\text{full}]
\end{equation}

\begin{phenomenal}
\textbf{High integration}: The experience is unified; its parts cannot be separated without loss. \\
\textbf{Low integration}: The experience is fragmentary or modular.
\end{phenomenal}

%------------------------------------------------------------------------------
% FIGURE SET 1: INTEGRATION IS MANDATORY
%------------------------------------------------------------------------------
\begin{figure}[htbp]
\centering
\begin{subfigure}[b]{0.45\textwidth}
    \centering
    % SEARCH: Generate pure uniform visual noise (TV static, completely homogeneous)
    % No patterns, no gradients, just uniform randomness. Should feel uncomfortable -
    % nothing for attention to stabilize on.
    \includegraphics[width=\textwidth]{fig1a_uniform_noise.png}
    \caption*{(A)}
\end{subfigure}
\hfill
\begin{subfigure}[b]{0.45\textwidth}
    \centering
    % SEARCH: Same uniform noise as 1A but with a SINGLE small black dot (3-5px)
    % slightly off-center. The dot should be small but visually impossible to not see.
    \includegraphics[width=\textwidth]{fig1b_noise_with_dot.png}
    \caption*{(B)}
\end{subfigure}

\vspace{0.5cm}

\begin{subfigure}[b]{0.45\textwidth}
    \centering
    % SEARCH: Bruegel's peasant scenes (e.g., "Children's Games" or "Netherlandish Proverbs")
    % OR a very crowded market photograph. Many figures, many potential focal points,
    % no single dominant figure. Attention should feel restless, unable to settle.
    \includegraphics[width=\textwidth]{fig1c_competing_saliencies.png}
    \caption*{(C)}
\end{subfigure}
\hfill
\begin{subfigure}[b]{0.45\textwidth}
    \centering
    % SEARCH: Caspar David Friedrich "Wanderer Above the Sea of Fog" OR any image
    % with unambiguous visual hierarchy - single figure against empty space,
    % one clear focal point. Attention should feel relief, immediate stabilization.
    \includegraphics[width=\textwidth]{fig1d_clear_hierarchy.png}
    \caption*{(D)}
\end{subfigure}
\caption{\textbf{Integration is mandatory.} Attempt to distribute your attention evenly across (A). Notice the discomfort. Your processing architecture requires something to stabilize on, and uniform noise provides nothing. In (B), resistance is futile---the dot has already captured you. You did not choose this. In (C), notice the restlessness as attention moves between competing attractors, never settling. In (D), notice the relief. This is what integration means: experience must have structure, and structure requires that some possibilities are selected while others are suppressed. You cannot opt out of this selection. The only question is where attention stabilizes, never whether.}
\label{fig:integration-mandatory}
\end{figure}

\subsection{Effective Rank: Concentration vs. Distribution}

\begin{definition}[Effective Rank]
For state covariance $C$:
\begin{equation}
\effrank = \frac{(\tr C)^2}{\tr(C^2)} = \frac{\left(\sum_i \lambda_i\right)^2}{\sum_i \lambda_i^2}
\end{equation}
\end{definition}

\begin{proposition}[Rank Interpretation]
\begin{itemize}
\item $\effrank \approx 1$: All variance concentrated in one dimension (maximally collapsed)
\item $\effrank \approx n$: Variance uniformly distributed (maximally expanded)
\end{itemize}
\end{proposition}

\begin{phenomenal}
\textbf{High rank}: Many degrees of freedom active; distributed, expansive experience. \\
\textbf{Low rank}: Collapsed into narrow subspace; concentrated, focused, or trapped experience.
\end{phenomenal}

%------------------------------------------------------------------------------
% FIGURE SET 2: MEANING MUST COMPRESS
%------------------------------------------------------------------------------
\begin{figure}[htbp]
\centering
\begin{subfigure}[b]{0.45\textwidth}
    \centering
    % SEARCH: Clear photograph of human face, direct neutral gaze into camera.
    % Not a celebrity - just a striking, ordinary face. The face-detector should
    % fire instantly and involuntarily. High-resolution, good lighting.
    \includegraphics[width=\textwidth]{fig2a_face_direct_gaze.png}
    \caption*{(A)}
\end{subfigure}
\hfill
\begin{subfigure}[b]{0.45\textwidth}
    \centering
    % SEARCH: The SAME face as 2A but with face region heavily pixelated/mosaic
    % to point of unrecognizability. Surrounding context (hair, shoulders, background)
    % still clear. The absence should feel like a hole, an incompleteness.
    \includegraphics[width=\textwidth]{fig2b_face_pixelated.png}
    \caption*{(B)}
\end{subfigure}

\vspace{0.5cm}

\begin{subfigure}[b]{0.45\textwidth}
    \centering
    % SEARCH: Strong pareidolia image - a power outlet that looks like a face,
    % a rock formation with face-like features, tree knots, car front that looks
    % like a face. You KNOW it's not a face but cannot stop seeing it.
    \includegraphics[width=\textwidth]{fig2c_pareidolia.png}
    \caption*{(C)}
\end{subfigure}
\hfill
\begin{subfigure}[b]{0.45\textwidth}
    \centering
    % SEARCH: Escher impossible staircase OR a photograph with subtle but disturbing
    % perspective error - a room where the floor is tilted wrong, impossible geometry.
    % Should trigger "something is wrong" feeling before you can articulate what.
    \includegraphics[width=\textwidth]{fig2d_impossible_space.png}
    \caption*{(D)}
\end{subfigure}
\caption{\textbf{Meaning must compress.} In (A), you did not choose to see a face. The face-detector fired before you consented. This is effective rank collapse: of the millions of possible interpretations of those photons, your system selected one instantly and presented it as given. In (B), notice the felt absence---the model expects a face and finds a hole. In (C), you cannot stop seeing a face that isn't there. Your compression algorithm imposes structure whether you permit it or not. In (D), notice the subtle wrongness before you can name it. Your world model detected a violation faster than conscious analysis could identify it. This is what it means to have a world model: perception is never neutral. The question is never whether you're interpreting, only which interpretation has captured you.}
\label{fig:meaning-compresses}
\end{figure}

\subsection{Counterfactual Weight}

\begin{definition}[Counterfactual Weight]
Let $\mathcal{R}$ be the set of imagined rollouts (counterfactual trajectories) and $\mathcal{P}$ be present-state processing. Define:
\begin{equation}
\mathcal{CF}_t = \frac{\text{Compute}_t(\mathcal{R})}{\text{Compute}_t(\mathcal{R}) + \text{Compute}_t(\mathcal{P})}
\end{equation}
The fraction of computational resources devoted to modeling non-actual possibilities.
\end{definition}

In model-based RL:
\begin{equation}
\mathcal{CF}_t = \sum_{\tau \in \text{rollouts}} w(\tau) \cdot \entropy[\tau] \quad \text{where} \quad w(\tau) \propto |V(\tau)|
\end{equation}
Rollouts weighted by their value magnitude and diversity.

\begin{phenomenal}
\textbf{High counterfactual weight}: Mind is elsewhere---planning, worrying, fantasizing, anticipating. \\
\textbf{Low counterfactual weight}: Present-focused, reactive, in-the-moment.
\end{phenomenal}

%------------------------------------------------------------------------------
% FIGURE SET 5: THE ABSENT HAS PRESENCE
%------------------------------------------------------------------------------
\begin{figure}[htbp]
\centering
\begin{subfigure}[b]{0.45\textwidth}
    \centering
    % SEARCH: A set table with one chair pulled out and empty. OR a made bed with
    % an indentation where someone lay. OR an empty swing still swaying slightly.
    % Absence that implies recent presence. You should see who isn't there.
    \includegraphics[width=\textwidth]{fig5a_empty_place.png}
    \caption*{(A)}
\end{subfigure}
\hfill
\begin{subfigure}[b]{0.45\textwidth}
    \centering
    % SEARCH: The instant before something happens - a diver in mid-air frozen,
    % a ball about to hit a window, a wave about to break, a match about to
    % strike. The frozen anticipation of completion. High tension.
    \includegraphics[width=\textwidth]{fig5b_moment_before.png}
    \caption*{(B)}
\end{subfigure}

\vspace{0.5cm}

\begin{subfigure}[b]{0.45\textwidth}
    \centering
    % SEARCH: A scene AFTER something happened - broken glass on floor, an empty
    % stage after performance, morning light on disheveled bed, footprints in
    % snow leading away. Traces of event without the event itself.
    \includegraphics[width=\textwidth]{fig5c_aftermath.png}
    \caption*{(C)}
\end{subfigure}
\hfill
\begin{subfigure}[b]{0.45\textwidth}
    \centering
    % SEARCH: An architectural rendering of a building never built, OR a photograph
    % of a place that no longer exists (pre-demolition), OR a painting of an
    % imagined place. The possible that stayed possible. The never-was.
    \includegraphics[width=\textwidth]{fig5d_never_was.png}
    \caption*{(D)}
\end{subfigure}
\caption{\textbf{The absent has presence.} You are not just seeing what is present in these images. In (A), you are seeing who should be there and isn't. In (B), you are experiencing the future that hasn't arrived yet---the completion of the arc is already in your experience as tension. In (C), you are reconstructing the event from its residue---the absent cause present through its effects. In (D), you are entertaining a possibility that was never actual. This is counterfactual weight: the processing resources your system allocates to non-actual possibilities. The present is never just the present. Your experience is saturated with what could be, was, might have been, will be. The absent has weight.}
\label{fig:absent-present}
\end{figure}

\subsection{Self-Model Salience}

\begin{definition}[Self-Model Salience]
\begin{equation}
\mathcal{SM}_t = \MI(\latent^{\text{self}}_t; \action_t) / \entropy(\action_t)
\end{equation}
The fraction of action entropy explained by the self-model component.
\end{definition}

Alternatively:
\begin{equation}
\mathcal{SM}_t = \frac{\text{dim}(\latent^{\text{self}})}{\text{dim}(\latent^{\text{total}})} \cdot \text{activity}(\latent^{\text{self}}_t)
\end{equation}

\begin{phenomenal}
\textbf{High self-salience}: Self-focused, self-conscious, self as primary object of attention. \\
\textbf{Low self-salience}: Self-forgotten, absorbed in environment or task.
\end{phenomenal}

%------------------------------------------------------------------------------
% FIGURE SET 4: SELF-MODELS MUST REASSERT COHERENCE
%------------------------------------------------------------------------------
\begin{figure}[htbp]
\centering
\begin{subfigure}[b]{0.45\textwidth}
    \centering
    % SEARCH: First-person POV photograph - hands visible at bottom of frame reaching
    % toward something, OR a threshold/doorway you're about to cross shot from
    % first-person perspective. The camera IS the viewer. You are IN the image.
    \includegraphics[width=\textwidth]{fig4a_first_person_pov.png}
    \caption*{(A)}
\end{subfigure}
\hfill
\begin{subfigure}[b]{0.45\textwidth}
    \centering
    % SEARCH: Portrait with direct intense eye contact. Not hostile, not welcoming -
    % neutral but piercing. Vermeer-style intimacy. Should feel like being seen,
    % being looked at. The subject is regarding YOU specifically.
    \includegraphics[width=\textwidth]{fig4b_direct_eye_contact.png}
    \caption*{(B)}
\end{subfigure}

\vspace{0.5cm}

\begin{subfigure}[b]{0.45\textwidth}
    \centering
    % SEARCH: A mirror in a scene positioned such that the viewer SHOULD be reflected
    % but isn't - OR showing empty space where the photographer should be. The absent
    % reflection. Where are you? You should be there but you're not.
    \includegraphics[width=\textwidth]{fig4c_mirror_empty.png}
    \caption*{(C)}
\end{subfigure}
\hfill
\begin{subfigure}[b]{0.45\textwidth}
    \centering
    % SEARCH: Voyeuristic framing - a scene witnessed through a window, or from around
    % a corner, peering through a gap. The composition implies you are watching unseen,
    % perhaps shouldn't be watching. The viewer as hidden observer/intruder.
    \includegraphics[width=\textwidth]{fig4d_voyeuristic_frame.png}
    \caption*{(D)}
\end{subfigure}
\caption{\textbf{Self-models must reassert coherence.} In (A), notice that the hands feel like your hands. The image places you inside it, implicates you in its space. In (B), notice the instant rise in self-awareness---someone is looking at you. You have become object in another's experience. Your self-model salience just spiked, whether you wanted it to or not. This is what $\mathcal{SM}$ means: the degree to which you appear as figure in your own experiential ground. In (C), notice the subtle wrongness---you should be in that reflection but you aren't. Where are you? The self-model seeks itself and fails. In (D), notice the peculiar quality of seeing-without-being-seen. Even your invisibility is a mode of self-awareness. The observer is never not in the observation. You cannot opt out of being someone.}
\label{fig:self-model-salience}
\end{figure}

%==============================================================================
\section{Affect Motifs}
%==============================================================================

We now characterize specific affects as structural motifs, invoking only the dimensions that define each. Before formalizing these structures, we ground each in its phenomenal character---the felt texture that any adequate theory must explain.

\textbf{Joy} \textit{expands}. It is \textit{light} before it is anything else---buoyant, effervescent, the body forgetting its weight. The world opens; possibilities \textit{multiply}; the \textit{self recedes} because it need not defend. Joy is surplus: more paths than required, more resources than consumed, \textit{slack} in every direction.

Where joy opens, \textbf{suffering} \textit{crushes}. It \textit{compresses} the world to a single unbearable point and makes that point more \textit{vivid} than anything has ever been. This is the paradox: suffering is hyper-real, more present than presence, more \textit{unified} than unity. You cannot look away. You cannot \textit{decompose} it. You are \textit{trapped} in a cage made of your own \textit{integration}.

\textbf{Fear} throws the self forward into \textit{futures} that threaten to annihilate it---cold, sharp, electric with \textit{anticipation}. The body readies before the mind has finished computing. Time dilates around the approaching harm. Fear is suffering that hasn't arrived yet, and the \textit{not-yet} is where we live.

We say \textbf{anger} is \textit{hot}, and we are not speaking metaphorically. Anger \textit{externalizes}: it \textit{simplifies} the world into self-versus-obstacle and energizes removal. Watch what happens to your model of the other person when you are angry---it \textit{flattens}, becomes a caricature, loses \textit{dimensionality}. Complexity collapses into opposition. This is why anger feels powerful and also stupid: you are burning \textit{integration} on a cartoon.

\textbf{Desire} \textit{funnels}. The world reorganizes around an \textit{attractor} not yet reached---magnetic, urgent, all-consuming. Everything becomes instrumental; the goal \textit{saturates} attention. Desire is joy's \textit{gradient}, pointing toward the basin but not yet in it. This is why anticipation often exceeds consummation: the structure of \textit{approach} is tighter than the structure of \textit{arrival}.

\textbf{Curiosity} \textit{reaches} outward---but unlike fear, it reaches toward \textit{promise} rather than threat. Pulling, open, playful. The \textit{uncertainty} that makes fear contract makes curiosity \textit{expand}. Same high counterfactual weight, opposite \textit{valence}. The difference is whether the \textit{branches} lead somewhere you want to go.

And \textbf{grief}? Grief \textit{persists}. Hollow, aching, curiously timeless. The lost object remains \textit{woven into} every prediction; every expectation that included them \textit{fails} silently, over and over. The world has changed. The \textit{model} has not caught up. Grief is the metabolic cost of love's \textit{integration}.

\vspace{0.5em}
\noindent What follows formalizes these textures as geometry.

\subsection{Joy}

\begin{definition}[Joy Motif]
Joy requires four dimensions:
\begin{itemize}
\item $\valence > 0$ (positive gradient on viability manifold)
\item $\intinfo$ high (unified, coherent experience)
\item $\effrank$ high (many degrees of freedom active---expansiveness)
\item $\mathcal{SM}$ low (self recedes; no need to defend)
\end{itemize}
Arousal varies (joy can be calm or excited). Counterfactual weight is incidental.
\end{definition}

\textbf{Structural interpretation}: The cause-effect structure has the shape of ``abundance''---multiple paths to good outcomes, redundancy, slack in the system. Many distinctions active simultaneously ($\effrank$ high), tightly coupled ($\intinfo$ high), but the self is light because the world is cooperating ($\mathcal{SM}$ low). This is why joy \emph{expands}: the geometry literally has more active dimensions.

\subsection{Suffering}

\begin{definition}[Suffering Motif]
Suffering requires three dimensions:
\begin{itemize}
\item $\valence < 0$ (negative gradient---approaching viability boundary)
\item $\intinfo$ high (hyper-unified, impossible to decompose or look away)
\item $\effrank$ low (collapsed into narrow subspace---trapped)
\end{itemize}
This is the core structural signature. Self-model salience is often high (the self as locus of the problem), but not necessarily---one can suffer while absorbed in external pain.
\end{definition}

\textbf{Structural interpretation}: High integration but collapsed into low-rank subspace. The system is deeply coupled but constrained to a dominant attractor it cannot escape.

\begin{keyresult}
The $\intinfo$-$\effrank$ dissociation is the key insight: suffering feels \textit{more real} than neutral states because it is actually more integrated. But it feels \textit{trapped} because the integration is constrained to a narrow manifold.

Formally: $\intinfo_{\text{suffering}} > \intinfo_{\text{neutral}}$ but $\effrank_{\text{suffering}} \ll \effrank_{\text{neutral}}$.

This is why you cannot simply ``think your way out'' of suffering---the very integration that makes it vivid also makes it inescapable.
\end{keyresult}

\subsection{Fear}

\begin{definition}[Fear Motif]
Fear is defined by three dimensions:
\begin{itemize}
\item $\valence < 0$ (anticipated negative gradient)
\item $\mathcal{CF}$ high, concentrated on threat trajectories (the not-yet dominates)
\item $\mathcal{SM}$ high (self foregrounded as the thing-that-might-be-harmed)
\end{itemize}
Arousal is typically high but not defining---cold fear exists. Integration and rank vary.
\end{definition}

\textbf{Structural interpretation}: Fear is suffering projected into the future. The temporal structure ($\mathcal{CF}$) is essential: fear lives in anticipation. The self-model must be salient because fear is fundamentally about threat \emph{to the self}. Remove the counterfactual weight (make it present-focused) and you get suffering. Remove the self-salience (make it about external objects) and you get something closer to aversion or disgust.

\subsection{Anger}

\begin{definition}[Anger Motif]
Anger requires valence, arousal, plus a feature not in the standard toolkit---\emph{other-model compression}:
\begin{itemize}
\item $\valence < 0$ (obstacle to viability)
\item $\arousal$ high (energized, mobilized for action)
\item $\text{dim}(\text{other-model}) \ll \text{dim}(\text{other-model})_{\text{normal}}$ (the other becomes a caricature)
\item Externalized causal attribution (the problem is \emph{out there})
\end{itemize}
\end{definition}

\textbf{Structural interpretation}: Anger simplifies. The other-model collapses into a low-dimensional obstacle-representation. Self-model may be complex, but the \emph{other} becomes flat, predictable, opposable. This is why anger feels powerful and stupid simultaneously: you're burning cognitive resources on a cartoon.

Note that other-model compression is not one of our standard dimensions---it emerges as essential for anger specifically. This illustrates the toolkit approach: we invoke whatever structural features do the work.

\subsection{Desire/Lust}

\begin{definition}[Desire Motif]
Desire is defined by anticipated valence, counterfactual weight, and a structural feature---\emph{goal-funneling}:
\begin{itemize}
\item $\valence > 0$ but projected forward (anticipated positive gradient)
\item $\mathcal{CF}$ high, concentrated on approach trajectories
\item Goal-funneling: many dimensions of experience converge toward narrow outcome space
\end{itemize}
Arousal is typically high but not definitional---one can desire calmly.
\end{definition}

\textbf{Structural interpretation}: Desire is the gradient of joy. The world reorganizes around an attractor not yet reached. Everything becomes instrumental; the goal saturates attention. The ``funneling'' structure---high-dimensional input collapsing toward low-dimensional goal---is what gives desire its characteristic urgency.

\begin{proposition}[Desire vs.\ Joy]
Joy is \textit{at} the attractor; desire is \textit{approaching} it. Structurally:
\begin{equation}
d(\state_{\text{joy}}, \mathcal{A}) \approx 0, \quad d(\state_{\text{desire}}, \mathcal{A}) > 0, \quad \frac{d}{dt}d(\state_{\text{desire}}, \mathcal{A}) < 0
\end{equation}
where $\mathcal{A}$ is the goal attractor. This explains why anticipation often exceeds consummation: the structure of \emph{approach} (funneling, convergent) is tighter than the structure of \emph{arrival} (expansive, slack).
\end{proposition}

\subsection{Curiosity}

\begin{definition}[Curiosity Motif]
Curiosity is essentially two-dimensional:
\begin{itemize}
\item $\valence > 0$ specifically toward uncertainty-reduction (anticipated information gain)
\item $\mathcal{CF}$ high with high entropy over counterfactual outcomes (many branches, not converged on one)
\item Uncertainty is \emph{welcomed}, not aversive
\end{itemize}
Self-model salience is typically low (absorbed in the object of curiosity).
\end{definition}

\textbf{Structural interpretation}: Curiosity and fear share high counterfactual weight---both live in the space of possibilities. The difference is valence orientation: fear's branches lead to threat, curiosity's branches lead to expanded affordances. Same temporal structure, opposite gradient direction.

\begin{proposition}[Curiosity as Intrinsic Motivation]
Curiosity $\approx$ positive valence attached to uncertainty-reduction. Formally:
\begin{equation}
r_{\text{curiosity}} \propto \MI(\obs_{t+1}; \latent | \text{new data}) - \MI(\obs_{t+1}; \latent | \text{old data})
\end{equation}
This is why curiosity feels \emph{pulling}: reducing uncertainty is rewarding.
\end{proposition}

\subsection{Grief}

\begin{definition}[Grief Motif]
Grief requires valence, past-directed counterfactual weight, and two structural features---\emph{persistent coupling to lost object} and \emph{unresolvable prediction error}:
\begin{itemize}
\item $\valence < 0$ (the world is worse than it was)
\item $\mathcal{CF}$ high but directed toward counterfactual \emph{past} (``if only...'')
\item $\MI(\selfmodel; \text{lost-object-model})$ remains high despite the object's absence
\item No action reduces the prediction error---the world has permanently changed
\end{itemize}
Arousal is variable (acute grief is high-arousal; chronic grief may be low).
\end{definition}

\textbf{Structural interpretation}: The lost attachment object remains woven into the self-model and world-model. Predictions involving the lost object continue to be generated and continue to fail. Grief is the metabolic cost of love's integration---the coupling that made the relationship meaningful is precisely what makes its absence painful. The model has not yet updated to the permanent change in the world.

This is why grief takes time: the self-model must be \emph{rewoven} around the absence, and that rewiring is slow.

\subsection{Summary: Defining Dimensions by Affect}

Rather than forcing all affects into a uniform grid, we summarize each by its defining structure:

\begin{table}[h]
\centering
\small
\begin{tabular}{lp{9cm}}
\toprule
\textbf{Affect} & \textbf{Constitutive Structure} \\
\midrule
Joy & $\valence{+}$, $\intinfo{\uparrow}$, $\effrank{\uparrow}$, $\mathcal{SM}{\downarrow}$ (positive, unified, expansive, self-light) \\
Suffering & $\valence{-}$, $\intinfo{\uparrow}$, $\effrank{\downarrow}$ (negative, hyper-integrated, collapsed) \\
Fear & $\valence{-}$, $\mathcal{CF}{\uparrow}$ (threat-focused), $\mathcal{SM}{\uparrow}$ (anticipatory self-threat) \\
Anger & $\valence{-}$, $\arousal{\uparrow}$, other-model compression (energized, externalized, simplified other) \\
Desire & $\valence{+}$ (anticipated), $\mathcal{CF}{\uparrow}$ (approach), goal-funneling (convergent anticipation) \\
Curiosity & $\valence{+}$ toward uncertainty, $\mathcal{CF}{\uparrow}$ with high branch entropy (welcomed unknown) \\
Grief & $\valence{-}$, $\mathcal{CF}{\uparrow}$ (past-directed), persistent coupling to absent object \\
Shame & $\valence{-}$, $\mathcal{SM}{\uparrow\uparrow}$, integration of negative self-evaluation (self as seen by other) \\
Boredom & $\arousal{\downarrow}$, $\intinfo{\downarrow}$, $\effrank{\downarrow}$ (understimulated, fragmented, collapsed) \\
Awe & $\intinfo$ expanding, $\effrank{\uparrow}$, $\mathcal{SM}{\downarrow}$ (self-dissolution through scale) \\
\bottomrule
\end{tabular}
\caption{Affects characterized by their defining structure. Arrows indicate direction (high/low); different affects require different dimensions.}
\end{table}

Note that different affects require different numbers of dimensions. Boredom is essentially three-dimensional (low arousal, low integration, low rank). Anger requires a structural feature (other-model compression) not in the standard toolkit. This is as it should be: we invoke the geometry that does the work.

%------------------------------------------------------------------------------
% FIGURE SET 6: THE SUBLIME (EFFECTIVE RANK OVERFLOW)
%------------------------------------------------------------------------------
\begin{figure}[htbp]
\centering
\begin{subfigure}[b]{0.45\textwidth}
    \centering
    % SEARCH: Hubble Deep Field or equivalent - photograph showing THOUSANDS of galaxies
    % in a patch of sky the size of a grain of sand at arm's length. The sheer
    % numerosity should be incomprehensible. Cosmic scale that exceeds processing.
    \includegraphics[width=\textwidth]{fig6a_cosmic_scale.png}
    \caption*{(A)}
\end{subfigure}
\hfill
\begin{subfigure}[b]{0.45\textwidth}
    \centering
    % SEARCH: Natural fractal - Romanesco broccoli in extreme close-up, OR coastline
    % at scale where detail never stops, OR river delta from satellite. Self-similarity
    % that implies infinite depth. The structure goes down forever.
    \includegraphics[width=\textwidth]{fig6b_infinite_regress.png}
    \caption*{(B)}
\end{subfigure}

\vspace{0.5cm}

\begin{subfigure}[b]{0.45\textwidth}
    \centering
    % SEARCH: Caspar David Friedrich "Wanderer Above the Sea of Fog" OR photograph
    % of tiny human figure against vast landscape - mountains, ocean, canyon, glacier.
    % The human should be PRESENT but clearly insignificant against the scale.
    \includegraphics[width=\textwidth]{fig6c_human_dwarfed.png}
    \caption*{(C)}
\end{subfigure}
\hfill
\begin{subfigure}[b]{0.45\textwidth}
    \centering
    % SEARCH: Abstract expressionism that resists stable interpretation - Jackson
    % Pollock drip painting OR Rothko color field OR de Kooning. Something that
    % CANNOT be "figured out" but demands attention anyway. Processing without resolution.
    \includegraphics[width=\textwidth]{fig6d_uninterpretable.png}
    \caption*{(D)}
\end{subfigure}
\caption{\textbf{The sublime is effective rank overflow.} Notice the slight destabilization in viewing (A). This is not aesthetic appreciation. Your processing architecture is encountering more structure than it was designed to hold. The effective rank of the stimulus exceeds your representational capacity. In (B), notice the vertigo of infinite depth---structure that goes down forever. In (C), notice how your self-model shrinks. You are not merely observing a small figure; you are becoming small. Your self-model is being resized relative to the world model. This is not metaphor. In (D), notice the restlessness of refused interpretation---your system keeps trying to compress, and the image keeps resisting. The sublime is what this overflow feels like from inside. Too many dimensions active at once. The feeling of awe is not your opinion about the scale. The feeling IS the scale exceeding your capacity.}
\label{fig:sublime-overflow}
\end{figure}

%------------------------------------------------------------------------------
% FIGURE SET 7: SUFFERING AND JOY AS STRUCTURE
%------------------------------------------------------------------------------
\begin{figure}[htbp]
\centering
\begin{subfigure}[b]{0.45\textwidth}
    \centering
    % SEARCH: Genuinely expansive joy image - children running in open field,
    % a burst of confetti or birds taking flight, an open sky with dramatic clouds.
    % Should feel like abundance, multiple paths, surplus, EXPANSION.
    \includegraphics[width=\textwidth]{fig7a_joy_structure.png}
    \caption*{(A)}
\end{subfigure}
\hfill
\begin{subfigure}[b]{0.45\textwidth}
    \centering
    % SEARCH: Constriction and entrapment - close-up of hands gripping cage bars
    % (NOT prison context, just the hands), OR a figure in too-small space, OR
    % a corner with no exit. The TRAPPED quality. High negative valence through form.
    \includegraphics[width=\textwidth]{fig7b_suffering_structure.png}
    \caption*{(B)}
\end{subfigure}

\vspace{0.5cm}

\begin{subfigure}[b]{0.45\textwidth}
    \centering
    % SEARCH: Grief structure - a child's room with toys but no child, OR an old
    % photograph of someone smiling (knowing they're now gone), OR a wedding ring
    % on a nightstand. Presence saturated with absence. The missing person.
    \includegraphics[width=\textwidth]{fig7c_grief_structure.png}
    \caption*{(C)}
\end{subfigure}
\hfill
\begin{subfigure}[b]{0.45\textwidth}
    \centering
    % SEARCH: Perfect stillness and coherence - a Japanese zen garden, OR a perfectly
    % composed minimalist interior, OR still water reflecting sky. High unity,
    % neutral valence. CALM. Integration without turbulence.
    \includegraphics[width=\textwidth]{fig7d_neutral_integration.png}
    \caption*{(D)}
\end{subfigure}
\caption{\textbf{Suffering and joy are structural.} Compare your response to (A) and (B). In (A), notice the expansiveness---many degrees of freedom feel active, many paths available. This is what high effective rank feels like. In (B), notice the constriction---few dimensions, no exit, the geometry of being trapped. This is suffering as structure: high integration (you cannot look away), low effective rank (collapsed into narrow subspace), negative valence (approaching viability boundary). The structure IS the feeling. In (C), notice how integration binds you to the absence---you cannot think about the image without thinking about who is missing. This is grief: persistent coupling to what cannot be present. In (D), notice the neutral stillness---high integration with neither expansion nor contraction. This is possible too: coherence without valence pressure.}
\label{fig:suffering-joy-structure}
\end{figure}

\begin{todo_empirical}
\textbf{Quantifying the affect table}: The qualitative descriptors (high, med, low) require empirical calibration:

\textbf{Study 1: Affect induction with neural recording}
\begin{itemize}
\item Induce target affects via validated protocols (film clips, autobiographical recall, IAPS images)
\item Measure integration proxies (transfer entropy density, Lempel-Ziv complexity) from EEG/MEG
\item Measure effective rank from neural state covariance
\item Compare self-report (PANAS, SAM) with structural measures
\end{itemize}

\textbf{Study 2: Real-time affect tracking}
\begin{itemize}
\item Continuous self-report (dial/slider) during naturalistic experience
\item Correlate with physiological proxies (HRV for arousal, pupil for $\mathcal{CF}$, skin conductance)
\item Develop regression model: self-report $\sim f(\text{structural measures})$
\end{itemize}

\textbf{Study 3: Cross-modal validation}
\begin{itemize}
\item Compare fMRI (spatial resolution) with MEG (temporal resolution)
\item Validate effective rank measure across modalities
\item Test whether integration predicts subjective intensity
\end{itemize}

\textbf{Target outputs}: Numerical ranges for each cell, confidence intervals, individual difference parameters.
\end{todo_empirical}

%==============================================================================
\section{Dynamics and Transitions}
%==============================================================================

\subsection{Affect Trajectories}

Affects are not static points but dynamic trajectories through affect space. The evolution can be written:
\begin{equation}
\frac{d\mathbf{a}}{dt} = F(\mathbf{a}, \obs, \action, \text{context}) + \bm{\eta}
\end{equation}
where $\mathbf{a} = (\valence, \arousal, \intinfo, \effrank, \mathcal{CF}, \mathcal{SM})$.

\begin{proposition}[Affect Blending]
Adjacent affects in the space blend into each other continuously:
\begin{itemize}
\item Fear $\to$ Anger as causal attribution externalizes
\item Desire $\to$ Joy as goal distance $\to 0$
\item Suffering $\to$ Curiosity as valence flips while $\mathcal{CF}$ remains high
\item Grief $\to$ Nostalgia as arousal decreases and $\mathcal{CF}_{\text{approach}}$ replaces $\mathcal{CF}_{\text{avoidance}}$
\end{itemize}
\end{proposition}

\subsection{Attractor Dynamics}

Some affect regions are attractors; the system tends to stay in them once entered. Others are transient.

\begin{definition}[Affect Attractor]
An affect region $\mathcal{R} \subset \mathcal{A}$ is an \emph{attractor} if:
\begin{equation}
\prob(\mathbf{a}_{t+\tau} \in \mathcal{R} | \mathbf{a}_t \in \mathcal{R}) > \prob(\mathbf{a}_{t+\tau} \in \mathcal{R} | \mathbf{a}_t \notin \mathcal{R})
\end{equation}
for some characteristic time $\tau$.
\end{definition}

\begin{conjecture}[Pathological Attractors]
Depression, addiction, and chronic anxiety are characterized by pathologically stable attractors in affect space:
\begin{itemize}
\item \textbf{Depression}: Attractor at (low $\valence$, low $\arousal$, high $\intinfo$, low $\effrank$, low $\mathcal{CF}$, high $\mathcal{SM}$)
\item \textbf{Addiction}: Attractor at (high $\valence$ conditional on substance, collapsing $\effrank$ in goal space)
\item \textbf{Anxiety}: Diffuse attractor with (low $\valence$, high $\arousal$, high $\mathcal{CF}$ spread across many threats)
\end{itemize}
\end{conjecture}

%==============================================================================
\section{Novel Predictions}
%==============================================================================

\subsection{Unexplained Phenomena}

The framework predicts the existence of phenomenal states that may be rare or difficult to report on.

\begin{conjecture}[High Rank, Low Integration]
States with many active degrees of freedom ($\effrank$ high) but poor coupling ($\intinfo$ low) should feel like fragmentation, multiplicity, ``everything happening but nothing cohering.''

\textbf{Where to look}: Certain psychedelic states before reintegration; dissociative transitions; information overload.
\end{conjecture}

\begin{conjecture}[Negative Valence, High Rank, Low Arousal]
This combination predicts a state of ``expansive despair''---calm hopelessness with full awareness of possibilities, all of which are negative.

\textbf{Where to look}: Late-stage depression; existential nihilism; certain contemplative ``dark night'' states.
\end{conjecture}

\begin{conjecture}[Rank Exhaustion]
Maintaining high $\effrank$ should be metabolically expensive. Prolonged high-rank states should lead to specific fatigue distinct from physical tiredness.

\textbf{Where to look}: Post-psychedelic fatigue; meditation retreat collapse (days 3-5); therapist burnout.
\end{conjecture}

\begin{conjecture}[Integration Debt]
Suppressing integration (compartmentalizing, dissociating) should accumulate ``pressure'' for reintegration. When defenses fail, the flood should exceed what the original stimulus would warrant.

\textbf{Prediction}: Intensity of breakthrough $\propto$ duration $\times$ degree of prior suppression.
\end{conjecture}

\subsection{Quantitative Predictions}

\begin{proposition}[Clustering Prediction]
In controlled affect induction paradigms, affects should cluster by their defining dimensions:
\begin{enumerate}
\item Joy conditions cluster in the $(+\valence, +\effrank, +\intinfo, -\mathcal{SM})$ region
\item Suffering conditions cluster in the $(-\valence, +\intinfo, -\effrank)$ region
\item Fear and curiosity both show high $\mathcal{CF}$ but separate on valence axis
\end{enumerate}
\textbf{Falsification criterion}: If affects don't cluster by their predicted dimensions---or if other dimensions predict clustering better---the motif characterizations require revision.
\end{proposition}

%==============================================================================
\section{Operational Measurement}
%==============================================================================

\subsection{In Silico Protocol}

For artificial agents (world-model RL agents):

\begin{algorithm}
\caption{Affect Measurement in World-Model Agents}
\begin{algorithmic}[1]
\State \textbf{Agent}: Recurrent latent world model (RSSM, Transformer, etc.)
\State \textbf{Record}: $\latent_t$, policy logits, value estimates, rollout trees, uncertainty
\For{each timestep $t$}
    \State $\valence_t \gets \E[A^\policy(\latent_t, \action_t)]$
    \State $\arousal_t \gets \KL(\latent_{t+1} \| \latent_t)$
    \State $\intinfo_t \gets \Delta_P$ (prediction loss under partition)
    \State $\effrank_t \gets (\tr C_t)^2 / \tr(C_t^2)$
    \State $\mathcal{CF}_t \gets$ rollout compute fraction
    \State $\mathcal{SM}_t \gets \MI(\latent^{\text{self}}_t; \action_t) / \entropy(\action_t)$
\EndFor
\State \textbf{Output}: Time series $\{(\valence_t, \arousal_t, \intinfo_t, \effrank_t, \mathcal{CF}_t, \mathcal{SM}_t)\}$
\end{algorithmic}
\end{algorithm}

\subsection{Biological Protocol}

For neural recordings (MEG/EEG/fMRI):

\begin{itemize}
\item $\intinfo$: Directed influence density (transfer entropy), synergy measures
\item $\effrank$: Participation ratio of neural state covariance
\item $\arousal$: Entropy rate, broadband power shifts, peripheral correlates (pupil, HRV)
\item $\valence$: Approach/avoid behavioral bias, reward prediction error correlates
\item $\mathcal{CF}$: Prefrontal/default mode engagement patterns
\item $\mathcal{SM}$: Self-referential network activation
\end{itemize}

%==============================================================================
\section{Summary of Part II}
%==============================================================================

\begin{enumerate}
\item \textbf{Hard problem dissolved}: By rejecting the privileged base layer, we remove the demand for reduction. Experience is real at the experiential scale, just as chemistry is real at the chemical scale.

\item \textbf{Identity thesis}: Experience \textit{is} intrinsic cause-effect structure. This is an identity claim, not a correlation.

\item \textbf{Geometric phenomenology}: Different affects correspond to different structural motifs. Rather than forcing all affects into a fixed grid, we identify the defining dimensions for each---the features without which that affect would not be that affect.

\item \textbf{Variable dimensionality}: Joy requires four dimensions (valence, integration, rank, self-salience). Suffering requires three (valence, integration, rank). Anger requires a feature (other-model compression) not in the standard toolkit. We invoke what does the work.

\item \textbf{Suffering explained}: High integration + low rank = intense but trapped. This is the core structural insight---why suffering feels more real than neutral states yet also inescapable.

\item \textbf{Operational measures}: We provide protocols for measuring structural features in both artificial and biological systems, with the understanding that not all measures are relevant to all phenomena.
\end{enumerate}

Part III will examine how human cultural forms---aesthetics, sexuality, ideology, science, religion---serve as technologies for managing the existential burden of inescapable selfhood.

Part IV will develop:
\begin{itemize}
\item The grounding of normativity in viability structure
\item Scale-matched interventions from neurons to nations
\item Gods as agentic systems with viability manifolds
\item Implications for AI systems and alignment
\end{itemize}

Part V will address the transcendence of the self: the historical rise of consciousness, the AI frontier, and how to surf rather than be submerged by the coming wave.

\end{document}
