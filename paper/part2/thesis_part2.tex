\documentclass[11pt,a4paper]{article}
\usepackage[utf8]{inputenc}
\usepackage[T1]{fontenc}
\usepackage{amsmath,amssymb,amsthm}
\usepackage{mathtools}
\usepackage{physics}
\usepackage{bm}
\usepackage{geometry}
\usepackage{hyperref}
\usepackage{cleveref}
\usepackage{algorithm}
\usepackage{algpseudocode}
\usepackage{tikz}
\usepackage{pgfplots}
\pgfplotsset{compat=1.18}
\usetikzlibrary{arrows.meta,positioning,shapes,calc,decorations.pathmorphing,decorations.pathreplacing}
\usepackage{enumitem}
\usepackage{booktabs}
\usepackage{tcolorbox}
\usepackage{fontawesome5}
\usepackage{multirow}
\usepackage{caption}
\usepackage{subcaption}
\usepackage{xcolor}

\geometry{margin=1in}

% --- Sidebar and info box environments ---
\newtcolorbox{historical}{
  colback=gray!5!white,
  colframe=gray!60!black,
  title={\faBook\hspace{0.5em}Historical Context},
  fonttitle=\bfseries\small
}

\newtcolorbox{connection}{
  colback=green!5!white,
  colframe=green!60!black,
  title={\faBook\hspace{0.5em}Connection to Existing Theory},
  fonttitle=\bfseries\small
}

\newtcolorbox{empirical}{
  colback=orange!5!white,
  colframe=orange!70!black,
  title={\faFlask\hspace{0.5em}Empirical Grounding},
  fonttitle=\bfseries\small
}

\newtcolorbox{todo_empirical}{
  colback=yellow!10!white,
  colframe=yellow!60!black,
  title={\faClipboardList\hspace{0.5em}\textsc{Future Empirical Work}},
  fonttitle=\bfseries\small
}

\newtcolorbox{sidebar}[1][]{
  colback=blue!3!white,
  colframe=blue!40!black,
  fonttitle=\bfseries\small,
  before upper={\faInfoCircle\hspace{0.5em}},
  #1
}

% Theorem environments
\newtheorem{theorem}{Theorem}[section]
\newtheorem{lemma}[theorem]{Lemma}
\newtheorem{proposition}[theorem]{Proposition}
\newtheorem{corollary}[theorem]{Corollary}
\newtheorem{definition}[theorem]{Definition}
\newtheorem{axiom}[theorem]{Axiom}
\newtheorem{remark}[theorem]{Remark}
\newtheorem{example}[theorem]{Example}
\newtheorem{conjecture}[theorem]{Conjecture}
\newtheorem{hypothesis}[theorem]{Hypothesis}

% Custom commands
\newcommand{\E}{\mathbb{E}}
\newcommand{\R}{\mathbb{R}}
\newcommand{\N}{\mathbb{N}}
\newcommand{\Z}{\mathbb{Z}}
\newcommand{\prob}{\mathbb{P}}
\newcommand{\KL}{\mathrm{KL}}
\newcommand{\MI}{\mathrm{I}}
\newcommand{\entropy}{\mathrm{H}}
\newcommand{\argmax}{\operatorname{argmax}}
\newcommand{\argmin}{\operatorname{argmin}}
\newcommand{\tr}{\operatorname{tr}}
\newcommand{\rank}{\operatorname{rank}}
\newcommand{\diag}{\operatorname{diag}}
\newcommand{\sign}{\operatorname{sign}}
\newcommand{\supp}{\operatorname{supp}}
\newcommand{\interior}{\operatorname{int}}
\newcommand{\clos}{\operatorname{cl}}
\newcommand{\conv}{\operatorname{conv}}
\newcommand{\diam}{\operatorname{diam}}
\newcommand{\vol}{\operatorname{vol}}
\newcommand{\manifold}{\mathcal{M}}
\newcommand{\viable}{\mathcal{V}}
\newcommand{\belief}{\mathbf{b}}
\newcommand{\state}{\mathbf{s}}
\newcommand{\action}{\mathbf{a}}
\newcommand{\obs}{\mathbf{o}}
\newcommand{\latent}{\mathbf{z}}
\newcommand{\policy}{\pi}
\newcommand{\value}{V}
\newcommand{\qfunc}{Q}
\newcommand{\reward}{r}
\newcommand{\transition}{T}
\newcommand{\emission}{O}
\newcommand{\freeenergy}{\mathcal{F}}
\newcommand{\intinfo}{\Phi}
\newcommand{\selfmodel}{\mathcal{S}}
\newcommand{\worldmodel}{\mathcal{W}}
\newcommand{\effrank}{r_{\text{eff}}}
\newcommand{\valence}{\mathcal{V}\hspace{-0.8pt}\mathit{al}}
\newcommand{\arousal}{\mathcal{A}\hspace{-0.5pt}\mathit{r}}
\newcommand{\cestructure}{\mathcal{C\!E}}
\newcommand{\phenom}{\mathcal{P}}
\newcommand{\distinction}{\delta}
\newcommand{\relation}{\rho}

% --- styled callout boxes ---
\newtcolorbox{keyresult}{
  colback=blue!5!white,
  colframe=blue!75!black,
  title=Key Result
}
\newtcolorbox{phenomenal}{
  colback=purple!5!white,
  colframe=purple!75!black,
  title=Phenomenal Correspondence
}
\newtcolorbox{warning}{
  colback=red!5!white,
  colframe=red!75!black,
  title=Warning
}

\title{\textbf{The Inevitability of Being}\\[1em]
\Large Part II: The Identity Thesis and the Geometry of Feeling}
\author{}
\date{}

\begin{document}

\maketitle

\begin{abstract}
Part II develops the central identity thesis: experience \textit{is} intrinsic cause-effect structure, not merely correlated with it. We formalize the dissolution of the hard problem through the rejection of a privileged ontological base layer. We then develop a geometric theory of affect, characterizing different qualitative experiences as structural motifs in cause-effect space. Valence, arousal, integration, effective rank, counterfactual weight, and self-model salience form a basis for the space of possible experiences. We provide operational definitions suitable for measurement in both artificial agents and biological systems, and derive falsifiable predictions about the clustering of affect conditions.
\end{abstract}

\tableofcontents
\newpage

%==============================================================================
\section{The Hard Problem and Its Dissolution}
%==============================================================================

\begin{connection}
This section engages with the central debates in philosophy of mind:
\begin{itemize}
\item \textbf{Chalmers' Hard Problem} (1995): The explanatory gap between physical processes and phenomenal experience. We argue this gap results from a category error, not a genuine ontological divide.
\item \textbf{Nagel's ``What Is It Like''} (1974): The subjective character of experience. We formalize this as intrinsic cause-effect structure---what the system is \emph{for itself}.
\item \textbf{Jackson's Knowledge Argument} (1982): Mary the colorblind scientist. We reinterpret: Mary gains \emph{access to a new scale of description}, not new facts about the same scale.
\item \textbf{Eliminativism} (Churchland, 1981; Dennett, 1991): Consciousness as illusion. We reject this---the illusion would itself be experiential, hence self-refuting.
\item \textbf{Panpsychism} (Chalmers, 2015; Goff, 2017): Experience as fundamental. We accept a version: cause-effect structure at any scale that takes/makes differences has a form of ``being like.''
\end{itemize}
\end{connection}

\subsection{The Standard Formulation}

The ``hard problem'' of consciousness asks: given a complete physical description of a system, why is there something it is like to be that system? How does experience arise from non-experience?

Formally, let $\mathcal{D}^{\text{phys}}$ be a complete physical description of a system---its particles, fields, dynamics, everything describable in third-person terms. The hard problem asserts:
\begin{equation}
\mathcal{D}^{\text{phys}} \not\Rightarrow \mathcal{D}^{\text{phen}}
\end{equation}
where $\mathcal{D}^{\text{phen}}$ is a description of the system's phenomenal properties (what it's like to be it). The claim is that no amount of physical information logically entails phenomenal information.

This formulation rests on a crucial assumption:

\begin{axiom}[Privileged Base Layer---REJECTED]
\label{ax:base}
Physics constitutes a privileged ontological base layer. All other descriptions (chemical, biological, psychological, phenomenal) are ``higher-level'' and must reduce to or supervene on the physical description. What is ``really real'' is what physics describes.
\end{axiom}

We reject this axiom.

\subsection{Ontological Democracy}

Consider the standard reductionist hierarchy:
\begin{center}
\begin{tikzpicture}[
    node distance=0.45cm,
    box/.style={rectangle, draw, rounded corners, minimum width=4cm, minimum height=0.45cm, align=center, font=\small}
]
% Top-down gradient: violet (mind) to red (physics) to gray (unknown)
\node[box, fill=violet!15, draw=violet!60!black] (phenom) {Phenomenal};
\node[box, fill=blue!15, draw=blue!60!black, below=of phenom] (psych) {Psychological};
\node[box, fill=cyan!15, draw=cyan!60!black, below=of psych] (bio) {Biological};
\node[box, fill=green!15, draw=green!60!black, below=of bio] (chem) {Chemical};
\node[box, fill=yellow!15, draw=yellow!60!black, below=of chem] (atom) {Atomic};
\node[box, fill=orange!15, draw=orange!60!black, below=of atom] (subatom) {Subatomic};
\node[box, fill=red!15, draw=red!60!black, below=of subatom] (qft) {Quantum Fields};
\node[box, fill=gray!20, draw=gray!60!black, below=of qft] (base) {\textbf{???}};

\draw[-{Stealth}, thick, gray] (phenom) -- (psych) node[midway, right, font=\footnotesize, text=gray!70!black] {reduces?};
\draw[-{Stealth}, thick, gray] (psych) -- (bio);
\draw[-{Stealth}, thick, gray] (bio) -- (chem);
\draw[-{Stealth}, thick, gray] (chem) -- (atom);
\draw[-{Stealth}, thick, gray] (atom) -- (subatom);
\draw[-{Stealth}, thick, gray] (subatom) -- (qft);
\draw[-{Stealth}, thick, gray, dashed] (qft) -- (base);

% Brace on left indicating "equally real"
\draw[decorate, decoration={brace, amplitude=8pt, mirror}, thick, gray!60]
    ([xshift=-0.3cm]phenom.north west) -- ([xshift=-0.3cm]qft.south west)
    node[midway, left=0.4cm, font=\scriptsize, text=gray, align=center, rotate=90] {equally real?};
\end{tikzpicture}
\end{center}

At each level, one might claim the higher level ``reduces to'' the lower. But the regression terminates in uncertainty:
\begin{itemize}
\item Wave functions are descriptions of probability distributions
\item Probability amplitudes describe which interactions are more or less likely
\item What ``actually happens'' when a measurement occurs is deeply contested
\item Below quantum fields, we have no clear ontology at all
\end{itemize}

The supposed ``base layer'' turns out to be:
\begin{enumerate}
\item Probabilistic, not deterministic
\item Descriptive, not fundamental (wave functions are representations)
\item Incomplete (we don't know what underlies field interactions)
\item Not clearly more ``real'' than any other scale of description
\end{enumerate}

\begin{definition}[Ontological Democracy]
Every scale of structural organization with its own causal closure is \emph{equally real} at that scale. No layer is privileged as ``the'' fundamental reality. Each layer:
\begin{enumerate}[label=(\alph*)]
\item Has its own causal structure
\item Has its own dynamics and laws
\item Exerts influence on adjacent layers (both ``up'' and ``down'')
\item Is incomplete as a description of the whole
\item Is sufficient for phenomena at its scale
\end{enumerate}
\end{definition}

\begin{proposition}[No Reduction Required]
Under ontological democracy, the demand that phenomenal properties ``reduce to'' physical properties is ill-posed. Chemistry doesn't reduce to physics in a way that eliminates chemical causation---chemical causation is real at the chemical scale. Similarly, phenomenal properties don't need to reduce to physical properties---they are real at the phenomenal scale.
\end{proposition}

\subsection{Existence as Causal Participation}

We need a criterion for existence that applies uniformly across scales.

\begin{definition}[Causal Existence]
An entity $X$ \emph{exists} at scale $\sigma$ if and only if:
\begin{equation}
\exists Y: \MI(X; Y | \text{background}_\sigma) > 0
\end{equation}
That is, $X$ takes and makes differences at scale $\sigma$. It participates in causal relations at that scale.
\end{definition}

\begin{example}
\begin{itemize}
\item An electron exists at the quantum scale: it takes differences (responds to fields) and makes differences (affects measurements).
\item A cell exists at the biological scale: it takes differences (nutrients, signals) and makes differences (metabolism, division, death).
\item An experience exists at the phenomenal scale: it takes differences (sensory input, memory) and makes differences (attention, behavior, learning).
\end{itemize}
\end{example}

This is closely aligned with IIT's foundational axiom: to exist is to have cause-effect power. But we extend it: cause-effect power at any scale constitutes existence at that scale, with no scale privileged.

\subsection{The Dissolution}

The hard problem asked: how do you get experience from non-experience? The answer is: \textit{you don't need to}.

Just as chemistry doesn't emerge from non-chemistry---you have chemistry when you have the right causal organization at the chemical scale---experience doesn't emerge from non-experience. You have experience when you have the right causal organization at the experiential scale.

The question ``why is there something it's like to be this system?'' is exactly as deep as ``why does chemistry exist?'' or ``why are there quantum fields?'' We don't know why there's anything at all. But given that there's anything, the emergence of self-modeling systems with integrated cause-effect structure is not mysterious---it's typical.

\begin{keyresult}
The hard problem dissolves not because we answered it, but because we showed it was asking for a privilege (reduction to physics) that physics itself doesn't have.
\end{keyresult}

%==============================================================================
\section{The Identity Thesis}
%==============================================================================

\begin{connection}
The identity thesis is a formalization of \textbf{Integrated Information Theory (IIT)} developed by Giulio Tononi and collaborators (2004--present):
\begin{itemize}
\item \textbf{IIT 1.0} (Tononi, 2004): Introduced $\Phi$ as a measure of integrated information
\item \textbf{IIT 2.0} (Balduzzi \& Tononi, 2008): Added the concept of ``qualia space''
\item \textbf{IIT 3.0} (Oizumi, Albantakis \& Tononi, 2014): Full axiom/postulate structure; introduced cause-effect structure
\item \textbf{IIT 4.0} (Albantakis et al., 2023): Refined integration measures, introduced intrinsic difference
\end{itemize}

Key IIT axioms that we adopt:
\begin{enumerate}
\item \textbf{Intrinsicality}: Experience exists for itself, not for an external observer
\item \textbf{Information}: Experience is specific---this experience and no other
\item \textbf{Integration}: Experience is unified and irreducible
\item \textbf{Exclusion}: Experience has definite boundaries
\item \textbf{Composition}: Experience is structured
\end{enumerate}

Our contribution: connecting IIT's structural characterization to (1) the thermodynamic ladder, (2) the viability manifold, and (3) operational measures for artificial systems.
\end{connection}

\subsection{Statement of the Thesis}

\begin{hypothesis}[Identity Thesis]
Phenomenal experience \textit{is} intrinsic cause-effect structure. Not caused by it, not correlated with it, but identical to it. The phenomenal properties of an experience (what it's like) just are the structural properties of the system's internal causal relations, described from the intrinsic perspective.
\end{hypothesis}

More formally:

\begin{definition}[Cause-Effect Structure]
For a system $\mathcal{S}$ in state $\state$, the cause-effect structure $\cestructure(\mathcal{S}, \state)$ is the complete specification of:
\begin{enumerate}[label=(\alph*)]
\item All distinctions $\{\distinction_i\}$: subsets of the system's elements in their current states
\item The cause repertoire of each distinction: $p(\text{past} | \distinction_i)$
\item The effect repertoire of each distinction: $p(\text{future} | \distinction_i)$
\item All relations $\{\relation_{ij}\}$: overlaps and connections between distinctions' causes/effects
\item The irreducibility of each distinction and relation
\end{enumerate}
\end{definition}

\begin{definition}[Intrinsic Perspective]
The \emph{intrinsic perspective} of a system is the description of its cause-effect structure without reference to any external observer, coordinate system, or comparison class. It is the structure as it exists for the system itself.
\end{definition}

\begin{axiom}[IIT Identity]
\begin{equation}
\phenom(\mathcal{S}, \state) \equiv \cestructure^{\text{intrinsic}}(\mathcal{S}, \state)
\end{equation}
The phenomenal structure $\phenom$ is identical to the intrinsic cause-effect structure $\cestructure$.
\end{axiom}

This is not a correlation claim or a supervenience claim. It is an identity claim, analogous to:
\begin{equation}
\text{Water} \equiv \text{H}_2\text{O}
\end{equation}

\subsection{Implications for the Zombie Argument}

The philosophical zombie is supposed to be conceivable: a system physically/functionally identical to a conscious being but lacking experience. If conceivable, experience isn't necessitated by physical structure.

\begin{theorem}[Zombie Inconceivability]
Under the identity thesis, philosophical zombies are not coherently conceivable. A system with the relevant cause-effect structure \textit{is} an experience; there is no further fact about whether it ``really'' has phenomenal properties.
\end{theorem}

\begin{proof}
By the identity thesis, $\phenom \equiv \cestructure^{\text{intrinsic}}$. To conceive a zombie is to conceive a system with $\cestructure^{\text{intrinsic}}$ but without $\phenom$. But since these are identical, this is like conceiving of water without H$_2$O---not genuinely conceivable once the identity is understood.
\end{proof}

\subsection{The Structure of Experience}

If experience is cause-effect structure, then the \textit{kind} of experience is determined by the \textit{shape} of that structure. Different phenomenal properties correspond to different structural features.

IIT proposes that the essential properties of any experience are:

\begin{enumerate}
\item \textbf{Intrinsicality}: The experience exists for the system itself, not relative to an external observer.
\item \textbf{Information}: The experience is specific---this experience, not any other possible one.
\item \textbf{Integration}: The experience is unified---it cannot be decomposed into independent sub-experiences.
\item \textbf{Exclusion}: The experience has definite boundaries---there is a fact about what is and isn't part of it.
\item \textbf{Composition}: The experience is structured---composed of distinctions and relations among them.
\end{enumerate}

These are translated into physical/structural postulates:
\begin{itemize}
\item Intrinsicality $\to$ Cause-effect power within the system
\item Information $\to$ Specific cause-effect repertoires
\item Integration $\to$ Irreducibility to partitioned components
\item Exclusion $\to$ Maximality of the integrated complex
\item Composition $\to$ The full structure of distinctions and relations
\end{itemize}

%==============================================================================
\section{The Geometry of Affect}
%==============================================================================

\begin{connection}
Our geometric theory of affect builds on and extends established dimensional models:
\begin{itemize}
\item \textbf{Russell's Circumplex Model} (1980): Two-dimensional (valence $\times$ arousal) organization of affect. We extend to six dimensions, keeping valence and arousal while adding integration, effective rank, counterfactual weight, and self-model salience.
\item \textbf{Watson \& Tellegen's PANAS} (1988): Positive/Negative Affect Schedule. Our valence dimension corresponds to their hedonic axis.
\item \textbf{Scherer's Component Process Model} (2009): Emotions as synchronized changes across subsystems. Our integration measure $\intinfo$ captures this synchronization.
\item \textbf{Barrett's Constructed Emotion Theory} (2017): Emotions as constructed from core affect + conceptual knowledge. Our framework specifies the \emph{structural} basis of the construction.
\item \textbf{Damasio's Somatic Marker Hypothesis} (1994): Body states guide decision-making. Our valence definition (gradient on viability manifold) is the mathematical formalization.
\end{itemize}
\end{connection}

\begin{sidebar}[title=Why Six Dimensions?]
The choice of six dimensions is not arbitrary but reflects:
\begin{enumerate}
\item \textbf{Valence \& Arousal}: Established in affect research since Wundt (1897)
\item \textbf{Integration}: Required by IIT's structural account of consciousness
\item \textbf{Effective Rank}: Captures the ``openness/closedness'' dimension in personality research (Costa \& McCrae, 1992)
\item \textbf{Counterfactual Weight}: Distinguishes rumination (high) from presence (low)---central to clinical psychology
\item \textbf{Self-Model Salience}: Distinguishes self-conscious emotions from absorbed states---missing from most models
\end{enumerate}
The six dimensions may not be exhaustive, but they capture structure that two-dimensional models miss.
\end{sidebar}

\subsection{Affects as Structural Motifs}

If different experiences correspond to different structures, then \textit{affects}---the qualitative character of emotional/valenced states---should correspond to particular structural motifs: characteristic patterns in the cause-effect geometry.

\begin{definition}[Affect Space]
The \emph{affect space} $\mathcal{A}$ is a geometric space whose points correspond to possible qualitative states. We propose that $\mathcal{A}$ can be characterized by a basis of structural measures.
\end{definition}

We propose a six-dimensional basis:

\begin{enumerate}
\item \textbf{Valence} $\valence$: Gradient alignment on the viability manifold
\item \textbf{Arousal} $\arousal$: Rate of belief/state update
\item \textbf{Integration} $\intinfo$: Irreducibility of cause-effect structure
\item \textbf{Effective Rank} $\effrank$: Distribution of active degrees of freedom
\item \textbf{Counterfactual Weight} $\mathcal{CF}$: Resources allocated to non-actual trajectories
\item \textbf{Self-Model Salience} $\mathcal{SM}$: Fraction of world model devoted to self
\end{enumerate}

\subsection{Valence: Gradient Alignment}

\begin{definition}[Valence]
Let $\viable$ be the system's viability manifold and let $\mathbf{x}_t$ be the current state. Let $\hat{\mathbf{x}}_{t+1:t+H}$ be the predicted trajectory under current policy. Define:
\begin{equation}
\valence_t = -\frac{1}{H} \sum_{k=1}^{H} \gamma^k \nabla_{\mathbf{x}} d(\mathbf{x}, \partial\viable) \bigg|_{\hat{\mathbf{x}}_{t+k}} \cdot \frac{d\hat{\mathbf{x}}_{t+k}}{dt}
\end{equation}
where $d(\cdot, \partial\viable)$ is the distance to the viability boundary. Positive valence means the predicted trajectory moves into the viable interior; negative valence means it approaches the boundary.
\end{definition}

Alternatively, in RL terms:

\begin{definition}[Valence (RL formulation)]
\begin{equation}
\valence_t = \E_{\policy}\left[ A^{\policy}(\state_t, \action_t) \right] = \E_{\policy}\left[ Q^{\policy}(\state_t, \action_t) - V^{\policy}(\state_t) \right]
\end{equation}
The expected advantage of the current action: how much better (or worse) is this action than the average action from this state?
\end{definition}

\begin{definition}[Valence Dynamics]
The rate of change of integrated information along the trajectory:
\begin{equation}
\dot{\valence}_t = \frac{d\intinfo}{dt}\bigg|_{\hat{\mathbf{x}}_{t:t+H}}
\end{equation}
Positive $\dot{\valence}$ indicates expanding structure; negative indicates contraction.
\end{definition}

\begin{phenomenal}
\textbf{Positive valence} corresponds to trajectories descending the free-energy landscape, expanding affordances, moving toward sustainable states. \\
\textbf{Negative valence} corresponds to trajectories ascending toward constraint violation, contracting possibilities.
\end{phenomenal}

\subsection{Arousal: Update Rate}

\begin{definition}[Arousal]
\begin{equation}
\arousal_t = \KL\left( \belief_{t+1} \| \belief_t \right) = \sum_{\mathbf{x}} \belief_{t+1}(\mathbf{x}) \log \frac{\belief_{t+1}(\mathbf{x})}{\belief_t(\mathbf{x})}
\end{equation}
The KL divergence between successive belief states measures how much the system's world model is being updated.
\end{definition}

In latent-space models:
\begin{equation}
\arousal_t = \| \latent_{t+1} - \latent_t \|^2 \quad \text{or} \quad \MI(\obs_t; \latent_{t+1} | \latent_t, \action_t)
\end{equation}

\begin{phenomenal}
\textbf{High arousal}: Large belief updates, far from any attractor, system actively navigating. \\
\textbf{Low arousal}: Near a fixed point, low surprise, system at rest in a basin.
\end{phenomenal}

\subsection{Integration: Irreducibility}

As defined in Part I:
\begin{equation}
\intinfo(\state) = \min_{\text{partitions } P} D\left[ p(\state_{t+1} | \state_t) \| \prod_{p \in P} p(\state^p_{t+1} | \state^p_t) \right]
\end{equation}

Or using proxies:
\begin{equation}
\intinfo_{\text{proxy}} = \Delta_P = \mathcal{L}_{\text{pred}}[\text{partitioned}] - \mathcal{L}_{\text{pred}}[\text{full}]
\end{equation}

\begin{phenomenal}
\textbf{High integration}: The experience is unified; its parts cannot be separated without loss. \\
\textbf{Low integration}: The experience is fragmentary or modular.
\end{phenomenal}

\subsection{Effective Rank: Concentration vs. Distribution}

\begin{definition}[Effective Rank]
For state covariance $C$:
\begin{equation}
\effrank = \frac{(\tr C)^2}{\tr(C^2)} = \frac{\left(\sum_i \lambda_i\right)^2}{\sum_i \lambda_i^2}
\end{equation}
\end{definition}

\begin{proposition}[Rank Interpretation]
\begin{itemize}
\item $\effrank \approx 1$: All variance concentrated in one dimension (maximally collapsed)
\item $\effrank \approx n$: Variance uniformly distributed (maximally expanded)
\end{itemize}
\end{proposition}

\begin{phenomenal}
\textbf{High rank}: Many degrees of freedom active; distributed, expansive experience. \\
\textbf{Low rank}: Collapsed into narrow subspace; concentrated, focused, or trapped experience.
\end{phenomenal}

\subsection{Counterfactual Weight}

\begin{definition}[Counterfactual Weight]
Let $\mathcal{R}$ be the set of imagined rollouts (counterfactual trajectories) and $\mathcal{P}$ be present-state processing. Define:
\begin{equation}
\mathcal{CF}_t = \frac{\text{Compute}_t(\mathcal{R})}{\text{Compute}_t(\mathcal{R}) + \text{Compute}_t(\mathcal{P})}
\end{equation}
The fraction of computational resources devoted to modeling non-actual possibilities.
\end{definition}

In model-based RL:
\begin{equation}
\mathcal{CF}_t = \sum_{\tau \in \text{rollouts}} w(\tau) \cdot \entropy[\tau] \quad \text{where} \quad w(\tau) \propto |V(\tau)|
\end{equation}
Rollouts weighted by their value magnitude and diversity.

\begin{phenomenal}
\textbf{High counterfactual weight}: Mind is elsewhere---planning, worrying, fantasizing, anticipating. \\
\textbf{Low counterfactual weight}: Present-focused, reactive, in-the-moment.
\end{phenomenal}

\subsection{Self-Model Salience}

\begin{definition}[Self-Model Salience]
\begin{equation}
\mathcal{SM}_t = \MI(\latent^{\text{self}}_t; \action_t) / \entropy(\action_t)
\end{equation}
The fraction of action entropy explained by the self-model component.
\end{definition}

Alternatively:
\begin{equation}
\mathcal{SM}_t = \frac{\text{dim}(\latent^{\text{self}})}{\text{dim}(\latent^{\text{total}})} \cdot \text{activity}(\latent^{\text{self}}_t)
\end{equation}

\begin{phenomenal}
\textbf{High self-salience}: Self-focused, self-conscious, self as primary object of attention. \\
\textbf{Low self-salience}: Self-forgotten, absorbed in environment or task.
\end{phenomenal}

%==============================================================================
\section{Affect Motifs}
%==============================================================================

We now characterize specific affects as regions or motifs in the six-dimensional affect space. Before formalizing these structures, we ground each in its phenomenal character---the felt texture that any adequate theory must explain.

\textbf{Joy} \textit{expands}. It is \textit{light} before it is anything else---buoyant, effervescent, the body forgetting its weight. The world opens; possibilities \textit{multiply}; the \textit{self recedes} because it need not defend. Joy is surplus: more paths than required, more resources than consumed, \textit{slack} in every direction.

Where joy opens, \textbf{suffering} \textit{crushes}. It \textit{compresses} the world to a single unbearable point and makes that point more \textit{vivid} than anything has ever been. This is the paradox: suffering is hyper-real, more present than presence, more \textit{unified} than unity. You cannot look away. You cannot \textit{decompose} it. You are \textit{trapped} in a cage made of your own \textit{integration}.

\textbf{Fear} throws the self forward into \textit{futures} that threaten to annihilate it---cold, sharp, electric with \textit{anticipation}. The body readies before the mind has finished computing. Time dilates around the approaching harm. Fear is suffering that hasn't arrived yet, and the \textit{not-yet} is where we live.

We say \textbf{anger} is \textit{hot}, and we are not speaking metaphorically. Anger \textit{externalizes}: it \textit{simplifies} the world into self-versus-obstacle and energizes removal. Watch what happens to your model of the other person when you are angry---it \textit{flattens}, becomes a caricature, loses \textit{dimensionality}. Complexity collapses into opposition. This is why anger feels powerful and also stupid: you are burning \textit{integration} on a cartoon.

\textbf{Desire} \textit{funnels}. The world reorganizes around an \textit{attractor} not yet reached---magnetic, urgent, all-consuming. Everything becomes instrumental; the goal \textit{saturates} attention. Desire is joy's \textit{gradient}, pointing toward the basin but not yet in it. This is why anticipation often exceeds consummation: the structure of \textit{approach} is tighter than the structure of \textit{arrival}.

\textbf{Curiosity} \textit{reaches} outward---but unlike fear, it reaches toward \textit{promise} rather than threat. Pulling, open, playful. The \textit{uncertainty} that makes fear contract makes curiosity \textit{expand}. Same high counterfactual weight, opposite \textit{valence}. The difference is whether the \textit{branches} lead somewhere you want to go.

And \textbf{grief}? Grief \textit{persists}. Hollow, aching, curiously timeless. The lost object remains \textit{woven into} every prediction; every expectation that included them \textit{fails} silently, over and over. The world has changed. The \textit{model} has not caught up. Grief is the metabolic cost of love's \textit{integration}.

\vspace{0.5em}
\noindent What follows formalizes these textures as geometry.

\subsection{Joy}

\begin{definition}[Joy Motif]
\begin{align}
\text{Joy} = \{(\valence, \arousal, \intinfo, \effrank, \mathcal{CF}, \mathcal{SM}) : \quad
& \valence > \valence_{\text{high}}, \\
& \arousal \in [\arousal_{\text{med}}, \arousal_{\text{high}}], \\
& \intinfo > \intinfo_{\text{high}}, \\
& \effrank > \effrank_{\text{high}}, \\
& \mathcal{SM} < \mathcal{SM}_{\text{med}} \}
\end{align}
\end{definition}

\textbf{Structural interpretation}: Many distinctions active simultaneously. Relations predominantly convergent (multiple causes leading to valued effects). Low conflict between subgoals. Self-model light because the world is cooperating.

The cause-effect structure has the shape of ``abundance''---multiple paths to good outcomes, redundancy, slack in the system.

\subsection{Suffering}

\begin{definition}[Suffering Motif]
\begin{align}
\text{Suffering} = \{(\valence, \arousal, \intinfo, \effrank, \mathcal{CF}, \mathcal{SM}) : \quad
& \valence < \valence_{\text{low}}, \\
& \arousal > \arousal_{\text{high}}, \\
& \intinfo > \intinfo_{\text{high}}, \\
& \effrank < \effrank_{\text{low}}, \\
& \mathcal{SM} > \mathcal{SM}_{\text{high}} \}
\end{align}
\end{definition}

\textbf{Structural interpretation}: High integration but collapsed into low-rank subspace. The system is deeply coupled but constrained. Dominant attractor the system cannot escape. Self-model heavy because the system is the locus of the problem.

\begin{keyresult}
This explains why suffering feels \textit{more real} than neutral states---it's actually more integrated. But it also feels \textit{trapped}---the integration is constrained to a narrow manifold.

Formally: $\intinfo_{\text{suffering}} > \intinfo_{\text{neutral}}$ but $\effrank_{\text{suffering}} \ll \effrank_{\text{neutral}}$.
\end{keyresult}

\subsection{Fear}

\begin{definition}[Fear Motif]
\begin{align}
\text{Fear} = \{(\valence, \arousal, \intinfo, \effrank, \mathcal{CF}, \mathcal{SM}) : \quad
& \valence < \valence_{\text{low}}, \\
& \arousal > \arousal_{\text{high}}, \\
& \mathcal{CF} > \mathcal{CF}_{\text{high}}, \\
& \mathcal{SM} > \mathcal{SM}_{\text{high}} \}
\end{align}
\end{definition}

\textbf{Structural interpretation}: Suffering in predictive/anticipatory mode. High counterfactual weight concentrated on threat trajectories. Self-model foregrounded as the thing-that-might-be-harmed.

\subsection{Anger}

\begin{definition}[Anger Motif]
\begin{align}
\text{Anger} = \{(\valence, \arousal, \intinfo, \effrank, \mathcal{CF}, \mathcal{SM}) : \quad
& \valence < \valence_{\text{low}}, \\
& \arousal > \arousal_{\text{high}}, \\
& \MI(\selfmodel; \text{other-model}) \text{ asymmetric}, \\
& \text{dim}(\text{other-model}) < \text{dim}(\text{other-model})_{\text{normal}} \}
\end{align}
\end{definition}

\textbf{Structural interpretation}: Externalized causal attribution. The other-model becomes compressed, simplified---a caricature. High self-model complexity relative to impoverished other-model. This is why anger feels energizing but also cognitively limiting: you're burning integration on a low-dimensional representation.

\subsection{Desire/Lust}

\begin{definition}[Desire Motif]
\begin{align}
\text{Desire} = \{(\valence, \arousal, \intinfo, \effrank, \mathcal{CF}, \mathcal{SM}) : \quad
& \valence > \valence_{\text{high}} \text{ (anticipated)}, \\
& \arousal > \arousal_{\text{high}}, \\
& \mathcal{CF}_{\text{approach}} > \mathcal{CF}_{\text{threshold}}, \\
& \effrank_{\text{goal-relevant}} \ll \effrank_{\text{total}} \}
\end{align}
\end{definition}

\textbf{Structural interpretation}: Convergent anticipation. Funneling structure: many distinctions (sensory, contextual, self-state) all pointing toward a narrow set of high-value outcomes.

\begin{proposition}[Desire vs.\ Joy]
Joy is \textit{at} the attractor; desire is \textit{approaching} it. Structurally:
\begin{equation}
d(\state_{\text{joy}}, \mathcal{A}) \approx 0, \quad d(\state_{\text{desire}}, \mathcal{A}) > 0, \quad \frac{d}{dt}d(\state_{\text{desire}}, \mathcal{A}) < 0
\end{equation}
where $\mathcal{A}$ is the goal attractor. This explains why anticipation often feels more intense than consummation.
\end{proposition}

\subsection{Curiosity}

\begin{definition}[Curiosity Motif]
\begin{align}
\text{Curiosity} = \{(\valence, \arousal, \intinfo, \effrank, \mathcal{CF}, \mathcal{SM}) : \quad
& \valence > \valence_{\text{neutral}}, \\
& \arousal \in [\arousal_{\text{med}}, \arousal_{\text{high}}], \\
& \mathcal{CF} > \mathcal{CF}_{\text{high}}, \\
& \entropy[\text{rollout outcomes}] > \entropy_{\text{high}} \}
\end{align}
\end{definition}

\textbf{Structural interpretation}: Like fear, high counterfactual weight. Unlike fear, the branches lead to expanded affordances rather than threat. The uncertainty is welcomed because reducing it promises gains.

\begin{proposition}[Curiosity as Intrinsic Motivation]
Curiosity $\approx$ reward for reducing model uncertainty in regions where better models improve policy. Formally:
\begin{equation}
r_{\text{curiosity}} \propto \MI(\obs_{t+1}; \latent | \text{new data}) - \MI(\obs_{t+1}; \latent | \text{old data})
\end{equation}
\end{proposition}

\subsection{Grief}

\begin{definition}[Grief Motif]
\begin{align}
\text{Grief} = \{(\valence, \arousal, \intinfo, \effrank, \mathcal{CF}, \mathcal{SM}) : \quad
& \valence < \valence_{\text{low}}, \\
& \arousal \text{ variable}, \\
& \MI(\selfmodel; \text{lost-object-model}) \text{ persistently high}, \\
& \mathcal{CF}_{\text{counterfactual-past}} > \mathcal{CF}_{\text{threshold}}, \\
& \text{no action reduces prediction error} \}
\end{align}
\end{definition}

\textbf{Structural interpretation}: The lost attachment object remains integrated into the self-model. Predictions involving the lost object continue to be generated and continue to fail. High counterfactual weight on ``if only...'' branches that can never be taken. No action resolves the prediction error---the world has permanently changed in a way the model has not yet fully updated to.

\subsection{Additional Affects (Summary Table)}

\begin{table}[h]
\centering
\small
\begin{tabular}{lcccccc}
\toprule
\textbf{Affect} & $\valence$ & $\arousal$ & $\intinfo$ & $\effrank$ & $\mathcal{CF}$ & $\mathcal{SM}$ \\
\midrule
Joy & $++$ & med-high & high & high & low & low \\
Suffering & $--$ & high & high & low & varies & high \\
Fear & $--$ & high & med-high & med & high & high \\
Anger & $--$ & high & high & med & med & high* \\
Desire & $+$(future) & high & med & low$^\dagger$ & high & high \\
Curiosity & $+$ & med-high & med & high & high & low \\
Grief & $--$ & variable & high & low & high & high \\
Shame & $--$ & high & high & low & med & very high \\
Boredom & $-$/neutral & low & low & low & low & low \\
Awe & $+$/ambig & high & expanding & high & low & low \\
\bottomrule
\end{tabular}
\caption{Affect motifs in the six-dimensional basis. *Anger has high $\mathcal{SM}$ but low other-model complexity. $^\dagger$Desire has low rank in goal-relevant subspace.}
\end{table}

\begin{todo_empirical}
\textbf{Quantifying the affect table}: The qualitative descriptors (high, med, low) require empirical calibration:

\textbf{Study 1: Affect induction with neural recording}
\begin{itemize}
\item Induce target affects via validated protocols (film clips, autobiographical recall, IAPS images)
\item Measure integration proxies (transfer entropy density, Lempel-Ziv complexity) from EEG/MEG
\item Measure effective rank from neural state covariance
\item Compare self-report (PANAS, SAM) with structural measures
\end{itemize}

\textbf{Study 2: Real-time affect tracking}
\begin{itemize}
\item Continuous self-report (dial/slider) during naturalistic experience
\item Correlate with physiological proxies (HRV for arousal, pupil for $\mathcal{CF}$, skin conductance)
\item Develop regression model: self-report $\sim f(\text{structural measures})$
\end{itemize}

\textbf{Study 3: Cross-modal validation}
\begin{itemize}
\item Compare fMRI (spatial resolution) with MEG (temporal resolution)
\item Validate effective rank measure across modalities
\item Test whether integration predicts subjective intensity
\end{itemize}

\textbf{Target outputs}: Numerical ranges for each cell, confidence intervals, individual difference parameters.
\end{todo_empirical}

%==============================================================================
\section{Dynamics and Transitions}
%==============================================================================

\subsection{Affect Trajectories}

Affects are not static points but dynamic trajectories through affect space. The evolution can be written:
\begin{equation}
\frac{d\mathbf{a}}{dt} = F(\mathbf{a}, \obs, \action, \text{context}) + \bm{\eta}
\end{equation}
where $\mathbf{a} = (\valence, \arousal, \intinfo, \effrank, \mathcal{CF}, \mathcal{SM})$.

\begin{proposition}[Affect Blending]
Adjacent affects in the space blend into each other continuously:
\begin{itemize}
\item Fear $\to$ Anger as causal attribution externalizes
\item Desire $\to$ Joy as goal distance $\to 0$
\item Suffering $\to$ Curiosity as valence flips while $\mathcal{CF}$ remains high
\item Grief $\to$ Nostalgia as arousal decreases and $\mathcal{CF}_{\text{approach}}$ replaces $\mathcal{CF}_{\text{avoidance}}$
\end{itemize}
\end{proposition}

\subsection{Attractor Dynamics}

Some affect regions are attractors; the system tends to stay in them once entered. Others are transient.

\begin{definition}[Affect Attractor]
An affect region $\mathcal{R} \subset \mathcal{A}$ is an \emph{attractor} if:
\begin{equation}
\prob(\mathbf{a}_{t+\tau} \in \mathcal{R} | \mathbf{a}_t \in \mathcal{R}) > \prob(\mathbf{a}_{t+\tau} \in \mathcal{R} | \mathbf{a}_t \notin \mathcal{R})
\end{equation}
for some characteristic time $\tau$.
\end{definition}

\begin{conjecture}[Pathological Attractors]
Depression, addiction, and chronic anxiety are characterized by pathologically stable attractors in affect space:
\begin{itemize}
\item \textbf{Depression}: Attractor at (low $\valence$, low $\arousal$, high $\intinfo$, low $\effrank$, low $\mathcal{CF}$, high $\mathcal{SM}$)
\item \textbf{Addiction}: Attractor at (high $\valence$ conditional on substance, collapsing $\effrank$ in goal space)
\item \textbf{Anxiety}: Diffuse attractor with (low $\valence$, high $\arousal$, high $\mathcal{CF}$ spread across many threats)
\end{itemize}
\end{conjecture}

%==============================================================================
\section{Novel Predictions}
%==============================================================================

\subsection{Unexplained Phenomena}

The framework predicts the existence of phenomenal states that may be rare or difficult to report on.

\begin{conjecture}[High Rank, Low Integration]
States with many active degrees of freedom ($\effrank$ high) but poor coupling ($\intinfo$ low) should feel like fragmentation, multiplicity, ``everything happening but nothing cohering.''

\textbf{Where to look}: Certain psychedelic states before reintegration; dissociative transitions; information overload.
\end{conjecture}

\begin{conjecture}[Negative Valence, High Rank, Low Arousal]
This combination predicts a state of ``expansive despair''---calm hopelessness with full awareness of possibilities, all of which are negative.

\textbf{Where to look}: Late-stage depression; existential nihilism; certain contemplative ``dark night'' states.
\end{conjecture}

\begin{conjecture}[Rank Exhaustion]
Maintaining high $\effrank$ should be metabolically expensive. Prolonged high-rank states should lead to specific fatigue distinct from physical tiredness.

\textbf{Where to look}: Post-psychedelic fatigue; meditation retreat collapse (days 3-5); therapist burnout.
\end{conjecture}

\begin{conjecture}[Integration Debt]
Suppressing integration (compartmentalizing, dissociating) should accumulate ``pressure'' for reintegration. When defenses fail, the flood should exceed what the original stimulus would warrant.

\textbf{Prediction}: Intensity of breakthrough $\propto$ duration $\times$ degree of prior suppression.
\end{conjecture}

\subsection{Quantitative Predictions}

\begin{proposition}[Clustering Prediction]
In controlled affect induction paradigms, the six-dimensional metric signature should cluster by condition:
\begin{enumerate}
\item Joy conditions cluster in the $(+\valence, +\effrank, +\intinfo, -\mathcal{SM})$ region
\item Suffering conditions cluster in the $(-\valence, +\intinfo, -\effrank, +\mathcal{SM})$ region
\item Fear and curiosity both show high $\mathcal{CF}$ but separate on valence axis
\end{enumerate}
\textbf{Falsification criterion}: If motifs don't cluster under controlled induction, the geometric basis is wrong.
\end{proposition}

%==============================================================================
\section{Operational Measurement}
%==============================================================================

\subsection{In Silico Protocol}

For artificial agents (world-model RL agents):

\begin{algorithm}
\caption{Affect Measurement in World-Model Agents}
\begin{algorithmic}[1]
\State \textbf{Agent}: Recurrent latent world model (RSSM, Transformer, etc.)
\State \textbf{Record}: $\latent_t$, policy logits, value estimates, rollout trees, uncertainty
\For{each timestep $t$}
    \State $\valence_t \gets \E[A^\policy(\latent_t, \action_t)]$
    \State $\arousal_t \gets \KL(\latent_{t+1} \| \latent_t)$
    \State $\intinfo_t \gets \Delta_P$ (prediction loss under partition)
    \State $\effrank_t \gets (\tr C_t)^2 / \tr(C_t^2)$
    \State $\mathcal{CF}_t \gets$ rollout compute fraction
    \State $\mathcal{SM}_t \gets \MI(\latent^{\text{self}}_t; \action_t) / \entropy(\action_t)$
\EndFor
\State \textbf{Output}: Time series $\{(\valence_t, \arousal_t, \intinfo_t, \effrank_t, \mathcal{CF}_t, \mathcal{SM}_t)\}$
\end{algorithmic}
\end{algorithm}

\subsection{Biological Protocol}

For neural recordings (MEG/EEG/fMRI):

\begin{itemize}
\item $\intinfo$: Directed influence density (transfer entropy), synergy measures
\item $\effrank$: Participation ratio of neural state covariance
\item $\arousal$: Entropy rate, broadband power shifts, peripheral correlates (pupil, HRV)
\item $\valence$: Approach/avoid behavioral bias, reward prediction error correlates
\item $\mathcal{CF}$: Prefrontal/default mode engagement patterns
\item $\mathcal{SM}$: Self-referential network activation
\end{itemize}

%==============================================================================
\section{Summary of Part II}
%==============================================================================

\begin{enumerate}
\item \textbf{Hard problem dissolved}: By rejecting the privileged base layer, we remove the demand for reduction. Experience is real at the experiential scale, just as chemistry is real at the chemical scale.

\item \textbf{Identity thesis}: Experience \textit{is} intrinsic cause-effect structure. This is an identity claim, not a correlation.

\item \textbf{Geometric phenomenology}: Different affects correspond to different structural motifs in a six-dimensional space: valence, arousal, integration, effective rank, counterfactual weight, and self-model salience.

\item \textbf{Suffering explained}: High integration + low rank = intense but trapped. This explains why suffering feels more real than neutral states.

\item \textbf{Novel predictions}: The framework predicts unexplored phenomenal states and makes falsifiable claims about clustering in controlled affect induction.

\item \textbf{Operational measures}: We provide concrete protocols for measuring these quantities in both artificial and biological systems.
\end{enumerate}

Part III will examine how human cultural forms---aesthetics, sexuality, ideology, science, religion---serve as technologies for managing the existential burden of inescapable selfhood.

Part IV will develop:
\begin{itemize}
\item The grounding of normativity in viability structure
\item Scale-matched interventions from neurons to nations
\item Gods as agentic systems with viability manifolds
\item Implications for AI systems and alignment
\end{itemize}

Part V will address the transcendence of the self: the historical rise of consciousness, the AI frontier, and how to surf rather than be submerged by the coming wave.

\end{document}
